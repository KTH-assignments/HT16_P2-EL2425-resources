\documentclass[oneside,12pt]{article}
\usepackage{fontspec}
\usepackage{lmodern}
%\setmainfont{cmr12}
\defaultfontfeatures{Ligatures=TeX} % To support LaTeX quoting style
\usepackage{amsmath}
\usepackage{lscape}
\usepackage{graphicx}
\usepackage{pgfplots}
\usepackage{subcaption}
\usepackage[margin=1in]{geometry}
\usepackage{float}
\usepackage{url}
\usepackage[hidelinks]{hyperref}
\usepackage{pstricks}
\usepackage{steinmetz}

\widowpenalty=1000
\clubpenalty=1000


\title{EL2425 - Slip Control \\ Meeting agenda 2016-11-16}
%\date{}

\begin{document}
\maketitle

\section{Done}

\begin{itemize}

  \item \textbf{Microcontroller messaging}. Teensy was eventually flushed with
    new firmware that allows only messages of type
    \texttt{slip\_control\_communications/input\_drive} to be passed to it.
    (Previously, a message type of \texttt{f1tenth\_msgs/drive\_values}
    was hardcoded into Arduino's firmware, making communication within package
    \texttt{slip\_control} impossible due to incompatibility.)

  \item \textbf{Time constant} pertaining to the velocity response of the vehicle
    found, although significant differences between time constants
    for different velocity references have been found.

  \item \textbf{ROS infrastructure} set.

  \item \textbf{MPC python package} found: \texttt{cvxopt}. It seems to be able to
    capture the essence of what our goal is.
    See \url{http://nbviewer.jupyter.org/github/cvxgrp/cvx_short_course/blob/master/intro/control.ipynb}

  \item Theoretical solution involving PID and traveling on the centerline of a
    lane found, and code for it written. However, the involved gains
    have to be tuned experimentally.

  \item Theoretical solution involving MPC and tracking the centerline of a lane
    found, however code for it has not been written yet.

\end{itemize}



\section{Ongoing}

  \subsection{Theoretical solution involving PID and tracking the centerline of a lane}

    The problem can be decomposed into two separate and independent
    components involving a translation and a rotation of the vehicle.


    \textbf{Translational component}
    We assume that the pose of the vehicle at time $t$ is
    $(x_c, y_c, v_c, \psi_v)$ and that two range scans at $+90^\circ$ and
    $-90^\circ$ with respect to the longitudinal axis of the
    vehicle are available, which are denoted as $CL$ and $CR$ respectively. Also,
    in figure \ref{fig:centerline_pid_translation}, point $O'$ is a future
    reference and the angle $\lambda$ is the angle that the vehicle should
    make in order to reach $O'$ in a future moment.

    Then, since $CL + CR = L = OL + OR$ and $CL = OC + OL$, this means that

    \begin{equation}
      OC = OR - CR = \dfrac{L}{2} - CR = \dfrac{CL + CR}{2} - CR = \dfrac{CL-CR}{2}
    \end{equation}
    where $L$ is the width of the lane whose centerline the vehicle is to track.

    Furthermore, in the $CC'O'$ triangle

    \begin{align}
      \text{tan}\lambda &= \dfrac{O'C'}{CC'} = \dfrac{OC}{CC'} = \dfrac{CL-CR}{2CC'} \\
      \lambda &= \text{tan}^{-1}\dfrac{CL-CR}{2CC'}
    \end{align}


    \begin{figure}[H]\centering
      \scalebox{1}{% Generated with LaTeXDraw 2.0.8
% Thu Nov 17 16:22:33 CET 2016
% \usepackage[usenames,dvipsnames]{pstricks}
% \usepackage{epsfig}
% \usepackage{pst-grad} % For gradients
% \usepackage{pst-plot} % For axes
\scalebox{1} % Change this value to rescale the drawing.
{
\begin{pspicture}(0,-4.77)(10.711719,4.77)
\psline[linewidth=0.04cm](0.0,-3.19)(4.24,3.25)
\psline[linewidth=0.04cm](5.84,-4.75)(10.08,1.69)
\psline[linewidth=0.012cm,linestyle=dashed,dash=0.16cm 0.16cm](3.26,-3.45)(7.5,2.99)
\psline[linewidth=0.04cm,arrowsize=0.05291667cm 2.0,arrowlength=1.4,arrowinset=0.4]{->}(6.94,-0.73)(9.34,2.91)
\usefont{T1}{ptm}{m}{n}
\rput(2.5671873,1.86){$L$}
\usefont{T1}{ptm}{m}{n}
\rput(8.297188,-1.8){$R$}
\psdots[dotsize=0.012](8.08,-1.05)
\psdots[dotsize=0.012](3.46,1.81)
\psline[linewidth=0.04cm](8.04,-1.41)(3.18,1.63)
\psdots[dotsize=0.12](5.6,0.13)
\usefont{T1}{ptm}{m}{n}
\rput(5.0181246,0.04){$O$}
\psline[linewidth=0.04cm](5.78,0.39)(5.5,0.59)
\psline[linewidth=0.04cm](5.5,0.59)(5.3,0.31)
\psdots[dotsize=0.066](5.54,0.37)
\psline[linewidth=0.04cm](8.2,-1.13)(7.92,-0.93)
\psline[linewidth=0.04cm](7.92,-0.93)(7.72,-1.21)
\psdots[dotsize=0.066](7.98,-1.17)
\psline[linewidth=0.04cm](7.12,-0.45)(6.84,-0.25)
\psline[linewidth=0.04cm](6.84,-0.25)(6.64,-0.53)
\psdots[dotsize=0.066](6.88,-0.47)
\psline[linewidth=0.04cm](9.88,1.41)(5.02,4.45)
\psline[linewidth=0.04cm](3.98,2.85)(5.2,4.75)
\usefont{T1}{ptm}{m}{n}
\rput(4.2671876,4.48){$L'$}
\usefont{T1}{ptm}{m}{n}
\rput(10.277187,1.04){$R'$}
\usefont{T1}{ptm}{m}{n}
\rput(7.3071876,3.32){$O'$}
\usefont{T1}{ptm}{m}{n}
\rput(9.317187,2.18){$C'$}
\psline[linewidth=0.01cm](6.96,-0.75)(7.48,2.95)
\psline[linewidth=0.04cm](10.06,1.69)(9.78,1.89)
\psline[linewidth=0.04cm](9.78,1.89)(9.58,1.61)
\psdots[dotsize=0.066](9.82,1.65)
\psarc[linewidth=0.02](7.32,0.27){0.22}{336.80142}{153.43495}
\usefont{T1}{ptm}{m}{n}
\rput(7.3835936,0.72){$\lambda$}
\usefont{T1}{ptm}{m}{n}
\rput(6.4571877,-1.04){$C$}
\end{pspicture} 
}

}
      \caption{The vehicle's heading angle is equal to that of the lane's.
        However, its position is off track.}
      \label{fig:centerline_pid_translation}
    \end{figure}

    \textbf{Rotational component}


    We assume that the the pose of the vehicle at time $t$ as
    $(x_c, y_c, v_c, \psi_v)$ and that three range scans
    at $+90^\circ$, $0^\circ$ and $-90^\circ$ with respect to the
    longitudinal axis of the vehicle are available, which are denoted as
    $CL$, $CF$ and $CR$ respectively. Then, the heading angle error is

    \begin{align}
      \phi = \dfrac{\pi}{2} - \text{tan}^{-1}\dfrac{CF}{CR}
    \end{align}

    since $\mu + \phi + \dfrac{\pi}{2} = \pi$ and $\text{tan}(\mu) = \dfrac{CF}{CR}$.

    \begin{figure}[H]\centering
      \scalebox{1}{% Generated with LaTeXDraw 2.0.8
% Tue Nov 15 16:19:30 CET 2016
% \usepackage[usenames,dvipsnames]{pstricks}
% \usepackage{epsfig}
% \usepackage{pst-grad} % For gradients
% \usepackage{pst-plot} % For axes
\scalebox{1} % Change this value to rescale the drawing.
{
\begin{pspicture}(0,-4.53)(11.419063,4.53)
\psline[linewidth=0.04cm](0.0,-2.95)(4.24,3.49)
\psline[linewidth=0.04cm](5.84,-4.51)(10.08,1.93)
\psdots[dotsize=0.12](5.76,0.59)
\psline[linewidth=0.012cm,linestyle=dashed,dash=0.16cm 0.16cm](3.26,-3.21)(7.5,3.23)
\psline[linewidth=0.04cm,arrowsize=0.05291667cm 2.0,arrowlength=1.4,arrowinset=0.4]{->}(5.76,0.57)(10.62,2.73)
\psline[linewidth=0.012cm,arrowsize=0.05291667cm 2.0,arrowlength=1.4,arrowinset=0.4]{<-}(5.76,2.59)(5.76,-1.41)
\psline[linewidth=0.012cm,arrowsize=0.05291667cm 2.0,arrowlength=1.4,arrowinset=0.4]{->}(3.76,0.59)(7.76,0.59)
\psarc[linewidth=0.02](6.18,0.59){0.26}{0.0}{60.0}
\usefont{T1}{ptm}{m}{n}
\rput(5.4045315,2.32){$y$}
\usefont{T1}{ptm}{m}{n}
\rput(7.5045314,0.26){$x$}
\usefont{T1}{ptm}{m}{n}
\rput(7.1545315,0.82){$\psi_c$}
\psarc[linewidth=0.02](6.11,0.9){0.21}{336.80142}{93.81407}
\usefont{T1}{ptm}{m}{n}
\rput(6.664531,1.3){$\phi$}
\psline[linewidth=0.04cm,arrowsize=0.05291667cm 2.0,arrowlength=1.4,arrowinset=0.4]{->}(5.76,0.59)(4.48,3.89)
\psline[linewidth=0.04cm,arrowsize=0.05291667cm 2.0,arrowlength=1.4,arrowinset=0.4]{->}(5.76,0.59)(7.08,-2.65)
\usefont{T1}{ptm}{m}{n}
\rput(5.324531,0.36){$C$}
\usefont{T1}{ptm}{m}{n}
\rput(3.8545313,3.74){$L$}
\usefont{T1}{ptm}{m}{n}
\rput(7.4445314,-2.88){$R$}
\psline[linewidth=0.04cm](9.8,1.51)(10.84,3.07)
\usefont{T1}{ptm}{m}{n}
\rput(11.054531,2.7){$F$}
\psarc[linewidth=0.02](10.13,2.34){0.21}{154.44003}{306.25385}
\usefont{T1}{ptm}{m}{n}
\rput(9.8645315,2.02){$\phi$}
\psdots[dotsize=0.012](8.08,-0.81)
\psdots[dotsize=0.012](3.46,2.05)
\psline[linewidth=0.04cm](3.78,2.79)(4.9,4.51)
\psarc[linewidth=0.02](7.11,-2.24){0.21}{0.0}{180.0}
\usefont{T1}{ptm}{m}{n}
\rput(7.1745315,-1.72){$\mu$}
\psline[linewidth=0.04cm](6.0,0.67)(6.12,0.43)
\psline[linewidth=0.04cm](6.12,0.45)(5.86,0.31)
\psdots[dotsize=0.034](5.94,0.51)
\end{pspicture} 
}

}
      \caption{The vehicle's position and its reference are equal. However,
        the vehicle's heading is off track.}
      \label{}
    \end{figure}

    The overall angle error of the vehicle is then

    \begin{align}
      e &= \text{tan}^{-1}\dfrac{CL-CR}{2CC'} + \dfrac{\pi}{2} - \text{tan}^{-1}\dfrac{CF}{CR}
    \end{align}

    Here, the length $CC'$ is unknown and can be set beforehand. Its magnitude
    will determine the vehicle's rate of convergence to the centerline of the
    lane.

    Since the input to the plant is in terms of angular displacement, this
    is in fact the error that the PID controller can include and utilize in
    order to determine the plant's input.

    The angular input to the vehicle will then be

    \begin{align}
      \delta = K_p \cdot e + K_d \cdot \dfrac{de}{dt} + K_i \cdot \int e dt
    \end{align}




  \subsection{Theoretical solution involving MPC and tracking the centerline of a lane}

    The problem can be decomposed into two separate and independent
    components involving a translation and a rotation of the vehicle.


    \textbf{Translational component} Given the pose of the vehicle at time $t$
    as $(x_c, y_c, v_c, \psi_v)$ and two range scans at $+90^\circ$ and
    $-90^\circ$ with respect to the longitudinal axis of the vehicle which are
    denoted as $CL$ and $CR$ respectively, the error in translational terms is

    \begin{align}
      e_x &= -\dfrac{CL-CR}{2}\text{sin}\psi \\
      e_y &= \dfrac{CL-CR}{2}\text{cos}\psi
    \end{align}

    since $CL + CR = L = OL + OR$, and $CL = OC + OL$, which means that

    \begin{equation}
      OC = OR - CR = \dfrac{L}{2} - CR = \dfrac{CL + CR}{2} - CR = \dfrac{CL-CR}{2}
    \end{equation}
    where $L$ is the width of the lane whose centerline the vehicle is to track.

    Furthermore, in the $COM$ triangle:

    \begin{align}
      O'C &= OC \text{cos}\mu = OC \text{cos}(\dfrac{\pi}{2} - \psi) = OC \text{sin}\psi = \dfrac{CL-CR}{2} \text{sin}\psi\\
      O'O &= OC \text{sin}\mu = OC \text{sin}(\dfrac{\pi}{2} - \psi) = OC \text{cos}\psi = \dfrac{CL-CR}{2} \text{cos}\psi
    \end{align}

    In other words, at time $t$ the vehicle should have been at point $O(x_o, y_o)$:

    \begin{align}
      x_o &=x_c - \dfrac{CL-CR}{2}\text{sin}\psi \\
      y_o &=y_c + \dfrac{CL-CR}{2}\text{cos}\psi
    \end{align}

    \begin{figure}[H]\centering
      \scalebox{1}{% Generated with LaTeXDraw 2.0.8
% Tue Nov 15 16:54:04 CET 2016
% \usepackage[usenames,dvipsnames]{pstricks}
% \usepackage{epsfig}
% \usepackage{pst-grad} % For gradients
% \usepackage{pst-plot} % For axes
\scalebox{1} % Change this value to rescale the drawing.
{
\begin{pspicture}(0,-4.53)(10.1,4.53)
\psline[linewidth=0.04cm](0.0,-2.95)(4.24,3.49)
\psline[linewidth=0.04cm](5.84,-4.51)(10.08,1.93)
\psline[linewidth=0.012cm,linestyle=dashed,dash=0.16cm 0.16cm](3.26,-3.21)(7.5,3.23)
\psline[linewidth=0.04cm,arrowsize=0.05291667cm 2.0,arrowlength=1.4,arrowinset=0.4]{->}(6.94,-0.49)(9.34,3.15)
\psline[linewidth=0.012cm,arrowsize=0.05291667cm 2.0,arrowlength=1.4,arrowinset=0.4]{<-}(6.94,1.51)(6.94,-2.49)
\psline[linewidth=0.012cm,arrowsize=0.05291667cm 2.0,arrowlength=1.4,arrowinset=0.4]{->}(4.94,-0.49)(8.94,-0.49)
\psarc[linewidth=0.02](6.94,-0.49){0.26}{0.0}{60.0}
\usefont{T1}{ptm}{m}{n}
\rput(6.533594,1.3){$y$}
\usefont{T1}{ptm}{m}{n}
\rput(8.813594,-0.84){$x$}
\usefont{T1}{ptm}{m}{n}
\rput(7.743594,-0.2){$\psi_c$}
\usefont{T1}{ptm}{m}{n}
\rput(6.4535937,-0.72){$C$}
\usefont{T1}{ptm}{m}{n}
\rput(2.6435938,2.1){$L$}
\usefont{T1}{ptm}{m}{n}
\rput(8.373594,-1.56){$R$}
\psdots[dotsize=0.012](8.08,-0.81)
\psdots[dotsize=0.012](3.46,2.05)
\psline[linewidth=0.04cm](3.78,2.79)(4.9,4.51)
\psline[linewidth=0.04cm](8.04,-1.17)(3.18,1.87)
\psdots[dotsize=0.12](5.6,0.37)
\usefont{T1}{ptm}{m}{n}
\rput(5.094531,0.28){$O$}
\psline[linewidth=0.04cm](5.78,0.63)(5.5,0.83)
\psline[linewidth=0.04cm](5.5,0.83)(5.3,0.55)
\psdots[dotsize=0.066](5.54,0.61)
\psline[linewidth=0.04cm](8.2,-0.89)(7.92,-0.69)
\psline[linewidth=0.04cm](7.92,-0.69)(7.72,-0.97)
\psdots[dotsize=0.066](7.98,-0.93)
\psline[linewidth=0.04cm](7.12,-0.21)(6.84,-0.01)
\psline[linewidth=0.04cm](6.84,-0.01)(6.64,-0.29)
\psdots[dotsize=0.066](6.88,-0.23)
\psarc[linewidth=0.02](6.56,-0.33){0.24}{135.0}{225.0}
\usefont{T1}{ptm}{m}{n}
\rput(6.114531,-0.3){$\mu$}
\psline[linewidth=0.02cm](5.6,0.33)(5.58,-0.49)
\psdots[dotsize=0.07](5.58,-0.49)
\psdots[dotsize=0.07](5.04,-0.49)
\usefont{T1}{ptm}{m}{n}
\rput(5.6945314,-0.74){$O'$}
\usefont{T1}{ptm}{m}{n}
\rput(4.7345314,-0.76){$M$}
\end{pspicture} 
}

}
      \caption{The vehicle's heading angle is equal to that of the lane's.
        However, its position is off track.}
      \label{}
    \end{figure}

    \begin{figure}[H]\centering
      \scalebox{1}{% Generated with LaTeXDraw 2.0.8
% Tue Nov 15 17:14:40 CET 2016
% \usepackage[usenames,dvipsnames]{pstricks}
% \usepackage{epsfig}
% \usepackage{pst-grad} % For gradients
% \usepackage{pst-plot} % For axes
\scalebox{1} % Change this value to rescale the drawing.
{
\begin{pspicture}(0,-3.226)(9.08,3.226)
\psline[linewidth=0.012cm,linestyle=dashed,dash=0.16cm 0.16cm](0.0,-3.22)(4.24,3.22)
\psline[linewidth=0.04cm,arrowsize=0.05291667cm 2.0,arrowlength=1.4,arrowinset=0.4]{->}(6.66,-0.5)(9.06,3.14)
\psline[linewidth=0.012cm,arrowsize=0.05291667cm 2.0,arrowlength=1.4,arrowinset=0.4]{<-}(6.66,1.5)(6.66,-2.5)
\psline[linewidth=0.012cm,arrowsize=0.05291667cm 2.0,arrowlength=1.4,arrowinset=0.4]{->}(0.66,-0.5)(8.66,-0.5)
\psarc[linewidth=0.02](1.82,-0.5){0.26}{0.0}{60.0}
\usefont{T1}{ptm}{m}{n}
\rput(6.253594,1.29){$y$}
\psdots[dotsize=0.012](7.8,-0.82)
\psdots[dotsize=0.012](3.18,2.04)
\psline[linewidth=0.04cm](7.76,-1.18)(2.9,1.86)
\psline[linewidth=0.04cm](6.84,-0.22)(6.56,-0.02)
\psline[linewidth=0.04cm](6.56,-0.02)(6.36,-0.3)
\psdots[dotsize=0.066](6.6,-0.24)
\psarc[linewidth=0.02](6.28,-0.34){0.24}{135.0}{225.0}
\psarc[linewidth=0.02](6.7,-0.5){0.26}{0.0}{60.0}
\usefont{T1}{ptm}{m}{n}
\rput(2.6545312,-0.19){$\psi$}
\usefont{T1}{ptm}{m}{n}
\rput(7.474531,-0.29){$\psi$}
\usefont{T1}{ptm}{m}{n}
\rput(5.594531,-0.21){$\mu$}
\usefont{T1}{ptm}{m}{n}
\rput(8.604531,-0.79){$x$}
\psline[linewidth=0.04cm](3.0197916,1.3791598)(3.3004897,1.1801409)
\psline[linewidth=0.04cm](3.3004897,1.1801409)(3.4995086,1.460839)
\psdots[dotsize=0.07](3.26,1.44)
\usefont{T1}{ptm}{m}{n}
\rput(4.134531,1.93){$O$}
\usefont{T1}{ptm}{m}{n}
\rput(3.5345314,-0.87){$O'$}
\usefont{T1}{ptm}{m}{n}
\rput(1.0545312,-0.81){$M$}
\psline[linewidth=0.02cm](3.2,1.64)(3.22,-0.5)
\usefont{T1}{ptm}{m}{n}
\rput(6.264531,-0.81){$C$}
\end{pspicture} 
}

}
      \caption{}
      \label{}
    \end{figure}

    \textbf{Rotational component} With regard to rotation, given the pose of
    the vehicle at time $t$ as $(x_c, y_c, v_c, \psi_v)$ and three range scans
    at $+90^\circ$, $0^\circ$ and $-90^\circ$ with respect to the longitudinal
    axis of the vehicle which are denoted as $CL$, $CF$ and $CR$ respectively,
    the heading angle error is

    \begin{align}
      \phi = \dfrac{\pi}{2} - \text{tan}^{-1}\dfrac{CF}{CR}
    \end{align}

    since $\mu + \phi + \dfrac{\pi}{2} = \pi$ and $\text{tan}(\mu) = \dfrac{CF}{CR}$.

    In other words, at time $t$ the vehicle should have a heading angle of

    \begin{align}
      \psi_o = \psi_c + \phi = \psi_c + \dfrac{\pi}{2} - \text{tan}^{-1}\dfrac{CF}{CR}
    \end{align}


    \begin{figure}[H]\centering
      \scalebox{1}{% Generated with LaTeXDraw 2.0.8
% Tue Nov 15 16:19:30 CET 2016
% \usepackage[usenames,dvipsnames]{pstricks}
% \usepackage{epsfig}
% \usepackage{pst-grad} % For gradients
% \usepackage{pst-plot} % For axes
\scalebox{1} % Change this value to rescale the drawing.
{
\begin{pspicture}(0,-4.53)(11.419063,4.53)
\psline[linewidth=0.04cm](0.0,-2.95)(4.24,3.49)
\psline[linewidth=0.04cm](5.84,-4.51)(10.08,1.93)
\psdots[dotsize=0.12](5.76,0.59)
\psline[linewidth=0.012cm,linestyle=dashed,dash=0.16cm 0.16cm](3.26,-3.21)(7.5,3.23)
\psline[linewidth=0.04cm,arrowsize=0.05291667cm 2.0,arrowlength=1.4,arrowinset=0.4]{->}(5.76,0.57)(10.62,2.73)
\psline[linewidth=0.012cm,arrowsize=0.05291667cm 2.0,arrowlength=1.4,arrowinset=0.4]{<-}(5.76,2.59)(5.76,-1.41)
\psline[linewidth=0.012cm,arrowsize=0.05291667cm 2.0,arrowlength=1.4,arrowinset=0.4]{->}(3.76,0.59)(7.76,0.59)
\psarc[linewidth=0.02](6.18,0.59){0.26}{0.0}{60.0}
\usefont{T1}{ptm}{m}{n}
\rput(5.4045315,2.32){$y$}
\usefont{T1}{ptm}{m}{n}
\rput(7.5045314,0.26){$x$}
\usefont{T1}{ptm}{m}{n}
\rput(7.1545315,0.82){$\psi_c$}
\psarc[linewidth=0.02](6.11,0.9){0.21}{336.80142}{93.81407}
\usefont{T1}{ptm}{m}{n}
\rput(6.664531,1.3){$\phi$}
\psline[linewidth=0.04cm,arrowsize=0.05291667cm 2.0,arrowlength=1.4,arrowinset=0.4]{->}(5.76,0.59)(4.48,3.89)
\psline[linewidth=0.04cm,arrowsize=0.05291667cm 2.0,arrowlength=1.4,arrowinset=0.4]{->}(5.76,0.59)(7.08,-2.65)
\usefont{T1}{ptm}{m}{n}
\rput(5.324531,0.36){$C$}
\usefont{T1}{ptm}{m}{n}
\rput(3.8545313,3.74){$L$}
\usefont{T1}{ptm}{m}{n}
\rput(7.4445314,-2.88){$R$}
\psline[linewidth=0.04cm](9.8,1.51)(10.84,3.07)
\usefont{T1}{ptm}{m}{n}
\rput(11.054531,2.7){$F$}
\psarc[linewidth=0.02](10.13,2.34){0.21}{154.44003}{306.25385}
\usefont{T1}{ptm}{m}{n}
\rput(9.8645315,2.02){$\phi$}
\psdots[dotsize=0.012](8.08,-0.81)
\psdots[dotsize=0.012](3.46,2.05)
\psline[linewidth=0.04cm](3.78,2.79)(4.9,4.51)
\psarc[linewidth=0.02](7.11,-2.24){0.21}{0.0}{180.0}
\usefont{T1}{ptm}{m}{n}
\rput(7.1745315,-1.72){$\mu$}
\psline[linewidth=0.04cm](6.0,0.67)(6.12,0.43)
\psline[linewidth=0.04cm](6.12,0.45)(5.86,0.31)
\psdots[dotsize=0.034](5.94,0.51)
\end{pspicture} 
}

}
      \caption{The vehicle's position and its reference are equal. However,
        the vehicle's heading is off track.}
      \label{}
    \end{figure}

    Hence we can formulate the optimization problem as

    \begin{align}
      min &\sum\limits_{k=0}^N (X-X_o)^T Q (X-X_o) + U^T R U \\
      \text{subject to } & X[t+1] = A X[t] + B U[t] \\
      & -U_{min} \leq U \leq U_{max}
    \end{align}

    where $X=[x_c, y_c, v_c, \psi_c]^T$, $X_o = [x_o, y_o, v_o, \psi_o]^T$,
    $N$ is the horizon and $U=[v, \delta]^T$ is the input vector.
    Positive definite matrices $Q,R$ will have to be adjusted experimentally.

    However, the vehicle's velocity is not measurable since the vehicle does not
    have encoders connected to its wheels and MOCAP or range scans cannot
    provide measurements of velocity. Either a Kalman filter will have to be
    employed in order to estimate the vehicle's velocity, or the ESC feature of
    the vehicle will have to be investigated with regard to its ability to
    ensure that the input velocity is indeed the vehicle's velocity.



\section{Issues}

\begin{itemize}
  \item The ethernet adapter for the lidar is broken and needs to be replaced.
    This means that packages \texttt{circular\_mpc} and \texttt{centerline\_mpc}
    cannot be tested until communication with the lidar is fixed.
  \item The SML lab is booked for the week 14/11-18/11 (what about the weekend?),
    hence there is no access to MOCAP. This means that packages
    \texttt{circular\_pid} and \texttt{centerline\_pid} (gains need adjusting)
    cannot be tested until at least Saturday 19/11.
  \item Package \texttt{circular\_pid}, which was to be working out-of-the-box,
    does not work. The fault lies somewhere within ROS: it appears that when
    ROS\_MASTER runs outside Jetson, sometimes communication between Jetson and the
    nodes running outside it is not established. When it is established, no
    messages are getting through to teensy.
\end{itemize}


\section{To do}

\begin{itemize}
  \item Consult with Mohamed about the newly formed solution to the
    \texttt{centerline\_pid} problem. If correct, include it in the code.
  \item Tune gains of the PID concerning package \texttt{centerline\_pid}.
  \item Implement \texttt{centerline\_mpc}.
  \item Implement \texttt{circular\_mpc}.
  \item A node that handles the linearization of the kinematic model of the
    vehicle has to be written in ROS.
  \item Construct a Kalman filter to estimate the velocity of the vehicle.
    Even in the case where we had encoders, in the case of high slip, an
    encoder is irrelevant. For instance, think about the case where the
    surface that the vehicle's wheels touch is ice.

    However, the incorporation of a Kalman filter for estimating a state
    presupposes that this state \textit{is being measured}, however large its
    uncertainty might be. In our case the velocity is not being measured at all.
    Or is our input velocity considered to be a measurement of the vehicle's
    velocity?

    Can measuring displacements in $x,y$ from MOCAP and then deducing
    the vector of the vehicle's velocity be considered as velocity measurements?

    For instance, let us consider two consecutively sampled points in 2D, at
    time $t$ and at time $t+1$. Then, the magnitude of the vehicle's velocity at
    time $t$ \textit{was}

    \begin{align}
      V_{t} &= \dfrac{\sqrt{(x_{t+1} - x_t)^2 + (y_{t+1} - y_t)^2}}{T_s}
    \end{align}

    and the orientation of that velocity was

    \begin{align}
      \phase{V_t} = \beta_t + \psi_t = \text{tan}^{-1} \dfrac{y_{t+1} - y_t}{x_{t+1} - x_t}
    \end{align}

    \begin{figure}[H]\centering
      \scalebox{1}{% Generated with LaTeXDraw 2.0.8
% Wed Nov 16 01:07:48 CET 2016
% \usepackage[usenames,dvipsnames]{pstricks}
% \usepackage{epsfig}
% \usepackage{pst-grad} % For gradients
% \usepackage{pst-plot} % For axes
\scalebox{1} % Change this value to rescale the drawing.
{
\begin{pspicture}(0,-6.2364063)(15.369062,6.2164063)
\psline[linewidth=0.04cm,arrowsize=0.05291667cm 2.0,arrowlength=1.4,arrowinset=0.4]{<-}(0.83,6.1964064)(0.83,-5.603594)
\psline[linewidth=0.04cm,arrowsize=0.05291667cm 2.0,arrowlength=1.4,arrowinset=0.4]{->}(0.83,-5.603594)(15.19,-5.603594)
\psline[linewidth=0.04cm](3.41,-2.7035937)(5.99,-1.7035937)
\psline[linewidth=0.04cm](10.33,1.1364063)(11.67,3.6364062)
\psline[linewidth=0.012cm,arrowsize=0.05291667cm 2.0,arrowlength=1.4,arrowinset=0.4]{<-}(4.69,-0.20359375)(4.69,-2.2035937)
\psline[linewidth=0.012cm,arrowsize=0.05291667cm 2.0,arrowlength=1.4,arrowinset=0.4]{->}(4.69,-2.2035937)(6.69,-2.2035937)
\psline[linewidth=0.012cm,arrowsize=0.05291667cm 2.0,arrowlength=1.4,arrowinset=0.4]{<-}(10.99,4.376406)(10.99,2.3764062)
\psline[linewidth=0.012cm,arrowsize=0.05291667cm 2.0,arrowlength=1.4,arrowinset=0.4]{->}(10.99,2.3764062)(12.99,2.3764062)
\psdots[dotsize=0.12](4.7,-2.2035937)
\psdots[dotsize=0.12](11.0,2.3864062)
\psline[linewidth=0.02cm](4.7,-2.2035937)(11.0,2.3864062)
\usefont{T1}{ptm}{m}{n}
\rput(4.2345314,-0.51359373){$y$}
\usefont{T1}{ptm}{m}{n}
\rput(10.554531,4.0664062){$y$}
\usefont{T1}{ptm}{m}{n}
\rput(6.434531,-2.4735937){$x$}
\usefont{T1}{ptm}{m}{n}
\rput(12.754531,2.1464062){$x$}
\psarc[linewidth=0.012](5.3,-2.0935938){0.25}{331.69925}{49.76364}
\psarc[linewidth=0.012](5.26,-1.8135937){0.25}{331.69925}{36.253838}
\psarc[linewidth=0.012](11.2,2.5464063){0.25}{315.0}{85.601295}
\psarc[linewidth=0.012](11.02,2.6864061){0.25}{33.111343}{96.340195}
\usefont{T1}{ptm}{m}{n}
\rput(6.224531,-1.9535937){$\psi_t$}
\usefont{T1}{ptm}{m}{n}
\rput(6.2945313,-1.2735938){$\beta_t$}
\usefont{T1}{ptm}{m}{n}
\rput(11.884531,2.6664062){$\psi_{t+1}$}
\usefont{T1}{ptm}{m}{n}
\rput(11.534532,4.3264065){$\beta_{t+1}$}
\usefont{T1}{ptm}{m}{n}
\rput(4.684531,-2.8135939){$(x_t, y_t, \psi_t)$}
\usefont{T1}{ptm}{m}{n}
\rput(12.284532,1.7264062){$(x_{t+1}, y_{t+1}, \psi_{t+1})$}
\usefont{T1}{ptm}{m}{n}
\rput(0.27453125,5.726406){$y$}
\usefont{T1}{ptm}{m}{n}
\rput(14.554531,-6.0335937){$x$}
\end{pspicture} 
}

}
      \caption{}
      \label{}
    \end{figure}

    Provided that the sampling time of the vehicle's pose is sufficiently high,
    its velocity vector can be considered to be roughly equal between sampling
    times. In principle, MOCAP measures the pose of the vehicle 120 times a
    second. The maximum velocity that the specific car can achieve is $60$ km/h=
    $16.7$ m/s. In this case, the car travels $14$ cm in 1/120 seconds. The
    vehicle's heading is governed by

    \begin{align}
      \psi_{t+1} &= \psi_t + \dfrac{T_s V_t}{l_r} \text{sin}\beta
    \end{align}

    and its maximum heading angle difference between sampling times in our case
    is

    \begin{align}
      \Delta\psi_{max} &= \psi_{t+1} - \psi_t = \dfrac{T_s V_t}{l_r} \text{sin}\beta_{max}
    \end{align}

    but $\beta_{max}$ can be calculated from

    \begin{align}
      \beta &= +\text{tan}^{-1} (\dfrac{l_r}{l_r + l_f} \text{tan}\delta)
    \end{align}

    since the function is piecewise strictly increasing. Assuming that the
    maximum steering angle $\delta_{max} = 60^{\circ}$, and $\beta_{max} = 0.71$
    rad $= 27.2^{\circ}$, which makes $\Delta\psi_{max} = 21.7^{\circ}.$ In
    summation, the vehicle's orientation can change by as much
    as $22^\circ$ while its displacement can change by as much as $0.14$m
    between sampling instances.

    Is this rationale valid?

\end{itemize}

\section{Misc.}

The progress of the project can be observed in \texttt{trello} and \texttt{github}:

\begin{itemize}
  \item \url{https://trello.com/b/uEP0jl0B/slip-control}
  \item \url{https://gits-15.sys.kth.se/alefil/HT16_P2_EL2425}
  \item \url{https://gits-15.sys.kth.se/alefil/HT16_P2_EL2425_resources}
\end{itemize}


\end{document}
