In figure \ref{fig:circular_mpc}, the vehicle $C$, whose velocity is constant
and denoted by $v$, is to track a circle whose center is $O'$ and whose
radius is $O'R$. Its orientation relative to the global coordinate system is
$\psi$. $R(x_R, y_R)$ is the point $C$ is to track. The orientation of $R$
relative to the global coordinate system is $\psi_R$. The vehicle's coordinates
are $(x_c, y_c)$.


\begin{figure}[H]\centering
  \scalebox{0.8}{In figure \ref{fig:circular_mpc}, the vehicle $C$, whose velocity is constant
and denoted by $v$, is to track a circle whose center is $O'$ and whose
radius is $O'R$. Its orientation relative to the global coordinate system is
$\psi$. $R$ is the point $C$ is to track. The orientation of $R$ relative to
the global coordinate system is $\theta$. The vehicle's coordinates are
$(x_c, y_c)$. The aim is to find the vehicle's deviation from $R$ in terms of
translation along the $x$ and $y$ axes and rotation around the $x$-axis.

\begin{figure}[H]\centering
  \scalebox{0.8}{In figure \ref{fig:circular_mpc}, the vehicle $C$, whose velocity is constant
and denoted by $v$, is to track a circle whose center is $O'$ and whose
radius is $O'R$. Its orientation relative to the global coordinate system is
$\psi$. $R$ is the point $C$ is to track. The orientation of $R$ relative to
the global coordinate system is $\theta$. The vehicle's coordinates are
$(x_c, y_c)$. The aim is to find the vehicle's deviation from $R$ in terms of
translation along the $x$ and $y$ axes and rotation around the $x$-axis.

\begin{figure}[H]\centering
  \scalebox{0.8}{In figure \ref{fig:circular_mpc}, the vehicle $C$, whose velocity is constant
and denoted by $v$, is to track a circle whose center is $O'$ and whose
radius is $O'R$. Its orientation relative to the global coordinate system is
$\psi$. $R$ is the point $C$ is to track. The orientation of $R$ relative to
the global coordinate system is $\theta$. The vehicle's coordinates are
$(x_c, y_c)$. The aim is to find the vehicle's deviation from $R$ in terms of
translation along the $x$ and $y$ axes and rotation around the $x$-axis.

\begin{figure}[H]\centering
  \scalebox{0.8}{\input{./figures/circular_mpc.tex}}
  \caption{}
  \label{fig:circular_mpc}
\end{figure}

The solution to the problem can be decomposed into two separate components:
a translational and a rotational one.

\subsubsection{Translational component}

Here, we assume that the orientation of the tangent at $R$ is the same of that
of the vehicle and equal to $\psi$. Given the distance $RC$ (the coordinates of
both $R$ and $C$ are known) we are interested in finding the displacements
$RC_x$ and $RC_y$. From triangle $C_yRC$ we can immediately deduce that

\begin{align}
  RC_x &= RC sin\psi \\
  RC_y &= -RC cos\psi \\
\end{align}

where $RC = \sqrt{(x_C - x_R)^2 + (y_C - y_R)^2}$

\begin{figure}[H]\centering
  \scalebox{0.8}{\input{./figures/circular_mpc_translational.tex}}
  \caption{}
  \label{}
\end{figure}


\subsubsection{Rotational component}

Here, we assume that the only deviation of $C$ from $R$ is only in terms of
orientation. From figure \ref{fig:circular_mp_rotational} we can easily
discern that the angular error is $\theta -\psi$.

\begin{figure}[H]\centering
  \scalebox{0.8}{\input{./figures/circular_mpc_rotational.tex}}
  \caption{}
  \label{fig:circular_mp_rotational}
\end{figure}


\subsubsection{Obtaining the relevant linearized kinematic model}

The model constitutes the equations of motion of the vehicle, and has three
states ($x$, $y$ and $\psi$) and one input ($\delta$). The equations of the
vehicle's motion that are relevant here are

\begin{align}
  \dot{x} &= v cos(\psi + \beta) \\
  \dot{y} &= v sin(\psi + \beta) \\
  \dot{\psi} &= \dfrac{v}{l_r} sin\beta
\end{align}

Sampling with a sampling time of $T_s$ gives

\begin{align}
  x_{k+1} &= x_{k} + T_s v cos(\psi_k + \beta_k) \\
  y_{k+1} &= y_{k} + T_s v sin(\psi_k + \beta_k) \\
  \psi_{k+1} &= \psi_{k} + T_s \dfrac{v}{l_r} sin\beta_k
\end{align}

where

\begin{align}
  \beta_k = tan^{-1}\Big(\dfrac{l_r}{l_r + l_f} tan\delta_k\Big)
\end{align}


Forming the Jacobians for matrices $A$, $B$ and evaluating them at time
$t=k$ around $\delta = 0$ (which makes $\beta = 0$) gives

\begin{equation}
 A =
  \begin{bmatrix}
    1 & 0 & -T_s v sin(\psi_k + \beta_k) \\\\
    0 & 1 & T_s v cos(\psi_k + \beta_k) \\\\
    0 & 0 & 1
  \end{bmatrix}
  =
  \begin{bmatrix}
    1 & 0 & -T_s v sin\Big(\psi_k + tan^{-1} (l_q tan\delta_k)\Big) \\\\
    0 & 1 & T_s v cos\Big(\psi_k + tan^{-1} (l_q tan\delta_k)\Big) \\\\
    0 & 0 & 1
  \end{bmatrix}
\end{equation}
\begin{equation}
 A =
  \begin{bmatrix}
    1 & 0 & -T_s v sin(\psi_k) \\\\
    0 & 1 & T_s v cos(\psi_k) \\\\
    0 & 0 & 1
  \end{bmatrix}
\end{equation}

\begin{equation}
 B =
  \begin{bmatrix}
    -T_s v sin(\psi_k + \beta_k) \dfrac{l_q}{l_q^2 sin^2\delta_k + cos^2\delta_k} \\
    T_s v cos(\psi_k + \beta_k) \dfrac{l_q}{l_q^2 sin^2\delta_k + cos^2\delta_k} \\
    \dfrac{T_s v}{l_r} cos(\beta_k) \dfrac{l_q}{l_q^2 sin^2\delta_k + cos^2\delta_k}
  \end{bmatrix}
  =
  \begin{bmatrix}
    -T_s v sin\Big(\psi_k + tan^{-1} (l_q tan\delta_k)\Big) \dfrac{l_q}{l_q^2 sin^2\delta_k + cos^2\delta_k} \\
    T_s v cos\Big(\psi_k + tan^{-1} (l_q tan\delta_k)\Big) \dfrac{l_q}{l_q^2 sin^2\delta_k + cos^2\delta_k} \\
    \dfrac{T_s v}{l_r} cos\Bigg(tan^{-1} \Big(l_q tan\delta_k\Big)\Bigg) \dfrac{l_q}{l_q^2 sin^2\delta_k + cos^2\delta_k}
  \end{bmatrix}
\end{equation}
\begin{equation}
 B =
  \begin{bmatrix}
    -\dfrac{T_s l_r v}{l_r + l_f} sin\psi_k \\\\
    \dfrac{T_s l_r v}{l_r + l_f} cos\psi_k \\\\
    \dfrac{T_s v}{l_r+l_f}
  \end{bmatrix}
\end{equation}

where $l_q = \dfrac{l_r}{l_r + l_f}$




Now we can express the linear model as

\begin{align}
  s_{k+1} = A s_k + B \delta_k
\end{align}

where

\begin{equation}
  s=
  \begin{bmatrix}
    x_{k} \\
    y_{k} \\
    \psi_{k}
  \end{bmatrix}
\end{equation}

or

\begin{equation}
  \begin{bmatrix}
    x_{k+1} \\
    y_{k+1} \\
    \psi_{k+1}
  \end{bmatrix}
  =
  \begin{bmatrix}
    1 & 0 & -T_s v sin(\psi_k) \\\\
    0 & 1 & T_s v cos(\psi_k) \\\\
    0 & 0 & 1
  \end{bmatrix}
  \begin{bmatrix}
    x_{k} \\
    y_{k} \\
    \psi_{k}
  \end{bmatrix}
  +
  \begin{bmatrix}
    -\dfrac{T_s l_r v}{l_r + l_f} sin\psi_k \\\\
    \dfrac{T_s l_r v}{l_r + l_f} cos\psi_k \\\\
    \dfrac{T_s v}{l_r+l_f}
  \end{bmatrix}
  \delta_{k}
\end{equation}



\subsubsection{Stating the optimization problem}

We can now form the optimization problem as

\begin{align}
  \text{minimize }    & \sum\limits_{k=0}^N x_k^2 q_x + y_k^2 q_y + \phi_k^2 q_{\phi} + \delta_k^2 r \\
  \text{subject to }  & s_{k+1} = A s_k + B \delta_k,\text{ where } s_k = [x_k, y_k, \phi_k]^T \\
                      & \delta_{min} \leq \delta_k \leq \delta_{max} \\
                      & x_0 = RC sin\phi_0 \\
                      & y_0 = -RC cos\phi_0 \\
                      & \phi_0 = \theta_R - \psi
\end{align}
}
  \caption{}
  \label{fig:circular_mpc}
\end{figure}

The solution to the problem can be decomposed into two separate components:
a translational and a rotational one.

\subsubsection{Translational component}

Here, we assume that the orientation of the tangent at $R$ is the same of that
of the vehicle and equal to $\psi$. Given the distance $RC$ (the coordinates of
both $R$ and $C$ are known) we are interested in finding the displacements
$RC_x$ and $RC_y$. From triangle $C_yRC$ we can immediately deduce that

\begin{align}
  RC_x &= RC sin\psi \\
  RC_y &= -RC cos\psi \\
\end{align}

where $RC = \sqrt{(x_C - x_R)^2 + (y_C - y_R)^2}$

\begin{figure}[H]\centering
  \scalebox{0.8}{% Generated with LaTeXDraw 2.0.8
% Fri Nov 18 15:10:06 CET 2016
% \usepackage[usenames,dvipsnames]{pstricks}
% \usepackage{epsfig}
% \usepackage{pst-grad} % For gradients
% \usepackage{pst-plot} % For axes
\scalebox{1} % Change this value to rescale the drawing.
{
\begin{pspicture}(0,-7.3264065)(23.47,7.306406)
\pscircle[linewidth=0.04,dimen=outer](8.36,0.61640626){4.63}
\psline[linewidth=0.04cm,arrowsize=0.05291667cm 2.0,arrowlength=1.4,arrowinset=0.4]{<-}(0.97,7.286406)(0.97,-6.5135937)
\psline[linewidth=0.04cm,arrowsize=0.05291667cm 2.0,arrowlength=1.4,arrowinset=0.4]{->}(0.97,-6.5135937)(23.45,-6.5135937)
\psline[linewidth=0.04cm,arrowsize=0.05291667cm 2.0,arrowlength=1.4,arrowinset=0.4]{->}(15.11,-3.0335937)(18.47,2.1264062)
\psdots[dotsize=0.12](15.11,-3.0535936)
\usefont{T1}{ptm}{m}{n}
\rput(0.27453125,6.6764064){$y$}
\usefont{T1}{ptm}{m}{n}
\rput(22.734531,-7.123594){$x$}
\psdots[dotsize=0.12](8.36,0.61640626)
\psline[linewidth=0.02cm](8.35,0.6064063)(15.11,-3.0535936)
\usefont{T1}{ptm}{m}{n}
\rput(0.58453125,-7.043594){$O$}
\usefont{T1}{ptm}{m}{n}
\rput(14.994532,-3.8235939){$C$}
\usefont{T1}{ptm}{m}{n}
\rput(7.6045313,0.03640625){$O'$}
\psline[linewidth=0.02cm](9.57,-6.5135937)(15.37,3.4864063)
\psline[linewidth=0.02cm,arrowsize=0.05291667cm 2.0,arrowlength=1.4,arrowinset=0.4]{->}(12.39,-5.193594)(12.39,4.726406)
\psline[linewidth=0.02cm,arrowsize=0.05291667cm 2.0,arrowlength=1.4,arrowinset=0.4]{->}(10.19,-1.5735937)(19.53,-1.5935937)
\psline[linewidth=0.02cm,linestyle=dashed,dash=0.16cm 0.16cm](12.39,-3.0535936)(15.05,-3.0535936)
\psline[linewidth=0.02cm,linestyle=dashed,dash=0.16cm 0.16cm](15.11,-3.0535936)(15.11,-1.5735937)
\usefont{T1}{ptm}{m}{n}
\rput(11.694531,-1.8435937){$R$}
\usefont{T1}{ptm}{m}{n}
\rput(11.994532,-3.3235939){$C_y$}
\usefont{T1}{ptm}{m}{n}
\rput(15.094531,-1.2235937){$C_x$}
\psarc[linewidth=0.02](16.43,-1.4135938){0.24}{315.0}{113.62938}
\usefont{T1}{ptm}{m}{n}
\rput(17.244532,-1.2635938){$\psi$}
\usefont{T1}{ptm}{m}{n}
\rput(13.004531,-2.4635937){$\psi$}
\psarc[linewidth=0.02](12.6,-1.9635937){0.25}{218.6598}{42.27369}
\psarc[linewidth=0.02](12.77,-1.4135938){0.24}{315.0}{113.62938}
\usefont{T1}{ptm}{m}{n}
\rput(13.584531,-1.2635938){$\psi$}
\psline[linewidth=0.02cm](12.39,-1.0935937)(11.87,-1.0935937)
\psline[linewidth=0.02cm](11.85,-1.0935937)(11.85,-1.5735937)
\psdots[dotsize=0.12](12.11,-1.3335937)
\end{pspicture} 
}

}
  \caption{}
  \label{}
\end{figure}


\subsubsection{Rotational component}

Here, we assume that the only deviation of $C$ from $R$ is only in terms of
orientation. From figure \ref{fig:circular_mp_rotational} we can easily
discern that the angular error is $\theta -\psi$.

\begin{figure}[H]\centering
  \scalebox{0.8}{% Generated with LaTeXDraw 2.0.8
% Fri Nov 18 16:47:11 CET 2016
% \usepackage[usenames,dvipsnames]{pstricks}
% \usepackage{epsfig}
% \usepackage{pst-grad} % For gradients
% \usepackage{pst-plot} % For axes
\scalebox{1} % Change this value to rescale the drawing.
{
\begin{pspicture}(0,-7.3264065)(23.597345,7.306406)
\pscircle[linewidth=0.04,dimen=outer](8.487344,0.6164076){4.63}
\psline[linewidth=0.04cm,arrowsize=0.05291667cm 2.0,arrowlength=1.4,arrowinset=0.4]{<-}(1.0973437,7.286406)(1.0973437,-6.5135937)
\psline[linewidth=0.04cm,arrowsize=0.05291667cm 2.0,arrowlength=1.4,arrowinset=0.4]{->}(1.0973437,-6.5135937)(23.577345,-6.5135937)
\psline[linewidth=0.04cm,arrowsize=0.05291667cm 2.0,arrowlength=1.4,arrowinset=0.4]{->}(12.530937,-1.6135932)(15.837343,0.94640744)
\usefont{T1}{ptm}{m}{n}
\rput(0.27453125,6.6764073){$y$}
\usefont{T1}{ptm}{m}{n}
\rput(22.734531,-7.123594){$x$}
\psdots[dotsize=0.12](8.487344,0.6164076)
\psline[linewidth=0.02cm](8.4773445,0.6064075)(12.5364065,-1.613593)
\usefont{T1}{ptm}{m}{n}
\rput(0.58453125,-7.043594){$O$}
\usefont{T1}{ptm}{m}{n}
\rput(8.004531,0.0164075){$O'$}
\psline[linewidth=0.02cm,linestyle=dashed,dash=0.17638889cm 0.10583334cm](6.1373444,-6.5135937)(12.96,-1.2735938)
\psarc[linewidth=0.02](7.1018753,-6.213594){0.38}{310.9144}{99.09028}
\usefont{T1}{ptm}{m}{n}
\rput(8.034531,-5.943594){$\psi$}
\psline[linewidth=0.02cm](9.801874,-6.5135937)(13.541874,0.24640727)
\psline[linewidth=0.02cm](12.601875,-1.4535928)(16.341875,5.306406)
\psarc[linewidth=0.02](10.241876,-6.213594){0.38}{310.9144}{99.09028}
\usefont{T1}{ptm}{m}{n}
\rput(11.514531,-5.9235926){$\theta$}
\usefont{T1}{ptm}{m}{n}
\rput(10.764531,-1.6435927){$R \equiv C$}
\usefont{T1}{ptm}{m}{n}
\rput(14.304531,0.45640698){$\theta - \psi$}
\psarc[linewidth=0.02](13.546407,-0.443593){0.35}{340.46335}{130.76361}
\psline[linewidth=0.04cm](11.976405,-1.293593)(12.236406,-0.7735928)
\psline[linewidth=0.04cm](12.256407,-0.79359293)(12.756407,-1.073593)
\psdots[dotsize=0.12](12.376407,-1.213593)
\end{pspicture} 
}

}
  \caption{}
  \label{fig:circular_mp_rotational}
\end{figure}


\subsubsection{Obtaining the relevant linearized kinematic model}

The model constitutes the equations of motion of the vehicle, and has three
states ($x$, $y$ and $\psi$) and one input ($\delta$). The equations of the
vehicle's motion that are relevant here are

\begin{align}
  \dot{x} &= v cos(\psi + \beta) \\
  \dot{y} &= v sin(\psi + \beta) \\
  \dot{\psi} &= \dfrac{v}{l_r} sin\beta
\end{align}

Sampling with a sampling time of $T_s$ gives

\begin{align}
  x_{k+1} &= x_{k} + T_s v cos(\psi_k + \beta_k) \\
  y_{k+1} &= y_{k} + T_s v sin(\psi_k + \beta_k) \\
  \psi_{k+1} &= \psi_{k} + T_s \dfrac{v}{l_r} sin\beta_k
\end{align}

where

\begin{align}
  \beta_k = tan^{-1}\Big(\dfrac{l_r}{l_r + l_f} tan\delta_k\Big)
\end{align}


Forming the Jacobians for matrices $A$, $B$ and evaluating them at time
$t=k$ around $\delta = 0$ (which makes $\beta = 0$) gives

\begin{equation}
 A =
  \begin{bmatrix}
    1 & 0 & -T_s v sin(\psi_k + \beta_k) \\\\
    0 & 1 & T_s v cos(\psi_k + \beta_k) \\\\
    0 & 0 & 1
  \end{bmatrix}
  =
  \begin{bmatrix}
    1 & 0 & -T_s v sin\Big(\psi_k + tan^{-1} (l_q tan\delta_k)\Big) \\\\
    0 & 1 & T_s v cos\Big(\psi_k + tan^{-1} (l_q tan\delta_k)\Big) \\\\
    0 & 0 & 1
  \end{bmatrix}
\end{equation}
\begin{equation}
 A =
  \begin{bmatrix}
    1 & 0 & -T_s v sin(\psi_k) \\\\
    0 & 1 & T_s v cos(\psi_k) \\\\
    0 & 0 & 1
  \end{bmatrix}
\end{equation}

\begin{equation}
 B =
  \begin{bmatrix}
    -T_s v sin(\psi_k + \beta_k) \dfrac{l_q}{l_q^2 sin^2\delta_k + cos^2\delta_k} \\
    T_s v cos(\psi_k + \beta_k) \dfrac{l_q}{l_q^2 sin^2\delta_k + cos^2\delta_k} \\
    \dfrac{T_s v}{l_r} cos(\beta_k) \dfrac{l_q}{l_q^2 sin^2\delta_k + cos^2\delta_k}
  \end{bmatrix}
  =
  \begin{bmatrix}
    -T_s v sin\Big(\psi_k + tan^{-1} (l_q tan\delta_k)\Big) \dfrac{l_q}{l_q^2 sin^2\delta_k + cos^2\delta_k} \\
    T_s v cos\Big(\psi_k + tan^{-1} (l_q tan\delta_k)\Big) \dfrac{l_q}{l_q^2 sin^2\delta_k + cos^2\delta_k} \\
    \dfrac{T_s v}{l_r} cos\Bigg(tan^{-1} \Big(l_q tan\delta_k\Big)\Bigg) \dfrac{l_q}{l_q^2 sin^2\delta_k + cos^2\delta_k}
  \end{bmatrix}
\end{equation}
\begin{equation}
 B =
  \begin{bmatrix}
    -\dfrac{T_s l_r v}{l_r + l_f} sin\psi_k \\\\
    \dfrac{T_s l_r v}{l_r + l_f} cos\psi_k \\\\
    \dfrac{T_s v}{l_r+l_f}
  \end{bmatrix}
\end{equation}

where $l_q = \dfrac{l_r}{l_r + l_f}$




Now we can express the linear model as

\begin{align}
  s_{k+1} = A s_k + B \delta_k
\end{align}

where

\begin{equation}
  s=
  \begin{bmatrix}
    x_{k} \\
    y_{k} \\
    \psi_{k}
  \end{bmatrix}
\end{equation}

or

\begin{equation}
  \begin{bmatrix}
    x_{k+1} \\
    y_{k+1} \\
    \psi_{k+1}
  \end{bmatrix}
  =
  \begin{bmatrix}
    1 & 0 & -T_s v sin(\psi_k) \\\\
    0 & 1 & T_s v cos(\psi_k) \\\\
    0 & 0 & 1
  \end{bmatrix}
  \begin{bmatrix}
    x_{k} \\
    y_{k} \\
    \psi_{k}
  \end{bmatrix}
  +
  \begin{bmatrix}
    -\dfrac{T_s l_r v}{l_r + l_f} sin\psi_k \\\\
    \dfrac{T_s l_r v}{l_r + l_f} cos\psi_k \\\\
    \dfrac{T_s v}{l_r+l_f}
  \end{bmatrix}
  \delta_{k}
\end{equation}



\subsubsection{Stating the optimization problem}

We can now form the optimization problem as

\begin{align}
  \text{minimize }    & \sum\limits_{k=0}^N x_k^2 q_x + y_k^2 q_y + \phi_k^2 q_{\phi} + \delta_k^2 r \\
  \text{subject to }  & s_{k+1} = A s_k + B \delta_k,\text{ where } s_k = [x_k, y_k, \phi_k]^T \\
                      & \delta_{min} \leq \delta_k \leq \delta_{max} \\
                      & x_0 = RC sin\phi_0 \\
                      & y_0 = -RC cos\phi_0 \\
                      & \phi_0 = \theta_R - \psi
\end{align}
}
  \caption{}
  \label{fig:circular_mpc}
\end{figure}

The solution to the problem can be decomposed into two separate components:
a translational and a rotational one.

\subsubsection{Translational component}

Here, we assume that the orientation of the tangent at $R$ is the same of that
of the vehicle and equal to $\psi$. Given the distance $RC$ (the coordinates of
both $R$ and $C$ are known) we are interested in finding the displacements
$RC_x$ and $RC_y$. From triangle $C_yRC$ we can immediately deduce that

\begin{align}
  RC_x &= RC sin\psi \\
  RC_y &= -RC cos\psi \\
\end{align}

where $RC = \sqrt{(x_C - x_R)^2 + (y_C - y_R)^2}$

\begin{figure}[H]\centering
  \scalebox{0.8}{% Generated with LaTeXDraw 2.0.8
% Fri Nov 18 15:10:06 CET 2016
% \usepackage[usenames,dvipsnames]{pstricks}
% \usepackage{epsfig}
% \usepackage{pst-grad} % For gradients
% \usepackage{pst-plot} % For axes
\scalebox{1} % Change this value to rescale the drawing.
{
\begin{pspicture}(0,-7.3264065)(23.47,7.306406)
\pscircle[linewidth=0.04,dimen=outer](8.36,0.61640626){4.63}
\psline[linewidth=0.04cm,arrowsize=0.05291667cm 2.0,arrowlength=1.4,arrowinset=0.4]{<-}(0.97,7.286406)(0.97,-6.5135937)
\psline[linewidth=0.04cm,arrowsize=0.05291667cm 2.0,arrowlength=1.4,arrowinset=0.4]{->}(0.97,-6.5135937)(23.45,-6.5135937)
\psline[linewidth=0.04cm,arrowsize=0.05291667cm 2.0,arrowlength=1.4,arrowinset=0.4]{->}(15.11,-3.0335937)(18.47,2.1264062)
\psdots[dotsize=0.12](15.11,-3.0535936)
\usefont{T1}{ptm}{m}{n}
\rput(0.27453125,6.6764064){$y$}
\usefont{T1}{ptm}{m}{n}
\rput(22.734531,-7.123594){$x$}
\psdots[dotsize=0.12](8.36,0.61640626)
\psline[linewidth=0.02cm](8.35,0.6064063)(15.11,-3.0535936)
\usefont{T1}{ptm}{m}{n}
\rput(0.58453125,-7.043594){$O$}
\usefont{T1}{ptm}{m}{n}
\rput(14.994532,-3.8235939){$C$}
\usefont{T1}{ptm}{m}{n}
\rput(7.6045313,0.03640625){$O'$}
\psline[linewidth=0.02cm](9.57,-6.5135937)(15.37,3.4864063)
\psline[linewidth=0.02cm,arrowsize=0.05291667cm 2.0,arrowlength=1.4,arrowinset=0.4]{->}(12.39,-5.193594)(12.39,4.726406)
\psline[linewidth=0.02cm,arrowsize=0.05291667cm 2.0,arrowlength=1.4,arrowinset=0.4]{->}(10.19,-1.5735937)(19.53,-1.5935937)
\psline[linewidth=0.02cm,linestyle=dashed,dash=0.16cm 0.16cm](12.39,-3.0535936)(15.05,-3.0535936)
\psline[linewidth=0.02cm,linestyle=dashed,dash=0.16cm 0.16cm](15.11,-3.0535936)(15.11,-1.5735937)
\usefont{T1}{ptm}{m}{n}
\rput(11.694531,-1.8435937){$R$}
\usefont{T1}{ptm}{m}{n}
\rput(11.994532,-3.3235939){$C_y$}
\usefont{T1}{ptm}{m}{n}
\rput(15.094531,-1.2235937){$C_x$}
\psarc[linewidth=0.02](16.43,-1.4135938){0.24}{315.0}{113.62938}
\usefont{T1}{ptm}{m}{n}
\rput(17.244532,-1.2635938){$\psi$}
\usefont{T1}{ptm}{m}{n}
\rput(13.004531,-2.4635937){$\psi$}
\psarc[linewidth=0.02](12.6,-1.9635937){0.25}{218.6598}{42.27369}
\psarc[linewidth=0.02](12.77,-1.4135938){0.24}{315.0}{113.62938}
\usefont{T1}{ptm}{m}{n}
\rput(13.584531,-1.2635938){$\psi$}
\psline[linewidth=0.02cm](12.39,-1.0935937)(11.87,-1.0935937)
\psline[linewidth=0.02cm](11.85,-1.0935937)(11.85,-1.5735937)
\psdots[dotsize=0.12](12.11,-1.3335937)
\end{pspicture} 
}

}
  \caption{}
  \label{}
\end{figure}


\subsubsection{Rotational component}

Here, we assume that the only deviation of $C$ from $R$ is only in terms of
orientation. From figure \ref{fig:circular_mp_rotational} we can easily
discern that the angular error is $\theta -\psi$.

\begin{figure}[H]\centering
  \scalebox{0.8}{% Generated with LaTeXDraw 2.0.8
% Fri Nov 18 16:47:11 CET 2016
% \usepackage[usenames,dvipsnames]{pstricks}
% \usepackage{epsfig}
% \usepackage{pst-grad} % For gradients
% \usepackage{pst-plot} % For axes
\scalebox{1} % Change this value to rescale the drawing.
{
\begin{pspicture}(0,-7.3264065)(23.597345,7.306406)
\pscircle[linewidth=0.04,dimen=outer](8.487344,0.6164076){4.63}
\psline[linewidth=0.04cm,arrowsize=0.05291667cm 2.0,arrowlength=1.4,arrowinset=0.4]{<-}(1.0973437,7.286406)(1.0973437,-6.5135937)
\psline[linewidth=0.04cm,arrowsize=0.05291667cm 2.0,arrowlength=1.4,arrowinset=0.4]{->}(1.0973437,-6.5135937)(23.577345,-6.5135937)
\psline[linewidth=0.04cm,arrowsize=0.05291667cm 2.0,arrowlength=1.4,arrowinset=0.4]{->}(12.530937,-1.6135932)(15.837343,0.94640744)
\usefont{T1}{ptm}{m}{n}
\rput(0.27453125,6.6764073){$y$}
\usefont{T1}{ptm}{m}{n}
\rput(22.734531,-7.123594){$x$}
\psdots[dotsize=0.12](8.487344,0.6164076)
\psline[linewidth=0.02cm](8.4773445,0.6064075)(12.5364065,-1.613593)
\usefont{T1}{ptm}{m}{n}
\rput(0.58453125,-7.043594){$O$}
\usefont{T1}{ptm}{m}{n}
\rput(8.004531,0.0164075){$O'$}
\psline[linewidth=0.02cm,linestyle=dashed,dash=0.17638889cm 0.10583334cm](6.1373444,-6.5135937)(12.96,-1.2735938)
\psarc[linewidth=0.02](7.1018753,-6.213594){0.38}{310.9144}{99.09028}
\usefont{T1}{ptm}{m}{n}
\rput(8.034531,-5.943594){$\psi$}
\psline[linewidth=0.02cm](9.801874,-6.5135937)(13.541874,0.24640727)
\psline[linewidth=0.02cm](12.601875,-1.4535928)(16.341875,5.306406)
\psarc[linewidth=0.02](10.241876,-6.213594){0.38}{310.9144}{99.09028}
\usefont{T1}{ptm}{m}{n}
\rput(11.514531,-5.9235926){$\theta$}
\usefont{T1}{ptm}{m}{n}
\rput(10.764531,-1.6435927){$R \equiv C$}
\usefont{T1}{ptm}{m}{n}
\rput(14.304531,0.45640698){$\theta - \psi$}
\psarc[linewidth=0.02](13.546407,-0.443593){0.35}{340.46335}{130.76361}
\psline[linewidth=0.04cm](11.976405,-1.293593)(12.236406,-0.7735928)
\psline[linewidth=0.04cm](12.256407,-0.79359293)(12.756407,-1.073593)
\psdots[dotsize=0.12](12.376407,-1.213593)
\end{pspicture} 
}

}
  \caption{}
  \label{fig:circular_mp_rotational}
\end{figure}


\subsubsection{Obtaining the relevant linearized kinematic model}

The model constitutes the equations of motion of the vehicle, and has three
states ($x$, $y$ and $\psi$) and one input ($\delta$). The equations of the
vehicle's motion that are relevant here are

\begin{align}
  \dot{x} &= v cos(\psi + \beta) \\
  \dot{y} &= v sin(\psi + \beta) \\
  \dot{\psi} &= \dfrac{v}{l_r} sin\beta
\end{align}

Sampling with a sampling time of $T_s$ gives

\begin{align}
  x_{k+1} &= x_{k} + T_s v cos(\psi_k + \beta_k) \\
  y_{k+1} &= y_{k} + T_s v sin(\psi_k + \beta_k) \\
  \psi_{k+1} &= \psi_{k} + T_s \dfrac{v}{l_r} sin\beta_k
\end{align}

where

\begin{align}
  \beta_k = tan^{-1}\Big(\dfrac{l_r}{l_r + l_f} tan\delta_k\Big)
\end{align}


Forming the Jacobians for matrices $A$, $B$ and evaluating them at time
$t=k$ around $\delta = 0$ (which makes $\beta = 0$) gives

\begin{equation}
 A =
  \begin{bmatrix}
    1 & 0 & -T_s v sin(\psi_k + \beta_k) \\\\
    0 & 1 & T_s v cos(\psi_k + \beta_k) \\\\
    0 & 0 & 1
  \end{bmatrix}
  =
  \begin{bmatrix}
    1 & 0 & -T_s v sin\Big(\psi_k + tan^{-1} (l_q tan\delta_k)\Big) \\\\
    0 & 1 & T_s v cos\Big(\psi_k + tan^{-1} (l_q tan\delta_k)\Big) \\\\
    0 & 0 & 1
  \end{bmatrix}
\end{equation}
\begin{equation}
 A =
  \begin{bmatrix}
    1 & 0 & -T_s v sin(\psi_k) \\\\
    0 & 1 & T_s v cos(\psi_k) \\\\
    0 & 0 & 1
  \end{bmatrix}
\end{equation}

\begin{equation}
 B =
  \begin{bmatrix}
    -T_s v sin(\psi_k + \beta_k) \dfrac{l_q}{l_q^2 sin^2\delta_k + cos^2\delta_k} \\
    T_s v cos(\psi_k + \beta_k) \dfrac{l_q}{l_q^2 sin^2\delta_k + cos^2\delta_k} \\
    \dfrac{T_s v}{l_r} cos(\beta_k) \dfrac{l_q}{l_q^2 sin^2\delta_k + cos^2\delta_k}
  \end{bmatrix}
  =
  \begin{bmatrix}
    -T_s v sin\Big(\psi_k + tan^{-1} (l_q tan\delta_k)\Big) \dfrac{l_q}{l_q^2 sin^2\delta_k + cos^2\delta_k} \\
    T_s v cos\Big(\psi_k + tan^{-1} (l_q tan\delta_k)\Big) \dfrac{l_q}{l_q^2 sin^2\delta_k + cos^2\delta_k} \\
    \dfrac{T_s v}{l_r} cos\Bigg(tan^{-1} \Big(l_q tan\delta_k\Big)\Bigg) \dfrac{l_q}{l_q^2 sin^2\delta_k + cos^2\delta_k}
  \end{bmatrix}
\end{equation}
\begin{equation}
 B =
  \begin{bmatrix}
    -\dfrac{T_s l_r v}{l_r + l_f} sin\psi_k \\\\
    \dfrac{T_s l_r v}{l_r + l_f} cos\psi_k \\\\
    \dfrac{T_s v}{l_r+l_f}
  \end{bmatrix}
\end{equation}

where $l_q = \dfrac{l_r}{l_r + l_f}$




Now we can express the linear model as

\begin{align}
  s_{k+1} = A s_k + B \delta_k
\end{align}

where

\begin{equation}
  s=
  \begin{bmatrix}
    x_{k} \\
    y_{k} \\
    \psi_{k}
  \end{bmatrix}
\end{equation}

or

\begin{equation}
  \begin{bmatrix}
    x_{k+1} \\
    y_{k+1} \\
    \psi_{k+1}
  \end{bmatrix}
  =
  \begin{bmatrix}
    1 & 0 & -T_s v sin(\psi_k) \\\\
    0 & 1 & T_s v cos(\psi_k) \\\\
    0 & 0 & 1
  \end{bmatrix}
  \begin{bmatrix}
    x_{k} \\
    y_{k} \\
    \psi_{k}
  \end{bmatrix}
  +
  \begin{bmatrix}
    -\dfrac{T_s l_r v}{l_r + l_f} sin\psi_k \\\\
    \dfrac{T_s l_r v}{l_r + l_f} cos\psi_k \\\\
    \dfrac{T_s v}{l_r+l_f}
  \end{bmatrix}
  \delta_{k}
\end{equation}



\subsubsection{Stating the optimization problem}

We can now form the optimization problem as

\begin{align}
  \text{minimize }    & \sum\limits_{k=0}^N x_k^2 q_x + y_k^2 q_y + \phi_k^2 q_{\phi} + \delta_k^2 r \\
  \text{subject to }  & s_{k+1} = A s_k + B \delta_k,\text{ where } s_k = [x_k, y_k, \phi_k]^T \\
                      & \delta_{min} \leq \delta_k \leq \delta_{max} \\
                      & x_0 = RC sin\phi_0 \\
                      & y_0 = -RC cos\phi_0 \\
                      & \phi_0 = \theta_R - \psi
\end{align}
}
  \caption{}
  \label{fig:circular_mpc}
\end{figure}

$R$ is a moving reference point and its pose is found as follows. Given $C$ and
the trajectory of the circle, point $R$ is the point with the least distance
to $C$ among all points of the trajectory. Since the circle is known a priori
and comprised by a set of poses, $R$ is known.

The aim is for the vehicle $C$ to minimize its deviation from $R$, in other
words we want to drive the differences

\begin{equation}
  \begin{bmatrix}
    x_c - x_R \\
    y_c - y_R \\
    v_c - v_R \\
    \psi_c - \psi_R
  \end{bmatrix}
  \rightarrow 0
\end{equation}

where $v_c$ and $v_R$ are the velocities of the vehicle and the reference
point, respectively.

\textbf{Obtaining the linearized kinematic model}

The model constitutes the equations of motion of the vehicle, and has four
states ($x$, $y$, $v$ and $\psi$) and two inputs ($v_i$ and $\delta$). The
equations of the vehicle's motion that are relevant here are

\begin{align}
  \dot{x} &= v cos(\psi + \beta) \\
  \dot{y} &= v sin(\psi + \beta) \\
  \dot{v} &= \dfrac{v_i - v}{\tau} \\
  \dot{\psi} &= \dfrac{v}{l_r} sin\beta
\end{align}

Sampling with a sampling time of $T_s$ gives

\begin{align}
  x_{k+1} &= x_{k} + T_s v_k cos(\psi_k + \beta_k) \\
  y_{k+1} &= y_{k} + T_s v_k sin(\psi_k + \beta_k) \\
  v_{k+1} &= v_{k} + \dfrac{T_s}{\tau} (v_{i,k} - v_{k}) \\
  \psi_{k+1} &= \psi_{k} + T_s \dfrac{v}{l_r} sin\beta_k
\end{align}

where

\begin{align}
  \beta_k = tan^{-1}\Big(\dfrac{l_r}{l_r + l_f} tan\delta_{k-1}\Big)
\end{align}


Forming the Jacobians for matrices $A$, $B$ and evaluating them at time
$t=k$ around the current state $\psi = \psi_k$, $v = v_k$ and
$\delta = \delta_{k-1}$ ($\delta_k$ is to be determined at time $k$):

\begin{equation}
 A =
  \begin{bmatrix}
    1 & 0 & T_s cos(\psi_k + \beta_k) & -T_s v_k sin(\psi_k + \beta_k) \\\\
    0 & 1 & T_s sin(\psi_k + \beta_k) & T_s v_k cos(\psi_k + \beta_k) \\\\
    0 & 0 & 1-\dfrac{T_s}{\tau} & 0 \\\\
    0 & 0 & \dfrac{T_s}{l_r}sin(\beta_k) & 1
  \end{bmatrix}
\end{equation}
\begin{equation}
  A =
  \begin{bmatrix}
    1 & 0 & T_s cos\Big(\psi_k + tan^{-1} (l_q tan\delta_{k-1})\Big) & -T_s v_k sin\Big(\psi_k + tan^{-1} (l_q tan\delta_{k-1})\Big) \\\\
    0 & 1 & T_s sin\Big(\psi_k + tan^{-1} (l_q tan\delta_{k-1})\Big) & T_s v_k cos\Big(\psi_k + tan^{-1} (l_q tan\delta_{k-1})\Big) \\\\
    0 & 0 & 1-\dfrac{T_s}{\tau} & 0 \\\\
    0 & 0 & \dfrac{T_s}{l_r}sin(tan^{-1} (l_q tan\delta_{k-1})) & 1
  \end{bmatrix}
\end{equation}


\begin{equation}
 B =
  \begin{bmatrix}
    0 & -T_s v_k sin(\psi_k + \beta_k) \dfrac{l_q}{l_q^2 sin^2\delta_{k-1} + cos^2\delta_{k-1}} \\
    0 & T_s v_k cos(\psi_k + \beta_k) \dfrac{l_q}{l_q^2 sin^2\delta_{k-1} + cos^2\delta_{k-1}} \\
    \dfrac{T_s}{\tau} v_k & 0 \\
    0 & \dfrac{T_s v_k}{l_r} cos(\beta_k) \dfrac{l_q}{l_q^2 sin^2\delta_{k-1} + cos^2\delta_{k-1}}
  \end{bmatrix}
\end{equation}
\begin{equation}
  B =
  \begin{bmatrix}
    0 & -T_s v_k sin\Big(\psi_k + tan^{-1} (l_q tan\delta_{k-1})\Big) \dfrac{l_q}{l_q^2 sin^2\delta_{k-1} + cos^2\delta_{k-1}} \\
    0 & T_s v_k cos\Big(\psi_k + tan^{-1} (l_q tan\delta_{k-1})\Big) \dfrac{l_q}{l_q^2 sin^2\delta_{k-1} + cos^2\delta_{k-1}} \\
    \dfrac{T_s}{\tau} v_k & 0 \\
    0 & \dfrac{T_s v_k}{l_r} cos\Bigg(tan^{-1} \Big(l_q tan\delta_{k-1}\Big)\Bigg) \dfrac{l_q}{l_q^2 sin^2\delta_{k-1} + cos^2\delta_{k-1}}
  \end{bmatrix}
\end{equation}


where $l_q = \dfrac{l_r}{l_r + l_f}$


Now we can express the linear model as

\begin{align}
  s_{k+1} = A s_k + B u_k
\end{align}

where

\begin{equation}
  s_k=
  \begin{bmatrix}
    x_{k} \\
    y_{k} \\
    v_{k} \\
    \psi_{k}
  \end{bmatrix}
\end{equation}

and

\begin{equation}
  u_k=
  \begin{bmatrix}
    v_{i,k} \\
    \delta_{k}
  \end{bmatrix}
\end{equation}



\textbf{Stating the optimization problem}

We can now form the optimization problem to be solved at time $t$ as

\begin{align}
  \text{minimize }    & \sum\limits_{k=0}^N (s_k - s_{ref})^T Q (s_k - s_{ref}) + u_k^T R u_k \\
  \text{subject to }  & s_{k+1} = A s_k + B u_k, \text{ where } s_k = [x_k, y_k, v_k, \psi_k]^T, u_k = [v_{i,k}, \delta_k]^T \\
                      & u_{i}^{min} \leq u_{i,k} \leq u_{i}^{max} \\
                      & \delta^{min} \leq \delta_k \leq \delta^{max} \\
                      & s_{ref} = (x_R, y_R, v_R, \psi_R) \\
                      & s_0 = (x_t, y_t, v_t, \psi_t)
\end{align}
