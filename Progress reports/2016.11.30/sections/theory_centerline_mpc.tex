In figure \ref{fig:centerline_mpc}, the $x$ axis is fixed on the lane's
centerline, while axis $y$ is perpendicular to it, facing the left boundary of
the lane. The origin is at point
$O$. The vehicle is represented by point $C$. The orientation of the vehicle
with respect to the lane (the $x$-axis) is $\phi$. Given this configuration
and two ranges from the range scan, at $-90^{\circ}$ and $90^{\circ}$ with
respect to the longitudinal axis of the vehicle (denoted by $CR$ and
$CL$ respectively), the objective is to find the distance $OC$ and the angle
$\phi$ so that a MPC optimization problem can be solved with $OC$ and $\phi$
acting as initial conditions. The only source of information is the lidar itself.

\begin{figure}[H]\centering
  \scalebox{1}{% Generated with LaTeXDraw 2.0.8
% Thu Nov 17 00:32:19 CET 2016
% \usepackage[usenames,dvipsnames]{pstricks}
% \usepackage{epsfig}
% \usepackage{pst-grad} % For gradients
% \usepackage{pst-plot} % For axes
\scalebox{1} % Change this value to rescale the drawing.
{
\begin{pspicture}(0,-3.5725)(13.853594,3.5525)
\psline[linewidth=0.04cm](0.0,2.1925)(13.34,2.1925)
\psline[linewidth=0.04cm](0.0,-3.0075)(13.34,-3.0075)
\psline[linewidth=0.03cm,linestyle=dashed,dash=0.16cm 0.16cm](0.0,-0.4075)(13.34,-0.4075)
\psline[linewidth=0.04cm,arrowsize=0.05291667cm 2.0,arrowlength=1.4,arrowinset=0.4]{->}(4.94,-0.4075)(4.94,3.5325)
\psline[linewidth=0.04cm,arrowsize=0.05291667cm 2.0,arrowlength=1.4,arrowinset=0.4]{->}(4.94,-0.4075)(13.76,-0.4075)
\usefont{T1}{ptm}{m}{n}
\rput(13.479062,-0.8775){$x$}
\usefont{T1}{ptm}{m}{n}
\rput(4.3790627,3.1625){$y$}
\psline[linewidth=0.04cm,linecolor=red](4.94,-1.8275)(10.54,2.1925)
\psline[linewidth=0.04cm,linecolor=red](5.72,-3.0075)(2.36,2.1925)
\usefont{T1}{ptm}{m}{n}
\rput(2.2690625,2.5225){$L$}
\usefont{T1}{ptm}{m}{n}
\rput(6.0590625,-3.3175){$R$}
\usefont{T1}{ptm}{m}{n}
\rput(10.7890625,2.6625){$F$}
\psarc[linewidth=0.02](7.41,-0.2575){0.17}{304.69516}{100.00798}
\psarc[linewidth=0.02](10.13,2.0625){0.17}{127.092834}{263.65982}
\psline[linewidth=0.02cm](4.94,-0.4275)(4.94,-3.0075)
\usefont{T1}{ptm}{m}{n}
\rput(5.6190624,-1.8775){$C$}
\usefont{T1}{ptm}{m}{n}
\rput(5.4890623,-0.0375){$O$}
\usefont{T1}{ptm}{m}{n}
\rput(5.4490623,2.4625){$L_o$}
\usefont{T1}{ptm}{m}{n}
\rput(4.6390624,-3.3375){$R_o$}
\usefont{T1}{ptm}{m}{n}
\rput(9.399062,1.8825){$\phi$}
\usefont{T1}{ptm}{m}{n}
\rput(8.019062,-0.1375){$\phi$}
\psline[linewidth=0.04cm](5.16,-1.6875)(5.3,-1.9075)
\psline[linewidth=0.04cm](5.28,-1.8875)(5.1,-2.0275)
\psdots[dotsize=0.06](5.1,-1.8475)
\end{pspicture} 
}

}
  \caption{}
  \label{fig:centerline_mpc}
\end{figure}


\textbf{Initial conditions} \\

\textbf{Finding $\phi$}

Here we can distinguish two cases, one where the vehicle is facing the right
lane boundary and one where it is facing the left one. It is not obvious
in this configuration where the vehicle is heading: the only available
information so far is only the two ranges $CR$ and $CL$.

In the first case, when the vehicle is on the right semilane, $CL-CR > 0$.
We retrieve the minimum distance from the range scan available at time $t$,
and denote the point which corresponds to this distance with $M$. The angle
between point $M$ and the longitudinal axis of the vehicle is denoted with
$\alpha$. In order to find this angle, we can exploit the fact that each ray
of the scan is separated from the next by $res$ degrees, while all of them
are stored in an array sequentially, in an anti-clockwise manner. In the case
of the \texttt{HOKUYO-UST-10LX-LASER}, the angular resolution is
$res=0.45^{\circ}$, and the starting angle is $-135^{\circ}$ with respect to the
longitudinal axis of the vehicle. This means that we can retrieve the angle
$\alpha$ by first calculating the number of indices between the one that
corresponds to the ray with the minimum range and the one that corresponds
to $+135^{\circ}$ (which in our case is $135 / 0.25 = 540$) with regard to the
laser's system of reference, and then by multiplying this number $\Delta i$ by
the angular resolution $res$. Hence, $\alpha = 0.25 \times \Delta i$.

At this point we do not know whether the vehicle is pointing to the
left or to the right lane boundary. However, we can determine the sign and the
magnitude of the orientation of the vehicle with respect to the orientation of
the lane by examining the sign of the difference $\alpha - 90^{\circ}$:
if it is negative, the vehicle is pointing towards the right boundary lane,
if it is positive, towards the left. Furthermore, we can now
calculate the magnitude of the orientation of the vehicle as the difference
$|\alpha - 90^{\circ}|$, as shown in figure \ref{fig:circular_mpc_rotation_car_at_right_side}.
In conclusion, if the car is located at the right semilane its orientation with
respect to the orientation of the lane is

$$\phi = sign(CL-CR)(\alpha - 90)$$


\begin{figure}[H]\centering
  \scalebox{1}{% Generated with LaTeXDraw 2.0.8
% Tue Nov 22 22:05:15 CET 2016
% \usepackage[usenames,dvipsnames]{pstricks}
% \usepackage{epsfig}
% \usepackage{pst-grad} % For gradients
% \usepackage{pst-plot} % For axes
\scalebox{1} % Change this value to rescale the drawing.
{
\begin{pspicture}(0,-4.5399995)(17.355,4.5499997)
\definecolor{color8}{rgb}{0.00392156862745098,0.00392156862745098,0.00392156862745098}
\psline[linewidth=0.04cm](0.0,3.19)(17.06,3.1499999)
\psline[linewidth=0.04cm](0.0,-2.01)(16.9,-2.01)
\psline[linewidth=0.03cm,linestyle=dashed,dash=0.16cm 0.16cm](0.0,0.59000003)(17.34,0.59000003)
\psline[linewidth=0.04cm,arrowsize=0.05291667cm 2.0,arrowlength=1.4,arrowinset=0.4]{->}(2.16,0.59000003)(2.16,4.5299997)
\usefont{T1}{ptm}{m}{n}
\rput(16.426249,0.08000001){$x$}
\usefont{T1}{ptm}{m}{n}
\rput(1.3462504,4.16){$y$}
\psline[linewidth=0.04cm,linecolor=color8,arrowsize=0.05291667cm 2.0,arrowlength=1.4,arrowinset=0.4]{->}(2.132057,-0.86982274)(5.92,-2.01)
\psline[linewidth=0.04cm,linecolor=color8](1.74,-2.03)(3.58,3.17)
\usefont{T1}{ptm}{m}{n}
\rput(3.5562503,3.44){$L^{-}$}
\usefont{T1}{ptm}{m}{n}
\rput(1.2262503,-2.3600001){$R^{-}$}
\psarc[linewidth=0.02](4.93,-1.8399998){0.17}{127.092834}{263.65982}
\psline[linewidth=0.02cm](2.16,0.57000005)(2.16,-2.01)
\usefont{T1}{ptm}{m}{n}
\rput(1.6462499,-0.89999986){$C^{-}$}
\usefont{T1}{ptm}{m}{n}
\rput(1.7562498,0.9200001){$O^{-}$}
\usefont{T1}{ptm}{m}{n}
\rput(4.2262487,-1.8199999){$\phi^{-}$}
\psline[linewidth=0.04cm](2.2230666,-0.63160646)(2.4694288,-0.71707916)
\psline[linewidth=0.04cm](2.461332,-0.71382576)(2.38,-0.93000007)
\psdots[dotsize=0.06](2.3,-0.79)
\usefont{T1}{ptm}{m}{n}
\rput(2.4362495,-2.34){$M^{-}$}
\psline[linewidth=0.04cm,linecolor=color8,arrowsize=0.05291667cm 2.0,arrowlength=1.4,arrowinset=0.4]{->}(11.33792,-1.0575329)(15.04,2.07)
\psline[linewidth=0.04cm,linecolor=color8](12.099635,-2.0164437)(8.1,3.21)
\psline[linewidth=0.04cm,arrowsize=0.05291667cm 2.0,arrowlength=1.4,arrowinset=0.4]{->}(11.32,0.5500001)(11.38,4.47)
\psline[linewidth=0.02cm](11.34,1.5500001)(11.34,-1.03)
\usefont{T1}{ptm}{m}{n}
\rput(10.936251,-2.3400002){$M^{+}$}
\usefont{T1}{ptm}{m}{n}
\rput(8.67625,3.48){$L^{+}$}
\usefont{T1}{ptm}{m}{n}
\rput(12.686251,-2.3400002){$R^{+}$}
\usefont{T1}{ptm}{m}{n}
\rput(10.60625,-1.06){$C$}
\psarc[linewidth=0.02](2.37,-1.22){0.33}{232.59465}{31.373005}
\psarc[linewidth=0.02](11.56,-1.05){0.44}{242.02052}{60.25512}
\usefont{T1}{ptm}{m}{n}
\rput(2.9535935,-1.58){$\alpha^{-}$}
\psarc[linewidth=0.02](13.96,0.78999996){0.36}{326.30994}{90.0}
\usefont{T1}{ptm}{m}{n}
\rput(14.72625,0.96000004){$\phi^{+}$}
\usefont{T1}{ptm}{m}{n}
\rput(12.206249,-3.46){$\phi^{+}$}
\usefont{T1}{ptm}{m}{n}
\rput(12.433595,-1.1400001){$\alpha^{+}$}
\psline[linewidth=0.02cm,linecolor=color8](0.9,-4.41)(1.78,-1.9299998)
\psline[linewidth=0.02cm](2.16,-1.9499999)(2.16,-4.5299997)
\usefont{T1}{ptm}{m}{n}
\rput(1.646249,-3.76){$\phi^{-}$}
\rput{-176.65169}(3.7486577,-5.770436){\psarc[linewidth=0.02](1.7900002,-2.9400005){0.37}{0.0}{158.21674}}
\psline[linewidth=0.02cm](11.34,-1.03)(11.34,-3.61)
\psline[linewidth=0.02cm,linecolor=color8](13.619635,-3.9964437)(9.62,1.23)
\rput{-160.37784}(23.858032,-0.92397004){\psarc[linewidth=0.02](11.849126,-2.5248597){0.53616136}{0.0}{180.0}}
\usefont{T1}{ptm}{m}{n}
\rput(10.876249,1.0400001){$O^{+}$}
\psline[linewidth=0.04cm](11.58,-0.81000006)(11.36,-0.5500001)
\psline[linewidth=0.04cm](11.38,-0.5500001)(11.14,-0.75000006)
\psdots[dotsize=0.06](11.34,-0.81000006)
\usefont{T1}{ptm}{m}{n}
\rput(10.54625,4.02){$y$}
\psline[linewidth=0.04cm,arrowsize=0.05291667cm 2.0,arrowlength=1.4,arrowinset=0.4]{->}(2.16,0.5899997)(6.68,0.5899997)
\psline[linewidth=0.04cm,arrowsize=0.05291667cm 2.0,arrowlength=1.4,arrowinset=0.4]{->}(11.34,0.5899997)(15.86,0.5899997)
\end{pspicture} 
}

}
  \caption{The vehicle is at the right side of the lane.}
  \label{fig:circular_mpc_rotation_car_at_right_side}
\end{figure}

In the second case, when the vehicle is on the left semilane, $CL-CR < 0$.
We retrieve the minimum distance from the range scan available at time $t$,
and denote the point which corresponds to this distance with $M$. The angle
between point $M$ and the longitudinal axis of the vehicle is denoted with
$\alpha$. In order to find this angle, we can exploit the fact that each ray
of the scan is separated from the next by $res$ degrees, while all of them
are stored in an array sequentially, in an anti-clockwise manner. In the case
of the \texttt{HOKUYO-UST-10LX-LASER}, the angular resolution is
$res=0.45^{\circ}$, and the starting angle is $-135^{\circ}$ with respect to the
longitudinal axis of the vehicle. This means that we can retrieve the angle
$\alpha$ by first calculating the number of indices between the one that
corresponds to the ray with the minimum range and the one that corresponds
to $+225^{\circ}$ (which in our case is $225 / 0.25 = 900$) with regard to
the laser's system of reference, and then by multiplying this number
$\Delta i$ by the angular resolution $res$. Hence,
$\alpha = 0.25 \times \Delta i$.

At this point we do not know whether the vehicle is pointing to the
left or to the right lane boundary. However, we can determine the sign and the
magnitude of the orientation of the vehicle with respect to the orientation of
the lane by examining the sign of the difference $\alpha - 90^{\circ}$:
if it is positive, the vehicle is pointing towards the right boundary lane,
if it is negative, towards the left. Furthermore, we can now
calculate the magnitude of the orientation of the vehicle as the difference
$|\alpha - 90^{\circ}|$, as shown in figure \ref{fig:circular_mpc_rotation_car_at_left_side}.
In conclusion, if the car is located at the left semilane its orientation with
respect to the orientation of the lane is

$$\phi = sign(CL-CR)(\alpha - 90)$$

\begin{figure}[H]\centering
  \scalebox{1}{% Generated with LaTeXDraw 2.0.8
% Tue Nov 22 22:07:18 CET 2016
% \usepackage[usenames,dvipsnames]{pstricks}
% \usepackage{epsfig}
% \usepackage{pst-grad} % For gradients
% \usepackage{pst-plot} % For axes
\scalebox{1} % Change this value to rescale the drawing.
{
\begin{pspicture}(0,-3.579375)(16.715,3.559375)
\definecolor{color8}{rgb}{0.00392156862745098,0.00392156862745098,0.00392156862745098}
\psline[linewidth=0.04cm](0.18,2.139375)(16.56,2.139375)
\psline[linewidth=0.04cm](0.0,-3.040625)(15.86,-3.040625)
\psline[linewidth=0.03cm,linestyle=dashed,dash=0.16cm 0.16cm](0.3,-0.440625)(16.7,-0.44062495)
\psline[linewidth=0.04cm,arrowsize=0.05291667cm 2.0,arrowlength=1.4,arrowinset=0.4]{->}(2.3,-0.44062495)(2.3,3.499375)
\usefont{T1}{ptm}{m}{n}
\rput(15.777186,-0.830625){$x$}
\usefont{T1}{ptm}{m}{n}
\rput(1.6371881,3.1293747){$y$}
\psline[linewidth=0.04cm,linecolor=color8,arrowsize=0.05291667cm 2.0,arrowlength=1.4,arrowinset=0.4]{->}(2.272057,0.6395523)(6.74,-0.920625)
\psline[linewidth=0.04cm,linecolor=color8](0.96,-3.020625)(3.36,3.539375)
\usefont{T1}{ptm}{m}{n}
\rput(3.3071873,2.369375){$L^{-}$}
\usefont{T1}{ptm}{m}{n}
\rput(0.8971875,-3.350625){$R^{-}$}
\rput{-204.45801}(11.8092375,-3.7407558){\psarc[linewidth=0.02](6.31,-0.590625){0.17}{127.092834}{263.65982}}
\usefont{T1}{ptm}{m}{n}
\rput(1.7971876,0.7493751){$C^{-}$}
\usefont{T1}{ptm}{m}{n}
\rput(2.2471871,-0.89062494){$O^{-}$}
\usefont{T1}{ptm}{m}{n}
\rput(7.0971866,-0.6706249){$\phi^{-}$}
\usefont{T1}{ptm}{m}{n}
\rput(1.8471874,2.369375){$M^{-}$}
\psline[linewidth=0.04cm,linecolor=color8](10.93792,0.09184203)(12.78,1.759375)
\psline[linewidth=0.04cm,linecolor=color8](13.3,-3.040625)(9.42,2.159375)
\psline[linewidth=0.04cm,arrowsize=0.05291667cm 2.0,arrowlength=1.4,arrowinset=0.4]{->}(10.96,-0.440625)(10.96,3.219375)
\usefont{T1}{ptm}{m}{n}
\rput(11.507188,2.389375){$M^{+}$}
\usefont{T1}{ptm}{m}{n}
\rput(9.407187,2.4093752){$L^{+}$}
\usefont{T1}{ptm}{m}{n}
\rput(13.237187,-3.370625){$R^{+}$}
\usefont{T1}{ptm}{m}{n}
\rput(10.6171875,-0.0506249){$C^{+}$}
\psarc[linewidth=0.02](2.49,0.929375){0.49}{295.70996}{111.297356}
\psarc[linewidth=0.02](11.15,0.709375){0.49}{2.29061}{111.297356}
\usefont{T1}{ptm}{m}{n}
\rput(10.417188,3.0493748){$y$}
\usefont{T1}{ptm}{m}{n}
\rput(10.887187,-0.87062496){$O^{+}$}
\psline[linewidth=0.04cm,linecolor=color8,arrowsize=0.05291667cm 2.0,arrowlength=1.4,arrowinset=0.4]{->}(12.097919,1.131842)(14.3,3.139375)
\usefont{T1}{ptm}{m}{n}
\rput(14.517186,2.4493752){$\phi^{+}$}
\psarc[linewidth=0.02](13.91,2.389375){0.29}{300.96375}{117.474434}
\psline[linewidth=0.04cm](2.62,0.519375)(2.48,0.159375)
\psline[linewidth=0.04cm](2.48,0.159375)(2.14,0.299375)
\psdots[dotsize=0.06](2.38,0.419375)
\psline[linewidth=0.04cm](11.42,0.099375)(11.14,-0.140625)
\psline[linewidth=0.04cm](11.42,0.079375)(11.22,0.339375)
\psdots[dotsize=0.06](11.18,0.119375)
\usefont{T1}{ptm}{m}{n}
\rput(10.457188,1.4893751){$\phi^{+}$}
\rput{93.50299}(12.272063,-9.72505){\psarc[linewidth=0.02](10.709999,0.9093752){0.29}{300.96375}{117.474434}}
\usefont{T1}{ptm}{m}{n}
\rput(2.6971865,3.129375){$\phi^{-}$}
\rput{93.50299}(5.052621,-0.19964838){\psarc[linewidth=0.02](2.620211,2.2765672){0.42896146}{300.96375}{41.497013}}
\usefont{T1}{ptm}{m}{n}
\rput(3.3590624,1.149375){$\alpha$}
\usefont{T1}{ptm}{m}{n}
\rput(11.579062,1.489375){$\alpha$}
\psline[linewidth=0.04cm,arrowsize=0.05291667cm 2.0,arrowlength=1.4,arrowinset=0.4]{->}(2.28,-0.440625)(7.2,-0.440625)
\psline[linewidth=0.04cm,arrowsize=0.05291667cm 2.0,arrowlength=1.4,arrowinset=0.4]{->}(10.96,-0.440625)(15.88,-0.440625)
\end{pspicture} 
}

}
  \caption{The vehicle is at the left side of the lane.}
  \label{fig:circular_mpc_rotation_car_at_left_side}
\end{figure}


In conclusion, angle $\phi = sign(CL-CR) (\alpha-90)$, where angle $\alpha$
is calculated as a (weighted by angular resolution) difference between the
indices of the minimum range provided by the range scan at time $t$ and that
of the minimum range between $CL$ and $CR$. \\


\textbf{Finding $OC$}

Looking at figure \ref{fig:centerline_mpc} we can see that

\begin{align}
  L_o O + OC + CR_o &= W = CR_o + CL_o
\end{align}

where $W$ is the width of the lane. But $L_o O = \dfrac{W}{2}$ hence

\begin{align}
  OC + CR_o &= \dfrac{1}{2}(CR_o + CL_o) \Leftrightarrow \\
  OC &= \dfrac{1}{2}(CL_o - CR_o)
\end{align}

But $CL_o = CL sin\lambda$ and $CR_o = CR cos\phi$, hence


\begin{align}
  OC &= \dfrac{1}{2}(CL sin\lambda - CR cos\phi)
\end{align}

From triangle LCF in the case where the vehicle is facing the left lane boundary,
we note that $\phi + \lambda + \dfrac{\pi}{2} = \pi$,
hence $\lambda = \dfrac{\pi}{2} - \phi$. Then, we conclude that

\begin{align}
  OC &= \dfrac{1}{2}(CL - CR) cos\phi
\end{align}




\textbf{Obtaining the linearized kinematic model}

The model constitutes the equations of motion of the vehicle, and has four
states ($x$, $y$, $v$ and $\psi$) and two inputs ($v_i$ and $\delta$). The
equations of the vehicle's motion that are relevant here are

\begin{align}
  \dot{x} &= v cos(\psi + \beta) \\
  \dot{y} &= v sin(\psi + \beta) \\
  \dot{v} &= \dfrac{v_i - v}{\tau} \\
  \dot{\psi} &= \dfrac{v}{l_r} sin\beta
\end{align}

Sampling with a sampling time of $T_s$ gives

\begin{align}
  x_{k+1} &= x_{k} + T_s v_k cos(\psi_k + \beta_k) \\
  y_{k+1} &= y_{k} + T_s v_k sin(\psi_k + \beta_k) \\
  v_{k+1} &= v_{k} + \dfrac{T_s}{\tau} (v_{i,k} - v_{k}) \\
  \psi_{k+1} &= \psi_{k} + T_s \dfrac{v}{l_r} sin\beta_k
\end{align}

where

\begin{align}
  \beta_k = tan^{-1}\Big(\dfrac{l_r}{l_r + l_f} tan\delta_{k-1}\Big)
\end{align}


Forming the Jacobians for matrices $A$, $B$ and evaluating them at time
$t=k$ around the state $\psi = \phi$, $v = v_k$ and
$\delta = \delta_{k-1}$ ($\delta_k$ is to be determined at time $k$):

\begin{equation}
 A =
  \begin{bmatrix}
    1 & 0 & T_s cos(\phi + \beta_k) & -T_s v_k sin(\phi + \beta_k) \\\\
    0 & 1 & T_s sin(\phi + \beta_k) & T_s v_k cos(\phi + \beta_k) \\\\
    0 & 0 & 1-\dfrac{T_s}{\tau} & 0 \\\\
    0 & 0 & \dfrac{T_s}{l_r}sin(\beta_k) & 1
  \end{bmatrix}
\end{equation}
\begin{equation}
  A =
  \begin{bmatrix}
    1 & 0 & T_s cos\Big(\phi + tan^{-1} (l_q tan\delta_{k-1})\Big) & -T_s v_k sin\Big(\psi_k + tan^{-1} (l_q tan\delta_{k-1})\Big) \\\\
    0 & 1 & T_s sin\Big(\phi + tan^{-1} (l_q tan\delta_{k-1})\Big) & T_s v_k cos\Big(\psi_k + tan^{-1} (l_q tan\delta_{k-1})\Big) \\\\
    0 & 0 & 1-\dfrac{T_s}{\tau} & 0 \\\\
    0 & 0 & \dfrac{T_s}{l_r}sin(tan^{-1} (l_q tan\delta_{k-1})) & 1
  \end{bmatrix}
\end{equation}


\begin{equation}
 B =
  \begin{bmatrix}
    0 & -T_s v_k sin(\phi + \beta_k) \dfrac{l_q}{l_q^2 sin^2\delta_{k-1} + cos^2\delta_{k-1}} \\
    0 & T_s v_k cos(\phi + \beta_k) \dfrac{l_q}{l_q^2 sin^2\delta_{k-1} + cos^2\delta_{k-1}} \\
    \dfrac{T_s}{\tau} v_k & 0 \\
    0 & \dfrac{T_s v_k}{l_r} cos(\beta_k) \dfrac{l_q}{l_q^2 sin^2\delta_{k-1} + cos^2\delta_{k-1}}
  \end{bmatrix}
\end{equation}
\begin{equation}
  B =
  \begin{bmatrix}
    0 & -T_s v_k sin\Big(\phi + tan^{-1} (l_q tan\delta_{k-1})\Big) \dfrac{l_q}{l_q^2 sin^2\delta_{k-1} + cos^2\delta_{k-1}} \\
    0 & T_s v_k cos\Big(\phi + tan^{-1} (l_q tan\delta_{k-1})\Big) \dfrac{l_q}{l_q^2 sin^2\delta_{k-1} + cos^2\delta_{k-1}} \\
    \dfrac{T_s}{\tau} v_k & 0 \\
    0 & \dfrac{T_s v_k}{l_r} cos\Bigg(tan^{-1} \Big(l_q tan\delta_{k-1}\Big)\Bigg) \dfrac{l_q}{l_q^2 sin^2\delta_{k-1} + cos^2\delta_{k-1}}
  \end{bmatrix}
\end{equation}


where $l_q = \dfrac{l_r}{l_r + l_f}$


Now we can express the linear model as

\begin{align}
  s_{k+1} = A s_k + B u_k
\end{align}

where

\begin{equation}
  s_k=
  \begin{bmatrix}
    x_{k} \\
    y_{k} \\
    v_{k} \\
    \psi_{k}
  \end{bmatrix},
  u_k=
  \begin{bmatrix}
    v_{i,k} \\
    \delta_{k}
  \end{bmatrix}
\end{equation}


However, states $x$ and $v$ cannot be measured under our configuration, since
no SLAM module is employed and no encoders are mounted on the wheels of the
vehicle. Hence, the model has to be reduced, while now the velocity will be
constant and set to $v_0$. The new system matrices and the new states are
modified as

\begin{equation}
  A =
  \begin{bmatrix}
    1 & T_s v_0 cos\Big(\phi + tan^{-1} (l_q tan\delta_{k-1})\Big) \\\\
    0 & 1
  \end{bmatrix}
\end{equation}

\begin{equation}
  B =
  \begin{bmatrix}
    0 & T_s v_0 cos\Big(\phi + tan^{-1} (l_q tan\delta_{k-1})\Big) \dfrac{l_q}{l_q^2 sin^2\delta_{k-1} + cos^2\delta_{k-1}} \\
    0 & \dfrac{T_s v_0}{l_r} cos\Bigg(tan^{-1} \Big(l_q tan\delta_{k-1}\Big)\Bigg) \dfrac{l_q}{l_q^2 sin^2\delta_{k-1} + cos^2\delta_{k-1}}
  \end{bmatrix}
\end{equation}

the model being

\begin{equation}
  s_{k+1} = A s_k + B u_k,
  s_k=
  \begin{bmatrix}
    v_{k} \\
    \psi_{k}
  \end{bmatrix},
  u_k=
  \begin{bmatrix}
    v_{i,k} \\
    \delta_{k}
  \end{bmatrix}
\end{equation}



\textbf{Stating the optimization problem}

We can now form the optimization problem to be solved at time step $t$ as

\begin{align}
  \text{minimize }    & \sum\limits_{k=0}^{N-1} (s_k - s_{ref})^T Q (s_k - s_{ref}) + u_k^T R u_k + (s_N-s_{ref})^T Q_f x_(s_N-s_{ref}) \\
  \text{subject to }  & s_{k+1} = A s_k + B u_k \\
                      & u_{i}^{min} \leq u_{i,k} \leq u_{i}^{max} \\
                      & \delta^{min} \leq \delta_k \leq \delta^{max} \\
                      & s_0 = (y_t, \psi_t) \equiv (OC, \phi)\\
                      & s_{ref} = (0,0)\\
                      & Q > 0, R > 0, Q_f > 0
\end{align}
