In figure \ref{fig:circular_mpc}, the vehicle $C$, whose velocity is constant
and denoted by $v$, is to track a circle whose center is $O'$ and whose
radius is $R$. Its orientation relative to the global coordinate system is
$\psi$. The vehicle's coordinates are $(x_c, y_c)$. The circular trajectory
is known a priori, and is comprised of a set of ordered points who encode
the $x$-wise, $y$-wise and $yaw$-wise states that the vehicle ought to have,
that is, each point on the circle serves as a reference to the vehicle. In terms
of yaw angle, the longitudinal axis of the vehicle ought to always be tangent
to the circle.


\begin{figure}[H]\centering
  \scalebox{0.8}{In figure \ref{fig:circular_mpc}, the vehicle $C$, whose velocity is constant
and denoted by $v$, is to track a circle whose center is $O'$ and whose
radius is $O'R$. Its orientation relative to the global coordinate system is
$\psi$. $R$ is the point $C$ is to track. The orientation of $R$ relative to
the global coordinate system is $\theta$. The vehicle's coordinates are
$(x_c, y_c)$. The aim is to find the vehicle's deviation from $R$ in terms of
translation along the $x$ and $y$ axes and rotation around the $x$-axis.

\begin{figure}[H]\centering
  \scalebox{0.8}{In figure \ref{fig:circular_mpc}, the vehicle $C$, whose velocity is constant
and denoted by $v$, is to track a circle whose center is $O'$ and whose
radius is $O'R$. Its orientation relative to the global coordinate system is
$\psi$. $R$ is the point $C$ is to track. The orientation of $R$ relative to
the global coordinate system is $\theta$. The vehicle's coordinates are
$(x_c, y_c)$. The aim is to find the vehicle's deviation from $R$ in terms of
translation along the $x$ and $y$ axes and rotation around the $x$-axis.

\begin{figure}[H]\centering
  \scalebox{0.8}{In figure \ref{fig:circular_mpc}, the vehicle $C$, whose velocity is constant
and denoted by $v$, is to track a circle whose center is $O'$ and whose
radius is $O'R$. Its orientation relative to the global coordinate system is
$\psi$. $R$ is the point $C$ is to track. The orientation of $R$ relative to
the global coordinate system is $\theta$. The vehicle's coordinates are
$(x_c, y_c)$. The aim is to find the vehicle's deviation from $R$ in terms of
translation along the $x$ and $y$ axes and rotation around the $x$-axis.

\begin{figure}[H]\centering
  \scalebox{0.8}{\input{./figures/circular_mpc.tex}}
  \caption{}
  \label{fig:circular_mpc}
\end{figure}

The solution to the problem can be decomposed into two separate components:
a translational and a rotational one.

\subsubsection{Translational component}

Here, we assume that the orientation of the tangent at $R$ is the same of that
of the vehicle and equal to $\psi$. Given the distance $RC$ (the coordinates of
both $R$ and $C$ are known) we are interested in finding the displacements
$RC_x$ and $RC_y$. From triangle $C_yRC$ we can immediately deduce that

\begin{align}
  RC_x &= RC sin\psi \\
  RC_y &= -RC cos\psi \\
\end{align}

where $RC = \sqrt{(x_C - x_R)^2 + (y_C - y_R)^2}$

\begin{figure}[H]\centering
  \scalebox{0.8}{\input{./figures/circular_mpc_translational.tex}}
  \caption{}
  \label{}
\end{figure}


\subsubsection{Rotational component}

Here, we assume that the only deviation of $C$ from $R$ is only in terms of
orientation. From figure \ref{fig:circular_mp_rotational} we can easily
discern that the angular error is $\theta -\psi$.

\begin{figure}[H]\centering
  \scalebox{0.8}{\input{./figures/circular_mpc_rotational.tex}}
  \caption{}
  \label{fig:circular_mp_rotational}
\end{figure}


\subsubsection{Obtaining the relevant linearized kinematic model}

The model constitutes the equations of motion of the vehicle, and has three
states ($x$, $y$ and $\psi$) and one input ($\delta$). The equations of the
vehicle's motion that are relevant here are

\begin{align}
  \dot{x} &= v cos(\psi + \beta) \\
  \dot{y} &= v sin(\psi + \beta) \\
  \dot{\psi} &= \dfrac{v}{l_r} sin\beta
\end{align}

Sampling with a sampling time of $T_s$ gives

\begin{align}
  x_{k+1} &= x_{k} + T_s v cos(\psi_k + \beta_k) \\
  y_{k+1} &= y_{k} + T_s v sin(\psi_k + \beta_k) \\
  \psi_{k+1} &= \psi_{k} + T_s \dfrac{v}{l_r} sin\beta_k
\end{align}

where

\begin{align}
  \beta_k = tan^{-1}\Big(\dfrac{l_r}{l_r + l_f} tan\delta_k\Big)
\end{align}


Forming the Jacobians for matrices $A$, $B$ and evaluating them at time
$t=k$ around $\delta = 0$ (which makes $\beta = 0$) gives

\begin{equation}
 A =
  \begin{bmatrix}
    1 & 0 & -T_s v sin(\psi_k + \beta_k) \\\\
    0 & 1 & T_s v cos(\psi_k + \beta_k) \\\\
    0 & 0 & 1
  \end{bmatrix}
  =
  \begin{bmatrix}
    1 & 0 & -T_s v sin\Big(\psi_k + tan^{-1} (l_q tan\delta_k)\Big) \\\\
    0 & 1 & T_s v cos\Big(\psi_k + tan^{-1} (l_q tan\delta_k)\Big) \\\\
    0 & 0 & 1
  \end{bmatrix}
\end{equation}
\begin{equation}
 A =
  \begin{bmatrix}
    1 & 0 & -T_s v sin(\psi_k) \\\\
    0 & 1 & T_s v cos(\psi_k) \\\\
    0 & 0 & 1
  \end{bmatrix}
\end{equation}

\begin{equation}
 B =
  \begin{bmatrix}
    -T_s v sin(\psi_k + \beta_k) \dfrac{l_q}{l_q^2 sin^2\delta_k + cos^2\delta_k} \\
    T_s v cos(\psi_k + \beta_k) \dfrac{l_q}{l_q^2 sin^2\delta_k + cos^2\delta_k} \\
    \dfrac{T_s v}{l_r} cos(\beta_k) \dfrac{l_q}{l_q^2 sin^2\delta_k + cos^2\delta_k}
  \end{bmatrix}
  =
  \begin{bmatrix}
    -T_s v sin\Big(\psi_k + tan^{-1} (l_q tan\delta_k)\Big) \dfrac{l_q}{l_q^2 sin^2\delta_k + cos^2\delta_k} \\
    T_s v cos\Big(\psi_k + tan^{-1} (l_q tan\delta_k)\Big) \dfrac{l_q}{l_q^2 sin^2\delta_k + cos^2\delta_k} \\
    \dfrac{T_s v}{l_r} cos\Bigg(tan^{-1} \Big(l_q tan\delta_k\Big)\Bigg) \dfrac{l_q}{l_q^2 sin^2\delta_k + cos^2\delta_k}
  \end{bmatrix}
\end{equation}
\begin{equation}
 B =
  \begin{bmatrix}
    -\dfrac{T_s l_r v}{l_r + l_f} sin\psi_k \\\\
    \dfrac{T_s l_r v}{l_r + l_f} cos\psi_k \\\\
    \dfrac{T_s v}{l_r+l_f}
  \end{bmatrix}
\end{equation}

where $l_q = \dfrac{l_r}{l_r + l_f}$




Now we can express the linear model as

\begin{align}
  s_{k+1} = A s_k + B \delta_k
\end{align}

where

\begin{equation}
  s=
  \begin{bmatrix}
    x_{k} \\
    y_{k} \\
    \psi_{k}
  \end{bmatrix}
\end{equation}

or

\begin{equation}
  \begin{bmatrix}
    x_{k+1} \\
    y_{k+1} \\
    \psi_{k+1}
  \end{bmatrix}
  =
  \begin{bmatrix}
    1 & 0 & -T_s v sin(\psi_k) \\\\
    0 & 1 & T_s v cos(\psi_k) \\\\
    0 & 0 & 1
  \end{bmatrix}
  \begin{bmatrix}
    x_{k} \\
    y_{k} \\
    \psi_{k}
  \end{bmatrix}
  +
  \begin{bmatrix}
    -\dfrac{T_s l_r v}{l_r + l_f} sin\psi_k \\\\
    \dfrac{T_s l_r v}{l_r + l_f} cos\psi_k \\\\
    \dfrac{T_s v}{l_r+l_f}
  \end{bmatrix}
  \delta_{k}
\end{equation}



\subsubsection{Stating the optimization problem}

We can now form the optimization problem as

\begin{align}
  \text{minimize }    & \sum\limits_{k=0}^N x_k^2 q_x + y_k^2 q_y + \phi_k^2 q_{\phi} + \delta_k^2 r \\
  \text{subject to }  & s_{k+1} = A s_k + B \delta_k,\text{ where } s_k = [x_k, y_k, \phi_k]^T \\
                      & \delta_{min} \leq \delta_k \leq \delta_{max} \\
                      & x_0 = RC sin\phi_0 \\
                      & y_0 = -RC cos\phi_0 \\
                      & \phi_0 = \theta_R - \psi
\end{align}
}
  \caption{}
  \label{fig:circular_mpc}
\end{figure}

The solution to the problem can be decomposed into two separate components:
a translational and a rotational one.

\subsubsection{Translational component}

Here, we assume that the orientation of the tangent at $R$ is the same of that
of the vehicle and equal to $\psi$. Given the distance $RC$ (the coordinates of
both $R$ and $C$ are known) we are interested in finding the displacements
$RC_x$ and $RC_y$. From triangle $C_yRC$ we can immediately deduce that

\begin{align}
  RC_x &= RC sin\psi \\
  RC_y &= -RC cos\psi \\
\end{align}

where $RC = \sqrt{(x_C - x_R)^2 + (y_C - y_R)^2}$

\begin{figure}[H]\centering
  \scalebox{0.8}{% Generated with LaTeXDraw 2.0.8
% Fri Nov 18 15:10:06 CET 2016
% \usepackage[usenames,dvipsnames]{pstricks}
% \usepackage{epsfig}
% \usepackage{pst-grad} % For gradients
% \usepackage{pst-plot} % For axes
\scalebox{1} % Change this value to rescale the drawing.
{
\begin{pspicture}(0,-7.3264065)(23.47,7.306406)
\pscircle[linewidth=0.04,dimen=outer](8.36,0.61640626){4.63}
\psline[linewidth=0.04cm,arrowsize=0.05291667cm 2.0,arrowlength=1.4,arrowinset=0.4]{<-}(0.97,7.286406)(0.97,-6.5135937)
\psline[linewidth=0.04cm,arrowsize=0.05291667cm 2.0,arrowlength=1.4,arrowinset=0.4]{->}(0.97,-6.5135937)(23.45,-6.5135937)
\psline[linewidth=0.04cm,arrowsize=0.05291667cm 2.0,arrowlength=1.4,arrowinset=0.4]{->}(15.11,-3.0335937)(18.47,2.1264062)
\psdots[dotsize=0.12](15.11,-3.0535936)
\usefont{T1}{ptm}{m}{n}
\rput(0.27453125,6.6764064){$y$}
\usefont{T1}{ptm}{m}{n}
\rput(22.734531,-7.123594){$x$}
\psdots[dotsize=0.12](8.36,0.61640626)
\psline[linewidth=0.02cm](8.35,0.6064063)(15.11,-3.0535936)
\usefont{T1}{ptm}{m}{n}
\rput(0.58453125,-7.043594){$O$}
\usefont{T1}{ptm}{m}{n}
\rput(14.994532,-3.8235939){$C$}
\usefont{T1}{ptm}{m}{n}
\rput(7.6045313,0.03640625){$O'$}
\psline[linewidth=0.02cm](9.57,-6.5135937)(15.37,3.4864063)
\psline[linewidth=0.02cm,arrowsize=0.05291667cm 2.0,arrowlength=1.4,arrowinset=0.4]{->}(12.39,-5.193594)(12.39,4.726406)
\psline[linewidth=0.02cm,arrowsize=0.05291667cm 2.0,arrowlength=1.4,arrowinset=0.4]{->}(10.19,-1.5735937)(19.53,-1.5935937)
\psline[linewidth=0.02cm,linestyle=dashed,dash=0.16cm 0.16cm](12.39,-3.0535936)(15.05,-3.0535936)
\psline[linewidth=0.02cm,linestyle=dashed,dash=0.16cm 0.16cm](15.11,-3.0535936)(15.11,-1.5735937)
\usefont{T1}{ptm}{m}{n}
\rput(11.694531,-1.8435937){$R$}
\usefont{T1}{ptm}{m}{n}
\rput(11.994532,-3.3235939){$C_y$}
\usefont{T1}{ptm}{m}{n}
\rput(15.094531,-1.2235937){$C_x$}
\psarc[linewidth=0.02](16.43,-1.4135938){0.24}{315.0}{113.62938}
\usefont{T1}{ptm}{m}{n}
\rput(17.244532,-1.2635938){$\psi$}
\usefont{T1}{ptm}{m}{n}
\rput(13.004531,-2.4635937){$\psi$}
\psarc[linewidth=0.02](12.6,-1.9635937){0.25}{218.6598}{42.27369}
\psarc[linewidth=0.02](12.77,-1.4135938){0.24}{315.0}{113.62938}
\usefont{T1}{ptm}{m}{n}
\rput(13.584531,-1.2635938){$\psi$}
\psline[linewidth=0.02cm](12.39,-1.0935937)(11.87,-1.0935937)
\psline[linewidth=0.02cm](11.85,-1.0935937)(11.85,-1.5735937)
\psdots[dotsize=0.12](12.11,-1.3335937)
\end{pspicture} 
}

}
  \caption{}
  \label{}
\end{figure}


\subsubsection{Rotational component}

Here, we assume that the only deviation of $C$ from $R$ is only in terms of
orientation. From figure \ref{fig:circular_mp_rotational} we can easily
discern that the angular error is $\theta -\psi$.

\begin{figure}[H]\centering
  \scalebox{0.8}{% Generated with LaTeXDraw 2.0.8
% Fri Nov 18 16:47:11 CET 2016
% \usepackage[usenames,dvipsnames]{pstricks}
% \usepackage{epsfig}
% \usepackage{pst-grad} % For gradients
% \usepackage{pst-plot} % For axes
\scalebox{1} % Change this value to rescale the drawing.
{
\begin{pspicture}(0,-7.3264065)(23.597345,7.306406)
\pscircle[linewidth=0.04,dimen=outer](8.487344,0.6164076){4.63}
\psline[linewidth=0.04cm,arrowsize=0.05291667cm 2.0,arrowlength=1.4,arrowinset=0.4]{<-}(1.0973437,7.286406)(1.0973437,-6.5135937)
\psline[linewidth=0.04cm,arrowsize=0.05291667cm 2.0,arrowlength=1.4,arrowinset=0.4]{->}(1.0973437,-6.5135937)(23.577345,-6.5135937)
\psline[linewidth=0.04cm,arrowsize=0.05291667cm 2.0,arrowlength=1.4,arrowinset=0.4]{->}(12.530937,-1.6135932)(15.837343,0.94640744)
\usefont{T1}{ptm}{m}{n}
\rput(0.27453125,6.6764073){$y$}
\usefont{T1}{ptm}{m}{n}
\rput(22.734531,-7.123594){$x$}
\psdots[dotsize=0.12](8.487344,0.6164076)
\psline[linewidth=0.02cm](8.4773445,0.6064075)(12.5364065,-1.613593)
\usefont{T1}{ptm}{m}{n}
\rput(0.58453125,-7.043594){$O$}
\usefont{T1}{ptm}{m}{n}
\rput(8.004531,0.0164075){$O'$}
\psline[linewidth=0.02cm,linestyle=dashed,dash=0.17638889cm 0.10583334cm](6.1373444,-6.5135937)(12.96,-1.2735938)
\psarc[linewidth=0.02](7.1018753,-6.213594){0.38}{310.9144}{99.09028}
\usefont{T1}{ptm}{m}{n}
\rput(8.034531,-5.943594){$\psi$}
\psline[linewidth=0.02cm](9.801874,-6.5135937)(13.541874,0.24640727)
\psline[linewidth=0.02cm](12.601875,-1.4535928)(16.341875,5.306406)
\psarc[linewidth=0.02](10.241876,-6.213594){0.38}{310.9144}{99.09028}
\usefont{T1}{ptm}{m}{n}
\rput(11.514531,-5.9235926){$\theta$}
\usefont{T1}{ptm}{m}{n}
\rput(10.764531,-1.6435927){$R \equiv C$}
\usefont{T1}{ptm}{m}{n}
\rput(14.304531,0.45640698){$\theta - \psi$}
\psarc[linewidth=0.02](13.546407,-0.443593){0.35}{340.46335}{130.76361}
\psline[linewidth=0.04cm](11.976405,-1.293593)(12.236406,-0.7735928)
\psline[linewidth=0.04cm](12.256407,-0.79359293)(12.756407,-1.073593)
\psdots[dotsize=0.12](12.376407,-1.213593)
\end{pspicture} 
}

}
  \caption{}
  \label{fig:circular_mp_rotational}
\end{figure}


\subsubsection{Obtaining the relevant linearized kinematic model}

The model constitutes the equations of motion of the vehicle, and has three
states ($x$, $y$ and $\psi$) and one input ($\delta$). The equations of the
vehicle's motion that are relevant here are

\begin{align}
  \dot{x} &= v cos(\psi + \beta) \\
  \dot{y} &= v sin(\psi + \beta) \\
  \dot{\psi} &= \dfrac{v}{l_r} sin\beta
\end{align}

Sampling with a sampling time of $T_s$ gives

\begin{align}
  x_{k+1} &= x_{k} + T_s v cos(\psi_k + \beta_k) \\
  y_{k+1} &= y_{k} + T_s v sin(\psi_k + \beta_k) \\
  \psi_{k+1} &= \psi_{k} + T_s \dfrac{v}{l_r} sin\beta_k
\end{align}

where

\begin{align}
  \beta_k = tan^{-1}\Big(\dfrac{l_r}{l_r + l_f} tan\delta_k\Big)
\end{align}


Forming the Jacobians for matrices $A$, $B$ and evaluating them at time
$t=k$ around $\delta = 0$ (which makes $\beta = 0$) gives

\begin{equation}
 A =
  \begin{bmatrix}
    1 & 0 & -T_s v sin(\psi_k + \beta_k) \\\\
    0 & 1 & T_s v cos(\psi_k + \beta_k) \\\\
    0 & 0 & 1
  \end{bmatrix}
  =
  \begin{bmatrix}
    1 & 0 & -T_s v sin\Big(\psi_k + tan^{-1} (l_q tan\delta_k)\Big) \\\\
    0 & 1 & T_s v cos\Big(\psi_k + tan^{-1} (l_q tan\delta_k)\Big) \\\\
    0 & 0 & 1
  \end{bmatrix}
\end{equation}
\begin{equation}
 A =
  \begin{bmatrix}
    1 & 0 & -T_s v sin(\psi_k) \\\\
    0 & 1 & T_s v cos(\psi_k) \\\\
    0 & 0 & 1
  \end{bmatrix}
\end{equation}

\begin{equation}
 B =
  \begin{bmatrix}
    -T_s v sin(\psi_k + \beta_k) \dfrac{l_q}{l_q^2 sin^2\delta_k + cos^2\delta_k} \\
    T_s v cos(\psi_k + \beta_k) \dfrac{l_q}{l_q^2 sin^2\delta_k + cos^2\delta_k} \\
    \dfrac{T_s v}{l_r} cos(\beta_k) \dfrac{l_q}{l_q^2 sin^2\delta_k + cos^2\delta_k}
  \end{bmatrix}
  =
  \begin{bmatrix}
    -T_s v sin\Big(\psi_k + tan^{-1} (l_q tan\delta_k)\Big) \dfrac{l_q}{l_q^2 sin^2\delta_k + cos^2\delta_k} \\
    T_s v cos\Big(\psi_k + tan^{-1} (l_q tan\delta_k)\Big) \dfrac{l_q}{l_q^2 sin^2\delta_k + cos^2\delta_k} \\
    \dfrac{T_s v}{l_r} cos\Bigg(tan^{-1} \Big(l_q tan\delta_k\Big)\Bigg) \dfrac{l_q}{l_q^2 sin^2\delta_k + cos^2\delta_k}
  \end{bmatrix}
\end{equation}
\begin{equation}
 B =
  \begin{bmatrix}
    -\dfrac{T_s l_r v}{l_r + l_f} sin\psi_k \\\\
    \dfrac{T_s l_r v}{l_r + l_f} cos\psi_k \\\\
    \dfrac{T_s v}{l_r+l_f}
  \end{bmatrix}
\end{equation}

where $l_q = \dfrac{l_r}{l_r + l_f}$




Now we can express the linear model as

\begin{align}
  s_{k+1} = A s_k + B \delta_k
\end{align}

where

\begin{equation}
  s=
  \begin{bmatrix}
    x_{k} \\
    y_{k} \\
    \psi_{k}
  \end{bmatrix}
\end{equation}

or

\begin{equation}
  \begin{bmatrix}
    x_{k+1} \\
    y_{k+1} \\
    \psi_{k+1}
  \end{bmatrix}
  =
  \begin{bmatrix}
    1 & 0 & -T_s v sin(\psi_k) \\\\
    0 & 1 & T_s v cos(\psi_k) \\\\
    0 & 0 & 1
  \end{bmatrix}
  \begin{bmatrix}
    x_{k} \\
    y_{k} \\
    \psi_{k}
  \end{bmatrix}
  +
  \begin{bmatrix}
    -\dfrac{T_s l_r v}{l_r + l_f} sin\psi_k \\\\
    \dfrac{T_s l_r v}{l_r + l_f} cos\psi_k \\\\
    \dfrac{T_s v}{l_r+l_f}
  \end{bmatrix}
  \delta_{k}
\end{equation}



\subsubsection{Stating the optimization problem}

We can now form the optimization problem as

\begin{align}
  \text{minimize }    & \sum\limits_{k=0}^N x_k^2 q_x + y_k^2 q_y + \phi_k^2 q_{\phi} + \delta_k^2 r \\
  \text{subject to }  & s_{k+1} = A s_k + B \delta_k,\text{ where } s_k = [x_k, y_k, \phi_k]^T \\
                      & \delta_{min} \leq \delta_k \leq \delta_{max} \\
                      & x_0 = RC sin\phi_0 \\
                      & y_0 = -RC cos\phi_0 \\
                      & \phi_0 = \theta_R - \psi
\end{align}
}
  \caption{}
  \label{fig:circular_mpc}
\end{figure}

The solution to the problem can be decomposed into two separate components:
a translational and a rotational one.

\subsubsection{Translational component}

Here, we assume that the orientation of the tangent at $R$ is the same of that
of the vehicle and equal to $\psi$. Given the distance $RC$ (the coordinates of
both $R$ and $C$ are known) we are interested in finding the displacements
$RC_x$ and $RC_y$. From triangle $C_yRC$ we can immediately deduce that

\begin{align}
  RC_x &= RC sin\psi \\
  RC_y &= -RC cos\psi \\
\end{align}

where $RC = \sqrt{(x_C - x_R)^2 + (y_C - y_R)^2}$

\begin{figure}[H]\centering
  \scalebox{0.8}{% Generated with LaTeXDraw 2.0.8
% Fri Nov 18 15:10:06 CET 2016
% \usepackage[usenames,dvipsnames]{pstricks}
% \usepackage{epsfig}
% \usepackage{pst-grad} % For gradients
% \usepackage{pst-plot} % For axes
\scalebox{1} % Change this value to rescale the drawing.
{
\begin{pspicture}(0,-7.3264065)(23.47,7.306406)
\pscircle[linewidth=0.04,dimen=outer](8.36,0.61640626){4.63}
\psline[linewidth=0.04cm,arrowsize=0.05291667cm 2.0,arrowlength=1.4,arrowinset=0.4]{<-}(0.97,7.286406)(0.97,-6.5135937)
\psline[linewidth=0.04cm,arrowsize=0.05291667cm 2.0,arrowlength=1.4,arrowinset=0.4]{->}(0.97,-6.5135937)(23.45,-6.5135937)
\psline[linewidth=0.04cm,arrowsize=0.05291667cm 2.0,arrowlength=1.4,arrowinset=0.4]{->}(15.11,-3.0335937)(18.47,2.1264062)
\psdots[dotsize=0.12](15.11,-3.0535936)
\usefont{T1}{ptm}{m}{n}
\rput(0.27453125,6.6764064){$y$}
\usefont{T1}{ptm}{m}{n}
\rput(22.734531,-7.123594){$x$}
\psdots[dotsize=0.12](8.36,0.61640626)
\psline[linewidth=0.02cm](8.35,0.6064063)(15.11,-3.0535936)
\usefont{T1}{ptm}{m}{n}
\rput(0.58453125,-7.043594){$O$}
\usefont{T1}{ptm}{m}{n}
\rput(14.994532,-3.8235939){$C$}
\usefont{T1}{ptm}{m}{n}
\rput(7.6045313,0.03640625){$O'$}
\psline[linewidth=0.02cm](9.57,-6.5135937)(15.37,3.4864063)
\psline[linewidth=0.02cm,arrowsize=0.05291667cm 2.0,arrowlength=1.4,arrowinset=0.4]{->}(12.39,-5.193594)(12.39,4.726406)
\psline[linewidth=0.02cm,arrowsize=0.05291667cm 2.0,arrowlength=1.4,arrowinset=0.4]{->}(10.19,-1.5735937)(19.53,-1.5935937)
\psline[linewidth=0.02cm,linestyle=dashed,dash=0.16cm 0.16cm](12.39,-3.0535936)(15.05,-3.0535936)
\psline[linewidth=0.02cm,linestyle=dashed,dash=0.16cm 0.16cm](15.11,-3.0535936)(15.11,-1.5735937)
\usefont{T1}{ptm}{m}{n}
\rput(11.694531,-1.8435937){$R$}
\usefont{T1}{ptm}{m}{n}
\rput(11.994532,-3.3235939){$C_y$}
\usefont{T1}{ptm}{m}{n}
\rput(15.094531,-1.2235937){$C_x$}
\psarc[linewidth=0.02](16.43,-1.4135938){0.24}{315.0}{113.62938}
\usefont{T1}{ptm}{m}{n}
\rput(17.244532,-1.2635938){$\psi$}
\usefont{T1}{ptm}{m}{n}
\rput(13.004531,-2.4635937){$\psi$}
\psarc[linewidth=0.02](12.6,-1.9635937){0.25}{218.6598}{42.27369}
\psarc[linewidth=0.02](12.77,-1.4135938){0.24}{315.0}{113.62938}
\usefont{T1}{ptm}{m}{n}
\rput(13.584531,-1.2635938){$\psi$}
\psline[linewidth=0.02cm](12.39,-1.0935937)(11.87,-1.0935937)
\psline[linewidth=0.02cm](11.85,-1.0935937)(11.85,-1.5735937)
\psdots[dotsize=0.12](12.11,-1.3335937)
\end{pspicture} 
}

}
  \caption{}
  \label{}
\end{figure}


\subsubsection{Rotational component}

Here, we assume that the only deviation of $C$ from $R$ is only in terms of
orientation. From figure \ref{fig:circular_mp_rotational} we can easily
discern that the angular error is $\theta -\psi$.

\begin{figure}[H]\centering
  \scalebox{0.8}{% Generated with LaTeXDraw 2.0.8
% Fri Nov 18 16:47:11 CET 2016
% \usepackage[usenames,dvipsnames]{pstricks}
% \usepackage{epsfig}
% \usepackage{pst-grad} % For gradients
% \usepackage{pst-plot} % For axes
\scalebox{1} % Change this value to rescale the drawing.
{
\begin{pspicture}(0,-7.3264065)(23.597345,7.306406)
\pscircle[linewidth=0.04,dimen=outer](8.487344,0.6164076){4.63}
\psline[linewidth=0.04cm,arrowsize=0.05291667cm 2.0,arrowlength=1.4,arrowinset=0.4]{<-}(1.0973437,7.286406)(1.0973437,-6.5135937)
\psline[linewidth=0.04cm,arrowsize=0.05291667cm 2.0,arrowlength=1.4,arrowinset=0.4]{->}(1.0973437,-6.5135937)(23.577345,-6.5135937)
\psline[linewidth=0.04cm,arrowsize=0.05291667cm 2.0,arrowlength=1.4,arrowinset=0.4]{->}(12.530937,-1.6135932)(15.837343,0.94640744)
\usefont{T1}{ptm}{m}{n}
\rput(0.27453125,6.6764073){$y$}
\usefont{T1}{ptm}{m}{n}
\rput(22.734531,-7.123594){$x$}
\psdots[dotsize=0.12](8.487344,0.6164076)
\psline[linewidth=0.02cm](8.4773445,0.6064075)(12.5364065,-1.613593)
\usefont{T1}{ptm}{m}{n}
\rput(0.58453125,-7.043594){$O$}
\usefont{T1}{ptm}{m}{n}
\rput(8.004531,0.0164075){$O'$}
\psline[linewidth=0.02cm,linestyle=dashed,dash=0.17638889cm 0.10583334cm](6.1373444,-6.5135937)(12.96,-1.2735938)
\psarc[linewidth=0.02](7.1018753,-6.213594){0.38}{310.9144}{99.09028}
\usefont{T1}{ptm}{m}{n}
\rput(8.034531,-5.943594){$\psi$}
\psline[linewidth=0.02cm](9.801874,-6.5135937)(13.541874,0.24640727)
\psline[linewidth=0.02cm](12.601875,-1.4535928)(16.341875,5.306406)
\psarc[linewidth=0.02](10.241876,-6.213594){0.38}{310.9144}{99.09028}
\usefont{T1}{ptm}{m}{n}
\rput(11.514531,-5.9235926){$\theta$}
\usefont{T1}{ptm}{m}{n}
\rput(10.764531,-1.6435927){$R \equiv C$}
\usefont{T1}{ptm}{m}{n}
\rput(14.304531,0.45640698){$\theta - \psi$}
\psarc[linewidth=0.02](13.546407,-0.443593){0.35}{340.46335}{130.76361}
\psline[linewidth=0.04cm](11.976405,-1.293593)(12.236406,-0.7735928)
\psline[linewidth=0.04cm](12.256407,-0.79359293)(12.756407,-1.073593)
\psdots[dotsize=0.12](12.376407,-1.213593)
\end{pspicture} 
}

}
  \caption{}
  \label{fig:circular_mp_rotational}
\end{figure}


\subsubsection{Obtaining the relevant linearized kinematic model}

The model constitutes the equations of motion of the vehicle, and has three
states ($x$, $y$ and $\psi$) and one input ($\delta$). The equations of the
vehicle's motion that are relevant here are

\begin{align}
  \dot{x} &= v cos(\psi + \beta) \\
  \dot{y} &= v sin(\psi + \beta) \\
  \dot{\psi} &= \dfrac{v}{l_r} sin\beta
\end{align}

Sampling with a sampling time of $T_s$ gives

\begin{align}
  x_{k+1} &= x_{k} + T_s v cos(\psi_k + \beta_k) \\
  y_{k+1} &= y_{k} + T_s v sin(\psi_k + \beta_k) \\
  \psi_{k+1} &= \psi_{k} + T_s \dfrac{v}{l_r} sin\beta_k
\end{align}

where

\begin{align}
  \beta_k = tan^{-1}\Big(\dfrac{l_r}{l_r + l_f} tan\delta_k\Big)
\end{align}


Forming the Jacobians for matrices $A$, $B$ and evaluating them at time
$t=k$ around $\delta = 0$ (which makes $\beta = 0$) gives

\begin{equation}
 A =
  \begin{bmatrix}
    1 & 0 & -T_s v sin(\psi_k + \beta_k) \\\\
    0 & 1 & T_s v cos(\psi_k + \beta_k) \\\\
    0 & 0 & 1
  \end{bmatrix}
  =
  \begin{bmatrix}
    1 & 0 & -T_s v sin\Big(\psi_k + tan^{-1} (l_q tan\delta_k)\Big) \\\\
    0 & 1 & T_s v cos\Big(\psi_k + tan^{-1} (l_q tan\delta_k)\Big) \\\\
    0 & 0 & 1
  \end{bmatrix}
\end{equation}
\begin{equation}
 A =
  \begin{bmatrix}
    1 & 0 & -T_s v sin(\psi_k) \\\\
    0 & 1 & T_s v cos(\psi_k) \\\\
    0 & 0 & 1
  \end{bmatrix}
\end{equation}

\begin{equation}
 B =
  \begin{bmatrix}
    -T_s v sin(\psi_k + \beta_k) \dfrac{l_q}{l_q^2 sin^2\delta_k + cos^2\delta_k} \\
    T_s v cos(\psi_k + \beta_k) \dfrac{l_q}{l_q^2 sin^2\delta_k + cos^2\delta_k} \\
    \dfrac{T_s v}{l_r} cos(\beta_k) \dfrac{l_q}{l_q^2 sin^2\delta_k + cos^2\delta_k}
  \end{bmatrix}
  =
  \begin{bmatrix}
    -T_s v sin\Big(\psi_k + tan^{-1} (l_q tan\delta_k)\Big) \dfrac{l_q}{l_q^2 sin^2\delta_k + cos^2\delta_k} \\
    T_s v cos\Big(\psi_k + tan^{-1} (l_q tan\delta_k)\Big) \dfrac{l_q}{l_q^2 sin^2\delta_k + cos^2\delta_k} \\
    \dfrac{T_s v}{l_r} cos\Bigg(tan^{-1} \Big(l_q tan\delta_k\Big)\Bigg) \dfrac{l_q}{l_q^2 sin^2\delta_k + cos^2\delta_k}
  \end{bmatrix}
\end{equation}
\begin{equation}
 B =
  \begin{bmatrix}
    -\dfrac{T_s l_r v}{l_r + l_f} sin\psi_k \\\\
    \dfrac{T_s l_r v}{l_r + l_f} cos\psi_k \\\\
    \dfrac{T_s v}{l_r+l_f}
  \end{bmatrix}
\end{equation}

where $l_q = \dfrac{l_r}{l_r + l_f}$




Now we can express the linear model as

\begin{align}
  s_{k+1} = A s_k + B \delta_k
\end{align}

where

\begin{equation}
  s=
  \begin{bmatrix}
    x_{k} \\
    y_{k} \\
    \psi_{k}
  \end{bmatrix}
\end{equation}

or

\begin{equation}
  \begin{bmatrix}
    x_{k+1} \\
    y_{k+1} \\
    \psi_{k+1}
  \end{bmatrix}
  =
  \begin{bmatrix}
    1 & 0 & -T_s v sin(\psi_k) \\\\
    0 & 1 & T_s v cos(\psi_k) \\\\
    0 & 0 & 1
  \end{bmatrix}
  \begin{bmatrix}
    x_{k} \\
    y_{k} \\
    \psi_{k}
  \end{bmatrix}
  +
  \begin{bmatrix}
    -\dfrac{T_s l_r v}{l_r + l_f} sin\psi_k \\\\
    \dfrac{T_s l_r v}{l_r + l_f} cos\psi_k \\\\
    \dfrac{T_s v}{l_r+l_f}
  \end{bmatrix}
  \delta_{k}
\end{equation}



\subsubsection{Stating the optimization problem}

We can now form the optimization problem as

\begin{align}
  \text{minimize }    & \sum\limits_{k=0}^N x_k^2 q_x + y_k^2 q_y + \phi_k^2 q_{\phi} + \delta_k^2 r \\
  \text{subject to }  & s_{k+1} = A s_k + B \delta_k,\text{ where } s_k = [x_k, y_k, \phi_k]^T \\
                      & \delta_{min} \leq \delta_k \leq \delta_{max} \\
                      & x_0 = RC sin\phi_0 \\
                      & y_0 = -RC cos\phi_0 \\
                      & \phi_0 = \theta_R - \psi
\end{align}
}
  \caption{}
  \label{fig:circular_mpc}
\end{figure}

In this setting, since an optimization problem will have to be solved at each
time instant $t$, the first thing that needs to be identified is the reference
pose of the vehicle at each time instant $t$. More precisely, since the
objective is the minimization of the deviation of the vehicle's trajectory
from the circular one, and the framework is that of MPC, we need to identify
the sequence of points on the circle that shall act as references while
iterating through the prediction equations of the linearized model of the
vehicle's behaviour. This task is divided into two sub-tasks: first, we
identify the point $T$ of the circle that lies on the line that is tangent to
the circle and that passes through the position of the vehicle. When this
point is known, the sequence of $N-1$ reference poses can be calculated,
given knowledge of the circle's radius and the vehicle's velocity. $N$ is the
horizon length. The following two sections illustrate the procedures by
which the $N$ reference poses are identified.

\subsubsection{Finding $T$}

Here, we are not concerned with the orientation of the vehicle; its coordinates
will only be used as a point to identify the tangent that passes through
the position of the vehicle.


\begin{figure}[H]\centering
  \scalebox{0.8}{% Generated with LaTeXDraw 2.0.8
% Mon Dec 05 00:44:20 CET 2016
% \usepackage[usenames,dvipsnames]{pstricks}
% \usepackage{epsfig}
% \usepackage{pst-grad} % For gradients
% \usepackage{pst-plot} % For axes
\scalebox{1} % Change this value to rescale the drawing.
{
\begin{pspicture}(0,-7.3264065)(23.597345,7.306406)
\pscircle[linewidth=0.04,dimen=outer](8.487344,0.6164076){4.63}
\psline[linewidth=0.04cm,arrowsize=0.05291667cm 2.0,arrowlength=1.4,arrowinset=0.4]{<-}(1.0973437,7.286406)(1.0973437,-6.5135937)
\psline[linewidth=0.04cm,arrowsize=0.05291667cm 2.0,arrowlength=1.4,arrowinset=0.4]{->}(1.0973437,-6.5135937)(23.577345,-6.5135937)
\usefont{T1}{ptm}{m}{n}
\rput(0.27453125,6.6764073){$y$}
\usefont{T1}{ptm}{m}{n}
\rput(22.734531,-7.123594){$x$}
\psdots[dotsize=0.12](8.487344,0.6164076)
\usefont{T1}{ptm}{m}{n}
\rput(0.58453125,-7.043594){$O$}
\usefont{T1}{ptm}{m}{n}
\rput(7.6045313,0.0364075){$O'$}
\psdots[dotsize=0.12](15.6,3.8064063)
\usefont{T1}{ptm}{m}{n}
\rput(16.424532,3.8964062){$C$}
\psline[linewidth=0.02cm](15.58,3.8064063)(3.36,6.5064063)
\psline[linewidth=0.02cm](8.48,0.6064063)(9.52,5.106406)
\psline[linewidth=0.02cm](8.5,0.62640625)(15.6,3.7864063)
\psarc[linewidth=0.02](14.5,3.6264062){0.42}{104.03625}{249.14554}
\usefont{T1}{ptm}{m}{n}
\rput(13.644531,3.6564062){$\alpha$}
\usefont{T1}{ptm}{m}{n}
\rput(9.834531,5.7964063){$T$}
\psline[linewidth=0.02cm](10.08,5.0264063)(9.94,4.4264064)
\psline[linewidth=0.02cm](9.94,4.4264064)(9.38,4.5664062)
\psdots[dotsize=0.06](9.72,4.746406)
\psline[linewidth=0.02cm,linestyle=dashed,dash=0.16cm 0.16cm](8.48,0.64640623)(18.2,0.64640623)
\psarc[linewidth=0.02](9.68,0.82640624){0.52}{341.56506}{54.29331}
\usefont{T1}{ptm}{m}{n}
\rput(10.824532,1.0164063){$\beta$}
\usefont{T1}{ptm}{m}{n}
\rput(8.344531,2.7564063){$R$}
\end{pspicture} 
}

}
  \caption{}
  \label{fig:circular_mpc_tan}
\end{figure}


In triangle $TO'C$, $\angle O'TC$ is right, and

\begin{align}
  \alpha = sin^{-1} \dfrac{R}{O'C}
\end{align}

while

\begin{align}
  \beta = tan^{-1} \dfrac{y_c - y_{O'}}{x_c - x_{O'}}
\end{align}

Point $T$ is then rudimentary to find: it is the point that lies on the
circle and is found while rotating by $\beta + \dfrac{\pi}{2} - \alpha$ from the
$x$ axis. The coordinates of point $T$ are

\begin{align}
  x_T &= x_{O'} + R cos\theta_T \\
  y_T &= y_{O'} + R sin\theta_T \\
  \theta_T &= \beta + \dfrac{\pi}{2} - \alpha \\
\end{align}

The same reasoning applies to when the vehicle is located at every other
quadrant. The pose that the vehicle ought to have at point $T$ of the circle
can now be retrieved from the given set of poses. Point $T$ shall serve as the
first reference for the vehicle in the statement of the optimization problem,
that is, $s_T = [x_T, y_T, \psi_T]^T = s_0^{ref}$. The next task is now to
find the next $N-1$ reference poses in the circle that the vehicle ought to
refer to.

\subsubsection{Beyond point $T$}

The next reference points depend on the velocity of the vehicle. Since its
velocity is constant, we know from elementary physics that $v = \omega R$,
which means that $\omega = v / R$, which is known, since the radius of the
circle is set by us in advance, and the velocity of the vehicle is measurable.
If we denote with $\theta_T$ the angle that the line passing through $O'$
and $T$ makes with the $x$ axis, with $\theta_1$ the angle that the line
passing through $O'$ and the first reference point after point $T$ makes with
the $x$ axis, and $T_s$ the sampling time of the pose of the vehicle, then

\begin{align}
  \omega &= \dfrac{\Delta \theta}{\Delta t} = \dfrac{\theta_1 - \theta_T}{T_s} \Leftrightarrow \\
  \theta_1 &= \theta_T + T_s \dfrac{v}{R} \\
\end{align}

In general, the $k$-th point in the sequence of the $N-1$ reference points will
lie at an angle of

\begin{align}
  \theta_k &= \theta_T + k T_s \dfrac{v}{R}\\
\end{align}

The coordinates of the $k$-th reference point ($1 \leq k \leq N-1$) are

\begin{align}
  x_k^{ref} &= x_{O'} + R cos\theta_k \\
  y_k^{ref} &= y_{O'} + R sin\theta_k \\
\end{align}

The pose that the vehicle ought to have at each point can be retrieved again
from the set of given poses.


\begin{figure}[H]\centering
  \scalebox{0.8}{% Generated with LaTeXDraw 2.0.8
% Mon Dec 05 01:54:20 CET 2016
% \usepackage[usenames,dvipsnames]{pstricks}
% \usepackage{epsfig}
% \usepackage{pst-grad} % For gradients
% \usepackage{pst-plot} % For axes
\scalebox{1} % Change this value to rescale the drawing.
{
\begin{pspicture}(0,-7.3264065)(23.597345,7.306406)
\pscircle[linewidth=0.04,dimen=outer](8.487344,0.6164076){4.63}
\psline[linewidth=0.04cm,arrowsize=0.05291667cm 2.0,arrowlength=1.4,arrowinset=0.4]{<-}(1.0973437,7.286406)(1.0973437,-6.5135937)
\psline[linewidth=0.04cm,arrowsize=0.05291667cm 2.0,arrowlength=1.4,arrowinset=0.4]{->}(1.0973437,-6.5135937)(23.577345,-6.5135937)
\usefont{T1}{ptm}{m}{n}
\rput(0.27453125,6.6764073){$y$}
\usefont{T1}{ptm}{m}{n}
\rput(22.734531,-7.123594){$x$}
\psdots[dotsize=0.12](8.487344,0.6164076)
\usefont{T1}{ptm}{m}{n}
\rput(0.58453125,-7.043594){$O$}
\usefont{T1}{ptm}{m}{n}
\rput(7.6045313,0.0364075){$O'$}
\psdots[dotsize=0.12](15.6,3.8064063)
\usefont{T1}{ptm}{m}{n}
\rput(16.424532,3.8964062){$C$}
\psline[linewidth=0.02cm,arrowsize=0.05291667cm 2.0,arrowlength=1.4,arrowinset=0.4]{->}(15.58,3.8064063)(4.6,6.226406)
\psline[linewidth=0.02cm](8.48,0.6064063)(9.52,5.106406)
\usefont{T1}{ptm}{m}{n}
\rput(9.834531,5.7964063){$T$}
\usefont{T1}{ptm}{m}{n}
\rput(9.564531,2.6764061){$R$}
\psline[linewidth=0.02cm](8.48,0.62640625)(8.72,5.226406)
\psline[linewidth=0.02cm](8.478426,0.61458015)(8.04,5.206406)
\psline[linewidth=0.02cm](8.476672,0.5947135)(6.34,4.706406)
\psdots[dotsize=0.12](7.4,5.106406)
\psdots[dotsize=0.12](8.04,5.186406)
\psdots[dotsize=0.12](8.72,5.226406)
\psdots[dotsize=0.12](9.52,5.146406)
\psline[linewidth=0.02cm,arrowsize=0.05291667cm 2.0,arrowlength=1.4,arrowinset=0.4]{->}(8.7,5.266406)(3.36,5.646406)
\psline[linewidth=0.02cm,arrowsize=0.05291667cm 2.0,arrowlength=1.4,arrowinset=0.4]{->}(8.038378,5.207967)(2.7016218,4.784846)
\psline[linewidth=0.02cm,arrowsize=0.05291667cm 2.0,arrowlength=1.4,arrowinset=0.4]{->}(6.3322954,4.6824946)(1.5896608,2.1990666)
\psline[linewidth=0.02cm](9.44,4.726406)(9.86,4.626406)
\psline[linewidth=0.02cm](9.86,4.626406)(9.96,5.0664062)
\psdots[dotsize=0.06](9.68,4.8664064)
\psline[linewidth=0.02cm,linestyle=dashed,dash=0.16cm 0.16cm](8.5,0.62640625)(17.64,0.62640625)
\psarc[linewidth=0.02](8.59,0.75640625){0.21}{320.7106}{92.12109}
\psarc[linewidth=0.02](9.0,1.0664062){0.66}{320.7106}{135.0}
\usefont{T1}{ptm}{m}{n}
\rput(9.184531,0.99640626){$\theta_T$}
\usefont{T1}{ptm}{m}{n}
\rput(10.034532,1.3564062){$\theta_1$}
\usefont{T1}{ptm}{m}{n}
\rput(7.4045315,3.9364061){$\dots$}
\psarc[linewidth=0.02](8.98,1.7064062){1.7}{320.7106}{151.69925}
\usefont{T1}{ptm}{m}{n}
\rput(11.504531,1.8564062){$\theta_{N-1}$}
\end{pspicture} 
}

}
  \caption{}
  \label{fig:circular_mpc_tan_and_refs}
\end{figure}




\subsubsection{Obtaining the linearized kinematic model}

The model constitutes the equations of motion of the vehicle, and has three
states ($x$, $y$, and $\psi$) and one input ($\delta$), since we assume a
constant velocity. The equations of the vehicle's motion that are relevant here
are

\begin{align}
  \dot{x} &= v cos(\psi + \beta) \\
  \dot{y} &= v sin(\psi + \beta) \\
  \dot{\psi} &= \dfrac{v}{l_r} sin\beta
\end{align}

Sampling with a sampling time of $T_s$ gives

\begin{align}
  x_{k+1} &= x_{k} + T_s v_k cos(\psi_k + \beta_k) \\
  y_{k+1} &= y_{k} + T_s v_k sin(\psi_k + \beta_k) \\
  \psi_{k+1} &= \psi_{k} + T_s \dfrac{v}{l_r} sin\beta_k
\end{align}

where

\begin{align}
  \beta_k = tan^{-1}\Big(\dfrac{l_r}{l_r + l_f} tan\delta_{k-1}\Big)
\end{align}


Forming the Jacobians for matrices $A$, $B$ and evaluating them at time
$t=k$ around the current state $\psi = \psi_k$, and
$\delta = \delta_{k-1}$ ($\delta_k$ is to be determined at time $k$):

\begin{equation}
 A =
  \begin{bmatrix}
    1 & 0 & -T_s v_k sin(\psi_k + \beta_k) \\\\
    0 & 1 & T_s v_k cos(\psi_k + \beta_k) \\\\
    0 & 0 & 1
  \end{bmatrix}, \text{or}
\end{equation}
\begin{equation}
  A =
  \begin{bmatrix}
    1 & 0 & -T_s v_k sin\Big(\psi_k + tan^{-1} (l_q tan\delta_{k-1})\Big) \\\\
    0 & 1 & T_s v_k cos\Big(\psi_k + tan^{-1} (l_q tan\delta_{k-1})\Big) \\\\
    0 & 0 & 1
  \end{bmatrix}
\end{equation}


\begin{equation}
 B =
  \begin{bmatrix}
    -T_s v_k sin(\psi_k + \beta_k) \dfrac{l_q}{l_q^2 sin^2\delta_{k-1} + cos^2\delta_{k-1}} \\
    T_s v_k cos(\psi_k + \beta_k) \dfrac{l_q}{l_q^2 sin^2\delta_{k-1} + cos^2\delta_{k-1}} \\
    \dfrac{T_s v_k}{l_r} cos(\beta_k) \dfrac{l_q}{l_q^2 sin^2\delta_{k-1} + cos^2\delta_{k-1}}
  \end{bmatrix}, \text{or}
\end{equation}
\begin{equation}
  B =
  \begin{bmatrix}
    -T_s v_k sin\Big(\psi_k + tan^{-1} (l_q tan\delta_{k-1})\Big) \dfrac{l_q}{l_q^2 sin^2\delta_{k-1} + cos^2\delta_{k-1}} \\
    T_s v_k cos\Big(\psi_k + tan^{-1} (l_q tan\delta_{k-1})\Big) \dfrac{l_q}{l_q^2 sin^2\delta_{k-1} + cos^2\delta_{k-1}} \\
    \dfrac{T_s v_k}{l_r} cos\Bigg(tan^{-1} \Big(l_q tan\delta_{k-1}\Big)\Bigg) \dfrac{l_q}{l_q^2 sin^2\delta_{k-1} + cos^2\delta_{k-1}}
  \end{bmatrix}
\end{equation}


where $l_q = \dfrac{l_r}{l_r + l_f}$

Now we can express the linear model as

\begin{align}
  s_{k+1} = A s_k + B u_k
\end{align}

where

\begin{equation}
  s_k=
  \begin{bmatrix}
    x_{k} \\
    y_{k} \\
    \psi_{k}
  \end{bmatrix}
\end{equation}

and

\begin{equation}
  u_k= \delta_{k}
\end{equation}


However, here there is another option. We can keep the A and B matrices
constant during the solution of the optimization problem, which means that
only one linearization is made (around the current state of the vehicle),
or we can solve the problem once with these matrices, extract the sequence of
predicted states, and linearize around them. This will result in $N-1$ new
linearizations around each predicted state of the vehicle, and hence $N-1$
additional $A_t$ matrices. In the same spirit, we linearize around the
corresponding optimal input $N-1$ times in total, and we obtain $N-1$
additional $B_t$ matrices.


\subsubsection{Stating the optimization problem}

We can now form the optimization problem to be solved at time $t$ as

\begin{align}
  \text{minimize }    & \sum\limits_{k=0}^{N-1} (s_k - s_k^{ref})^T Q (s_k - s_k^{ref}) + u_k^T R u_k \\
  \text{subject to }  & s_{k+1} = A_k s_k + B_k u_k, \text{ where } s_k = [x_k, y_k, \psi_k]^T, u_k = \delta_k \\
                      & \delta^{min} \leq \delta_k \leq \delta^{max} \\
                      & s_k^{ref} = (x_k^{ref}, y_k^{ref}, \psi_k^{ref}) \\
                      & s_0 = (x_t, y_t, \psi_t)
\end{align}

where $A_k$ or $B_k$ are the Jacobians obtained via linearization around
the state of the vehicle at time $t$, in which case they are constant across all
$k$'s, ($A_k = A$, $B_k = B$, $\forall k$) and the problem is solved only
once per sampling time; or they are obtained by linearizing around the
predicted state of the vehicle, in which case the problem is solved once
with $A_k = A$, $B_k = B$, and one final time with the variant-time A and B
matrices.
