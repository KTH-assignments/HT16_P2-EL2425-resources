In figure \ref{fig:centerline_mpc}, the $x$ axis is fixed on the lane's
centerline, while axis $y$ is perpendicular to it. The origin is at point
$O$. The vehicle is represented by point $C$. The orientation of the vehicle
with respect to the lane (the $x$-axis) is $\phi$. Given this configuration
and three scans, at $-90^{\circ}$, $0^{\circ}$ and $90^{\circ}$ with
respect to the longitudinal axis of the vehicle, denoted by $CR$, $CF$ and
$CL$ respectively, the objective is to find the distance $OC$ and the angle
$\phi$ so that a MPC optimization problem can be solved with $OC$ and $\phi$
acting as initial conditions. The velocity of the vehicle,
which is constant, and its displacement along the $x$-axis are at this point
irrelevant to the optimization problem given this configuration. The only
source of information is the lidar itself.

\begin{figure}[H]\centering
  \scalebox{1}{% Generated with LaTeXDraw 2.0.8
% Thu Nov 17 00:32:19 CET 2016
% \usepackage[usenames,dvipsnames]{pstricks}
% \usepackage{epsfig}
% \usepackage{pst-grad} % For gradients
% \usepackage{pst-plot} % For axes
\scalebox{1} % Change this value to rescale the drawing.
{
\begin{pspicture}(0,-3.5725)(13.853594,3.5525)
\psline[linewidth=0.04cm](0.0,2.1925)(13.34,2.1925)
\psline[linewidth=0.04cm](0.0,-3.0075)(13.34,-3.0075)
\psline[linewidth=0.03cm,linestyle=dashed,dash=0.16cm 0.16cm](0.0,-0.4075)(13.34,-0.4075)
\psline[linewidth=0.04cm,arrowsize=0.05291667cm 2.0,arrowlength=1.4,arrowinset=0.4]{->}(4.94,-0.4075)(4.94,3.5325)
\psline[linewidth=0.04cm,arrowsize=0.05291667cm 2.0,arrowlength=1.4,arrowinset=0.4]{->}(4.94,-0.4075)(13.76,-0.4075)
\usefont{T1}{ptm}{m}{n}
\rput(13.479062,-0.8775){$x$}
\usefont{T1}{ptm}{m}{n}
\rput(4.3790627,3.1625){$y$}
\psline[linewidth=0.04cm,linecolor=red](4.94,-1.8275)(10.54,2.1925)
\psline[linewidth=0.04cm,linecolor=red](5.72,-3.0075)(2.36,2.1925)
\usefont{T1}{ptm}{m}{n}
\rput(2.2690625,2.5225){$L$}
\usefont{T1}{ptm}{m}{n}
\rput(6.0590625,-3.3175){$R$}
\usefont{T1}{ptm}{m}{n}
\rput(10.7890625,2.6625){$F$}
\psarc[linewidth=0.02](7.41,-0.2575){0.17}{304.69516}{100.00798}
\psarc[linewidth=0.02](10.13,2.0625){0.17}{127.092834}{263.65982}
\psline[linewidth=0.02cm](4.94,-0.4275)(4.94,-3.0075)
\usefont{T1}{ptm}{m}{n}
\rput(5.6190624,-1.8775){$C$}
\usefont{T1}{ptm}{m}{n}
\rput(5.4890623,-0.0375){$O$}
\usefont{T1}{ptm}{m}{n}
\rput(5.4490623,2.4625){$L_o$}
\usefont{T1}{ptm}{m}{n}
\rput(4.6390624,-3.3375){$R_o$}
\usefont{T1}{ptm}{m}{n}
\rput(9.399062,1.8825){$\phi$}
\usefont{T1}{ptm}{m}{n}
\rput(8.019062,-0.1375){$\phi$}
\psline[linewidth=0.04cm](5.16,-1.6875)(5.3,-1.9075)
\psline[linewidth=0.04cm](5.28,-1.8875)(5.1,-2.0275)
\psdots[dotsize=0.06](5.1,-1.8475)
\end{pspicture} 
}

}
  \caption{}
  \label{fig:centerline_mpc}
\end{figure}


\textbf{Initial conditions}

\textbf{Finding $\phi$}

Here we can distinguish two cases: one where the vehicle is facing the right
lane boundary and one where it is facing the left one. It is not obvious
in this configuration where the vehicle is heading: the only available
information so far is only the three range scans.

In the first case, the heading angle error is

\begin{align}
  \phi = \text{tan}^{-1}\dfrac{CR}{CF}
\end{align}

\begin{figure}[H]\centering
  \scalebox{1}{% Generated with LaTeXDraw 2.0.8
% Thu Nov 17 19:28:33 CET 2016
% \usepackage[usenames,dvipsnames]{pstricks}
% \usepackage{epsfig}
% \usepackage{pst-grad} % For gradients
% \usepackage{pst-plot} % For axes
\scalebox{1} % Change this value to rescale the drawing.
{
\begin{pspicture}(0,-3.6364062)(13.802655,3.6164062)
\definecolor{color8}{rgb}{0.00392156862745098,0.00392156862745098,0.00392156862745098}
\psline[linewidth=0.04cm](0.0,2.2564063)(13.34,2.2564063)
\psline[linewidth=0.04cm](0.0,-2.9435937)(13.34,-2.9435937)
\psline[linewidth=0.03cm,linestyle=dashed,dash=0.16cm 0.16cm](0.0,-0.34359375)(13.34,-0.34359375)
\psline[linewidth=0.04cm,arrowsize=0.05291667cm 2.0,arrowlength=1.4,arrowinset=0.4]{->}(4.94,-0.34359375)(4.94,3.5964062)
\psline[linewidth=0.04cm,arrowsize=0.05291667cm 2.0,arrowlength=1.4,arrowinset=0.4]{->}(4.94,-0.34359375)(13.76,-0.34359375)
\usefont{T1}{ptm}{m}{n}
\rput(13.428124,-0.8135938){$x$}
\usefont{T1}{ptm}{m}{n}
\rput(4.3281255,3.226406){$y$}
\psline[linewidth=0.04cm,linecolor=color8](4.912057,-1.8034165)(8.7,-2.9435937)
\psline[linewidth=0.04cm,linecolor=color8](4.52,-2.9635937)(6.36,2.2364063)
\usefont{T1}{ptm}{m}{n}
\rput(6.258125,2.5064063){$L$}
\usefont{T1}{ptm}{m}{n}
\rput(4.328125,-3.4335938){$R$}
\usefont{T1}{ptm}{m}{n}
\rput(8.698125,-3.2735937){$F$}
\psarc[linewidth=0.02](7.81,-2.7735937){0.17}{127.092834}{263.65982}
\psline[linewidth=0.02cm](4.94,-0.36359376)(4.94,-2.9435937)
\usefont{T1}{ptm}{m}{n}
\rput(4.008125,-1.8135937){$C$}
\usefont{T1}{ptm}{m}{n}
\rput(4.3981247,-0.03359375){$O$}
\usefont{T1}{ptm}{m}{n}
\rput(7.308124,-2.7535937){$\phi$}
\psline[linewidth=0.04cm](5.0030665,-1.5652002)(5.2494287,-1.6506729)
\psline[linewidth=0.04cm](5.221332,-1.6474196)(5.167209,-1.8689386)
\psdots[dotsize=0.06](5.08,-1.7235938)
\end{pspicture} 
}

}
  \caption{The vehicle's heading is towards the right lane boundary.}
  \label{}
\end{figure}

In the second case, the heading angle error is

\begin{align}
  \phi = \text{tan}^{-1}\dfrac{CL}{CF}
\end{align}

\begin{figure}[H]\centering
  \scalebox{1}{% Generated with LaTeXDraw 2.0.8
% Thu Nov 17 19:02:56 CET 2016
% \usepackage[usenames,dvipsnames]{pstricks}
% \usepackage{epsfig}
% \usepackage{pst-grad} % For gradients
% \usepackage{pst-plot} % For axes
\scalebox{1} % Change this value to rescale the drawing.
{
\begin{pspicture}(0,-3.5464063)(13.802655,3.5264063)
\definecolor{color8}{rgb}{0.00392156862745098,0.00392156862745098,0.00392156862745098}
\psline[linewidth=0.04cm](0.0,2.1664062)(13.34,2.1664062)
\psline[linewidth=0.04cm](0.0,-3.0335937)(13.34,-3.0335937)
\psline[linewidth=0.03cm,linestyle=dashed,dash=0.16cm 0.16cm](0.0,-0.43359375)(13.34,-0.43359375)
\psline[linewidth=0.04cm,arrowsize=0.05291667cm 2.0,arrowlength=1.4,arrowinset=0.4]{->}(4.94,-0.43359375)(4.94,3.5064063)
\psline[linewidth=0.04cm,arrowsize=0.05291667cm 2.0,arrowlength=1.4,arrowinset=0.4]{->}(4.94,-0.43359375)(13.76,-0.43359375)
\usefont{T1}{ptm}{m}{n}
\rput(13.428124,-0.9035938){$x$}
\usefont{T1}{ptm}{m}{n}
\rput(4.3281255,3.1364062){$y$}
\psline[linewidth=0.04cm,linecolor=color8](4.94,-1.8535937)(10.54,2.1664062)
\psline[linewidth=0.04cm,linecolor=color8](5.72,-3.0335937)(2.36,2.1664062)
\usefont{T1}{ptm}{m}{n}
\rput(2.2181249,2.4964063){$L$}
\usefont{T1}{ptm}{m}{n}
\rput(6.008125,-3.3435938){$R$}
\usefont{T1}{ptm}{m}{n}
\rput(10.738125,2.4964063){$F$}
\psarc[linewidth=0.02](7.41,-0.28359374){0.17}{304.69516}{100.00798}
\psarc[linewidth=0.02](10.13,2.0364063){0.17}{127.092834}{263.65982}
\psline[linewidth=0.02cm](4.94,-0.45359376)(4.94,-3.0335937)
\usefont{T1}{ptm}{m}{n}
\rput(5.568125,-1.9035938){$C$}
\usefont{T1}{ptm}{m}{n}
\rput(5.4381247,-0.06359375){$O$}
\usefont{T1}{ptm}{m}{n}
\rput(9.3481245,1.8564062){$\phi$}
\usefont{T1}{ptm}{m}{n}
\rput(7.9681244,-0.16359375){$\phi$}
\psline[linewidth=0.04cm](5.16,-1.7135937)(5.3,-1.9335938)
\psline[linewidth=0.04cm](5.28,-1.9135938)(5.1,-2.0535936)
\psdots[dotsize=0.06](5.1,-1.8735938)
\end{pspicture} 
}

}
  \caption{The vehicle's heading is towards the left lane boundary.}
  \label{}
\end{figure}

In order to deduce the correct value of $\phi$ ($-\tan^{-1}\dfrac{CR}{CF}$ or
$\tan^{-1}\dfrac{CL}{CF}$) further ranges scans are needed. To this end,
a difference between ranges around point $F$ is taken: starting at the
right of $F$ and moving anti-clockwise, we calculate the difference between
two range scans for a given angle between them. If its sign
is negative then the vehicle is facing the right lane boundary; if not,
it is facing the left lane boundary. This concept is depicted in figures
\ref{fig:range_mpc_diff_negative} and  \ref{fig:range_mpc_diff_positive}.

\begin{figure}[H]\centering
  \scalebox{1}{% Generated with LaTeXDraw 2.0.8
% Thu Nov 17 19:30:07 CET 2016
% \usepackage[usenames,dvipsnames]{pstricks}
% \usepackage{epsfig}
% \usepackage{pst-grad} % For gradients
% \usepackage{pst-plot} % For axes
\scalebox{1} % Change this value to rescale the drawing.
{
\begin{pspicture}(0,-3.6364062)(13.78,3.6164062)
\definecolor{color8}{rgb}{0.00392156862745098,0.00392156862745098,0.00392156862745098}
\psline[linewidth=0.04cm](0.0,2.2564063)(13.34,2.2564063)
\psline[linewidth=0.04cm](0.0,-2.9435937)(13.34,-2.9435937)
\psline[linewidth=0.03cm,linestyle=dashed,dash=0.16cm 0.16cm](0.0,-0.3435937)(13.34,-0.3435937)
\psline[linewidth=0.04cm,arrowsize=0.05291667cm 2.0,arrowlength=1.4,arrowinset=0.4]{->}(4.94,-0.3435937)(4.94,3.5964062)
\psline[linewidth=0.04cm,arrowsize=0.05291667cm 2.0,arrowlength=1.4,arrowinset=0.4]{->}(4.94,-0.3435937)(13.76,-0.3435937)
\usefont{T1}{ptm}{m}{n}
\rput(13.402656,-0.81359375){$x$}
\usefont{T1}{ptm}{m}{n}
\rput(4.3026567,3.226406){$y$}
\psline[linewidth=0.04cm,linecolor=color8](4.912057,-1.8034165)(8.7,-2.9435937)
\psline[linewidth=0.04cm,linecolor=color8](4.52,-2.9635937)(6.36,2.2364063)
\usefont{T1}{ptm}{m}{n}
\rput(6.3926563,2.5464063){$L$}
\usefont{T1}{ptm}{m}{n}
\rput(4.302656,-3.4335938){$R$}
\usefont{T1}{ptm}{m}{n}
\rput(8.672656,-3.2735937){$F$}
\psarc[linewidth=0.02](7.81,-2.7735937){0.17}{127.092834}{263.65982}
\psline[linewidth=0.02cm](4.94,-0.3635937)(4.94,-2.9435937)
\usefont{T1}{ptm}{m}{n}
\rput(3.9826562,-1.8135936){$C$}
\usefont{T1}{ptm}{m}{n}
\rput(4.372656,-0.0335937){$O$}
\usefont{T1}{ptm}{m}{n}
\rput(7.2826552,-2.7535937){$\phi$}
\psline[linewidth=0.04cm](5.0030665,-1.5652001)(5.2494287,-1.6506728)
\psline[linewidth=0.04cm](5.221332,-1.6474196)(5.167209,-1.8689386)
\psdots[dotsize=0.06](5.08,-1.7235937)
\psline[linewidth=0.04cm,linecolor=red](4.9438024,-1.7790809)(12.56,-2.9435937)
\psline[linewidth=0.04cm,linecolor=red](4.9200974,-1.7871213)(7.28,-2.9435937)
\usefont{T1}{ptm}{m}{n}
\rput(12.532657,-3.2535937){$F_1$}
\usefont{T1}{ptm}{m}{n}
\rput(7.1726556,-3.3335936){$F_0$}
\rput{-52.526722}(4.0356064,4.3793244){\psarc[linewidth=0.04,linecolor=red,arrowsize=0.05291667cm 2.0,arrowlength=1.4,arrowinset=0.4]{->}(6.4553895,-1.8996336){0.74}{334.53665}{35.493298}}
\end{pspicture} 
}

}
  \caption{$CF_0 < CF < CF_1$, hence $CF_0 - CF_1 < 0$ and $\phi = -tan^{-1} \dfrac{CR}{CF}$}
  \label{fig:range_mpc_diff_negative}
\end{figure}

\begin{figure}[H]\centering
  \scalebox{1}{% Generated with LaTeXDraw 2.0.8
% Thu Nov 17 18:47:37 CET 2016
% \usepackage[usenames,dvipsnames]{pstricks}
% \usepackage{epsfig}
% \usepackage{pst-grad} % For gradients
% \usepackage{pst-plot} % For axes
\scalebox{1} % Change this value to rescale the drawing.
{
\begin{pspicture}(0,-3.5464063)(13.828124,3.5264063)
\definecolor{color8}{rgb}{0.00392156862745098,0.00392156862745098,0.00392156862745098}
\psline[linewidth=0.04cm](0.0,2.1664062)(13.34,2.1664062)
\psline[linewidth=0.04cm](0.0,-3.0335937)(13.34,-3.0335937)
\psline[linewidth=0.03cm,linestyle=dashed,dash=0.16cm 0.16cm](0.0,-0.43359375)(13.34,-0.43359375)
\psline[linewidth=0.04cm,arrowsize=0.05291667cm 2.0,arrowlength=1.4,arrowinset=0.4]{->}(4.94,-0.43359375)(4.94,3.5064063)
\psline[linewidth=0.04cm,arrowsize=0.05291667cm 2.0,arrowlength=1.4,arrowinset=0.4]{->}(4.94,-0.43359375)(13.76,-0.43359375)
\usefont{T1}{ptm}{m}{n}
\rput(13.453593,-0.9035938){$x$}
\usefont{T1}{ptm}{m}{n}
\rput(4.353594,3.1364062){$y$}
\psline[linewidth=0.04cm,linecolor=color8](4.94,-1.8535937)(10.54,2.1664062)
\psline[linewidth=0.04cm,linecolor=color8](5.72,-3.0335937)(2.36,2.1664062)
\usefont{T1}{ptm}{m}{n}
\rput(2.2435937,2.4964063){$L$}
\usefont{T1}{ptm}{m}{n}
\rput(6.0335937,-3.3435938){$R$}
\usefont{T1}{ptm}{m}{n}
\rput(10.763594,2.4964063){$F$}
\psarc[linewidth=0.02](7.41,-0.28359374){0.17}{304.69516}{100.00798}
\psarc[linewidth=0.02](10.13,2.0364063){0.17}{127.092834}{263.65982}
\psline[linewidth=0.02cm](4.94,-0.45359376)(4.94,-3.0335937)
\usefont{T1}{ptm}{m}{n}
\rput(5.5935936,-1.9035938){$C$}
\usefont{T1}{ptm}{m}{n}
\rput(5.4635935,-0.06359375){$O$}
\usefont{T1}{ptm}{m}{n}
\rput(9.373593,1.8564062){$\phi$}
\usefont{T1}{ptm}{m}{n}
\rput(7.993593,-0.16359375){$\phi$}
\psline[linewidth=0.04cm](5.16,-1.7135937)(5.3,-1.9335938)
\psline[linewidth=0.04cm](5.28,-1.9135938)(5.1,-2.0535936)
\psdots[dotsize=0.06](5.1,-1.8735938)
\psline[linewidth=0.04cm,linecolor=red](4.94,-1.8135937)(8.66,2.1664062)
\psline[linewidth=0.04cm,linecolor=red](4.96,-1.8335937)(12.52,2.1464062)
\usefont{T1}{ptm}{m}{n}
\rput(8.783594,2.4964063){$F_1$}
\usefont{T1}{ptm}{m}{n}
\rput(12.623593,2.4964063){$F_0$}
\psarc[linewidth=0.04,linecolor=red,arrowsize=0.05291667cm 2.0,arrowlength=1.4,arrowinset=0.4]{->}(8.08,0.50640625){0.74}{334.53665}{113.19859}
\end{pspicture} 
}

}
  \caption{$CF_0 > CF > CF_1$, hence $CF_0 - CF_1 > 0$ and $\phi = tan^{-1} \dfrac{CL}{CF}$}
  \label{fig:range_mpc_diff_positive}
\end{figure}


\textbf{Finding $OC$}

First we note that $OC$ does not depend on the orientation of the vehicle. Then

\begin{align}
  L_o O + OC + CR_o &= W = CR_o + CL_o
\end{align}

where $W$ is the width of the lane. But $L_o O = \dfrac{W}{2}$ hence

\begin{align}
  OC + CR_o &= \dfrac{1}{2}(CR_o + CL_o) \Leftrightarrow \\
  OC &= \dfrac{1}{2}(CL_o - CR_o)
\end{align}

But $CL_o = CL sin\lambda$ and $CR_o = CR cos\phi$, hence


\begin{align}
  OC &= \dfrac{1}{2}(CL sin\lambda - CR cos\phi)
\end{align}

From triangle LCF in the case where the vehicle is facing the left lane boundary,
we note that $\phi + \lambda + \dfrac{\pi}{2} = \pi$,
hence $\lambda = \dfrac{\pi}{2} - \phi$. Then, we conclude that

\begin{align}
  OC &= \dfrac{1}{2}(CL - CR) cos\phi
\end{align}




\textbf{Obtaining the relevant linearized kinematic model}

The model constitutes the equations of motion of the vehicle, and has two
states ($y$ and $\phi$) and one input ($\delta$). The equations of the
vehicle's motion that are relevant here are

\begin{align}
  \dot{y} &= v sin(\phi + \beta) \\
  \dot{\phi} &= \dfrac{v}{l_r} sin\beta
\end{align}

Sampling with a sampling time of $T_s$ gives

\begin{align}
  y_{k+1} &= y_{k} + T_s v sin(\phi_k + \beta_k) \\
  \phi_{k+1} &= \phi_{k} + T_s \dfrac{v}{l_r} sin\beta_k
\end{align}

where

\begin{align}
  \beta_k = tan^{-1}\Big(\dfrac{l_r}{l_r + l_f} tan\delta_k\Big)
\end{align}


Forming the Jacobians for matrices $A$, $B$ and evaluating them at time
$t=k$ around $\delta = 0$ (which makes $\beta = 0$) gives

\begin{equation}
 A =
  \begin{bmatrix}
    1 & T_s v cos(\phi_k + \beta_k) \\\\
    0 & 1
  \end{bmatrix}
  =
  \begin{bmatrix}
    1 & T_s v cos\Big(\phi_k + tan^{-1} (l_q tan\delta_k)\Big) \\\\
    0 & 1
  \end{bmatrix}
\end{equation}
\begin{equation}
 A =
  \begin{bmatrix}
    1 & T_s v cos\phi_k \\\\
    0 & 1
  \end{bmatrix}
\end{equation}

\begin{equation}
 B =
  \begin{bmatrix}
    T_s v cos(\phi_k + \beta_k) \dfrac{l_q}{l_q^2 sin^2\delta_k + cos^2\delta_k} \\
    \dfrac{T_s v}{l_r} cos(\beta_k) \dfrac{l_q}{l_q^2 sin^2\delta_k + cos^2\delta_k}
  \end{bmatrix}
  =
  \begin{bmatrix}
    T_s v cos\Big(\phi_k + tan^{-1} (l_q tan\delta_k)\Big) \dfrac{l_q}{l_q^2 sin^2\delta_k + cos^2\delta_k} \\
    \dfrac{T_s v}{l_r} cos\Bigg(tan^{-1} \Big(l_q tan\delta_k\Big)\Bigg) \dfrac{l_q}{l_q^2 sin^2\delta_k + cos^2\delta_k}
  \end{bmatrix}
\end{equation}
\begin{equation}
 B =
  \begin{bmatrix}
    \dfrac{T_s l_r v}{l_r + l_f} cos\phi_k \\\\
    \dfrac{T_s v}{l_r+l_f}
  \end{bmatrix}
\end{equation}

where $l_q = \dfrac{l_r}{l_r + l_f}$




Now we can express the linear model as

\begin{align}
  s_{k+1} = A s_k + B \delta_k
\end{align}

where

\begin{equation}
  s=
  \begin{bmatrix}
    y_{k} \\
    \phi_{k}
  \end{bmatrix}
\end{equation}

or

\begin{equation}
  \begin{bmatrix}
    y_{k+1} \\
    \phi_{k+1}
  \end{bmatrix}
  =
  \begin{bmatrix}
    1 & T_s v cos\phi_k \\\\
    0 & 1
  \end{bmatrix}
  \begin{bmatrix}
    y_{k} \\
    \phi_{k}
  \end{bmatrix}
  +
  \begin{bmatrix}
    \dfrac{T_s l_r v}{l_r + l_f} cos\phi_k \\\\
    \dfrac{T_s v}{l_r+l_f}
  \end{bmatrix}
  \delta_{k}
\end{equation}

\textbf{Stating the optimization problem}

We can now form the optimization problem as

\begin{align}
  \text{minimize }    & \sum\limits_{k=0}^N y_k^2 q_y + \phi_k^2 q_{\phi} + \delta_k^2 r \\
  \text{subject to }  & s_{k+1} = A s_k + B \delta_k,\text{ where } s_k = [y_k, \phi_k]^T \\
                      & \delta_{min} \leq \delta_k \leq \delta_{max} \\
                      & y_0 = -\dfrac{1}{2}(CL-CR)cos\phi_0 \\
                      & \phi_0 = tan^{-1}\dfrac{CL}{CF} \text{ or } \phi_0 = -tan^{-1}\dfrac{CR}{CF}
\end{align}
