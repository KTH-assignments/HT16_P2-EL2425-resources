\documentclass[oneside,12pt]{article}
\usepackage{fontspec}
\usepackage{lmodern}
%\setmainfont{cmr12}
\defaultfontfeatures{Ligatures=TeX} % To support LaTeX quoting style
\usepackage{amsmath}
\usepackage{lscape}
\usepackage{graphicx}
\usepackage{pgfplots}
\usepackage{subcaption}
\usepackage[margin=1in]{geometry}
\usepackage{float}
\usepackage{url}
\usepackage[hidelinks]{hyperref}
\usepackage{pstricks}
\usepackage{steinmetz}

\widowpenalty=1000
\clubpenalty=1000


\title{EL2425 - Slip Control \\ Meeting agenda 2016-11-16}
%\date{}

\begin{document}
\maketitle

\section{Done}

%\begin{itemize}
%\end{itemize}



\section{Ongoing}

  \subsection{Theoretical solution involving MPC and tracking the centerline of a lane}


    In figure \ref{fig:centerline_mpc}, the $x$ axis is fixed on the lane's
    centerline, while axis $y$ is perpendicular to it. The origin is at point
    $O$. The vehicle is represented by point $C$. The orientation of the vehicle
    with respect to the lane (the $x$-axis) is $\phi$. Given this configuration
    and three scans, at $-90^{\circ}$, $0^{\circ}$ and $90^{\circ}$ with
    respect to the longitudinal axis of the vehicle, denoted by $CR$, $CF$ and
    $CL$ respectively, the objective is to find the distance $OC$ and the angle
    $\phi$ so that a MPC optimization problem can be solved with $OC$ and $\phi$
    acting as initial conditions. The velocity of the vehicle,
    which is constant, and its displacement along the $x$-axis are at this point
    irrelevant to the optimization problem given this configuration. The only
    source of information is the lidar itself and nothing else.

    \begin{figure}[H]\centering
      \scalebox{1}{% Generated with LaTeXDraw 2.0.8
% Thu Nov 17 00:32:19 CET 2016
% \usepackage[usenames,dvipsnames]{pstricks}
% \usepackage{epsfig}
% \usepackage{pst-grad} % For gradients
% \usepackage{pst-plot} % For axes
\scalebox{1} % Change this value to rescale the drawing.
{
\begin{pspicture}(0,-3.5725)(13.853594,3.5525)
\psline[linewidth=0.04cm](0.0,2.1925)(13.34,2.1925)
\psline[linewidth=0.04cm](0.0,-3.0075)(13.34,-3.0075)
\psline[linewidth=0.03cm,linestyle=dashed,dash=0.16cm 0.16cm](0.0,-0.4075)(13.34,-0.4075)
\psline[linewidth=0.04cm,arrowsize=0.05291667cm 2.0,arrowlength=1.4,arrowinset=0.4]{->}(4.94,-0.4075)(4.94,3.5325)
\psline[linewidth=0.04cm,arrowsize=0.05291667cm 2.0,arrowlength=1.4,arrowinset=0.4]{->}(4.94,-0.4075)(13.76,-0.4075)
\usefont{T1}{ptm}{m}{n}
\rput(13.479062,-0.8775){$x$}
\usefont{T1}{ptm}{m}{n}
\rput(4.3790627,3.1625){$y$}
\psline[linewidth=0.04cm,linecolor=red](4.94,-1.8275)(10.54,2.1925)
\psline[linewidth=0.04cm,linecolor=red](5.72,-3.0075)(2.36,2.1925)
\usefont{T1}{ptm}{m}{n}
\rput(2.2690625,2.5225){$L$}
\usefont{T1}{ptm}{m}{n}
\rput(6.0590625,-3.3175){$R$}
\usefont{T1}{ptm}{m}{n}
\rput(10.7890625,2.6625){$F$}
\psarc[linewidth=0.02](7.41,-0.2575){0.17}{304.69516}{100.00798}
\psarc[linewidth=0.02](10.13,2.0625){0.17}{127.092834}{263.65982}
\psline[linewidth=0.02cm](4.94,-0.4275)(4.94,-3.0075)
\usefont{T1}{ptm}{m}{n}
\rput(5.6190624,-1.8775){$C$}
\usefont{T1}{ptm}{m}{n}
\rput(5.4890623,-0.0375){$O$}
\usefont{T1}{ptm}{m}{n}
\rput(5.4490623,2.4625){$L_o$}
\usefont{T1}{ptm}{m}{n}
\rput(4.6390624,-3.3375){$R_o$}
\usefont{T1}{ptm}{m}{n}
\rput(9.399062,1.8825){$\phi$}
\usefont{T1}{ptm}{m}{n}
\rput(8.019062,-0.1375){$\phi$}
\psline[linewidth=0.04cm](5.16,-1.6875)(5.3,-1.9075)
\psline[linewidth=0.04cm](5.28,-1.8875)(5.1,-2.0275)
\psdots[dotsize=0.06](5.1,-1.8475)
\end{pspicture} 
}

}
      \caption{}
      \label{fig:centerline_mpc}
    \end{figure}


    \subsubsection{Initial conditions}

    \textbf{Finding $\phi$}

    We turn our attention to triangle LCF, where

    \begin{align}
      tan\phi &= \dfrac{CL}{CF} \Leftrightarrow \phi = tan^{-1} \dfrac{CL}{CF} \\
    \end{align}


    \textbf{Finding $OC$}

    First we note that

    \begin{align}
      L_o O + OC + CR_o &= W = CR_o + CL_o
    \end{align}

    where $W$ is the width of the lane. But $L_o O = \dfrac{W}{2}$ hence

    \begin{align}
      OC + CR_o &= \dfrac{1}{2}(CR_o + CL_o) \Leftrightarrow \\
      OC &= \dfrac{1}{2}(CL_o - CR_o)
    \end{align}

    But $CL_o = CL sin\lambda$ and $CR_o = CR cos\phi$, hence


    \begin{align}
      OC &= \dfrac{1}{2}(CL sin\lambda - CR cos\phi)
    \end{align}

    From triangle LCF we note that $\phi + \lambda + \dfrac{\pi}{2} = \pi$,
    hence $\lambda = \dfrac{\pi}{2} - \phi$. Then, we conclude that

    \begin{align}
      OC &= \dfrac{1}{2}(CL - CR) cos\phi
    \end{align}




    \subsubsection{Obtaining the relevant linearized kinematic model}

    The equations of the vehicle's motion that are relevant are

    \begin{align}
      \dot{y} &= v sin(\phi + \beta) \\
      \dot{\phi} &= \dfrac{v}{l_r} sin\beta
    \end{align}

    Sampling with a sampling time of $T_s$ gives

    \begin{align}
      y[k+1] &= y[k] + T_s v sin(\phi + \beta) \\
      \phi[k+1] &= \phi[k] + T_s \dfrac{v}{l_r} sin\beta
    \end{align}

    Forming the Jacobians for matrices $A$, $B$ and evaluating them at time
    $t=k$ around $\delta = 0$ (which makes $\beta = 0$) gives

    \begin{equation}
     A =
      \begin{bmatrix}
        1 & T_s v cos(\phi + \beta) \\\\
        0 & 1
      \end{bmatrix}
      =
      \begin{bmatrix}
        1 & T_s v cos\Big(tan^{-1}\dfrac{CL_k}{CF_k} + tan^{-1} (l_q tan\delta_k)\Big) \\
        0 & 1
      \end{bmatrix}
    \end{equation}
    \begin{equation}
     A =
      \begin{bmatrix}
        1 & T_s v cos\Big(tan^{-1}\dfrac{CL_k}{CF_k}\Big) \\\\
        0 & 1
      \end{bmatrix}
    \end{equation}

    \begin{equation}
     B =
      \begin{bmatrix}
        T_s v cos(\phi + \beta) \dfrac{l_q}{l_q^2 sin^2\delta_k + cos^2\delta_k} \\
        \dfrac{T_s v}{l_r} cos(\beta) \dfrac{l_q}{l_q^2 sin^2\delta_k + cos^2\delta_k}
      \end{bmatrix}
      =
      \begin{bmatrix}
        T_s v cos\Big(tan^{-1}\dfrac{CL_k}{CF_k} + tan^{-1} (l_q tan\delta_k)\Big) \dfrac{l_q}{l_q^2 sin^2\delta_k + cos^2\delta_k} \\
        \dfrac{T_s v}{l_r} cos\Bigg(tan^{-1} \Big(l_q tan\delta_k\Big)\Bigg) \dfrac{l_q}{l_q^2 sin^2\delta_k + cos^2\delta_k}
      \end{bmatrix}
    \end{equation}
    \begin{equation}
     B =
      \begin{bmatrix}
        \dfrac{T_s l_r v}{l_r + l_f} cos\Big(tan^{-1}\dfrac{CL_k}{CF_k}\Big) \\\\
        \dfrac{T_s v}{l_r+l_f}
      \end{bmatrix}
    \end{equation}

    where $l_q = \dfrac{l_r}{l_r + l_f}$




  Now we can express the linear model as

    \begin{equation}
      \begin{bmatrix}
        y_{k+1} \\
        \phi_{k+1}
      \end{bmatrix}
      =
       A
      \begin{bmatrix}
        y_{k} \\
        \phi_{k}
      \end{bmatrix}
      +
      B
      \delta_{k}\text{, or}
    \end{equation}

    \begin{equation}
      \begin{bmatrix}
        y_{k+1} \\
        \phi_{k+1}
      \end{bmatrix}
      =
      \begin{bmatrix}
        1 & T_s v cos\Big(tan^{-1}\dfrac{CL_k}{CF_k}\Big) \\\\
        0 & 1
      \end{bmatrix}
      \begin{bmatrix}
        y_{k} \\
        \phi_{k}
      \end{bmatrix}
      +
      \begin{bmatrix}
        \dfrac{T_s l_r v}{l_r + l_f} cos\Big(tan^{-1}\dfrac{CL_k}{CF_k}\Big) \\\\
        \dfrac{T_s v}{l_r+l_f}
      \end{bmatrix}
      \delta_{k}
    \end{equation}



\section{Issues}

%\begin{itemize}
%\end{itemize}


\section{To do}

%\begin{itemize}
%\end{itemize}

\section{Misc.}

The progress of the project can be observed in \texttt{trello} and \texttt{github}:

\begin{itemize}
  \item \url{https://trello.com/b/uEP0jl0B/slip-control}
  \item \url{https://gits-15.sys.kth.se/alefil/HT16_P2_EL2425}
  \item \url{https://gits-15.sys.kth.se/alefil/HT16_P2_EL2425_resources}
\end{itemize}


\end{document}
