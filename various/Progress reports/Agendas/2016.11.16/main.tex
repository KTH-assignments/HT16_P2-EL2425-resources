\documentclass[oneside,12pt]{article}
\usepackage{fontspec}
\usepackage{lmodern}
%\setmainfont{cmr12}
\defaultfontfeatures{Ligatures=TeX} % To support LaTeX quoting style
\usepackage{amsmath}
\usepackage{lscape}
\usepackage{graphicx}
\usepackage{pgfplots}
\usepackage{subcaption}
\usepackage[margin=1in]{geometry}
\usepackage{float}
\usepackage{url}

\widowpenalty=1000
\clubpenalty=1000


\title{EL2425 - Slip Control - Meeting agenda 2016-11-16}
%\date{}

\begin{document}
\maketitle

\section{Done}

\begin{itemize}
  \item Microcontroller messaging. Teensy was flushed with new firmware that
    allows only messages of type \texttt{slip\_control\_communications/input\_drive}
    to be passed to it. (Previously, a message type of \texttt{f1tenth\_msgs/drive\_values}
    was hardcoded into Arduino's firmware, making communication within package
    \texttt{slip\_control} impossible.)
  \item Time constant for velocity found, although significant differences
    between time constants have been found for different velocity references.
  \item ROS infrastructure set.
  \item MPC python package found: \texttt{cvxopt}. It seems to capture the
    essence of what our goal is. See \url{http://nbviewer.jupyter.org/github/cvxgrp/cvx_short_course/blob/master/intro/control.ipynb}
\end{itemize}



\section{Ongoing}

\begin{itemize}
  \item Theoretical solution involving MPC and traveling in the middle of two
    walls found. The problem can be decomposed into two separate and independent
    components involving a translation and a rotation of the vehicle. Given the
    pose of the vehicle at time $t$ as $(x_c, y_c, v_c, \psi_v)$, two range
    scans at $-90^\circ$ and $+90^\circ$ with respect to the longitudinal axis of
    the vehicle which are denoted as $CL$ and $CR$ respectively, the error in
    translational terms is

    \begin{align}
      e_x &= \dfrac{CR-CL}{2}sin\psi \\
      e_y &= \dfrac{LC-CR}{2}cos\psi
    \end{align}

    In other words, at time $t$ the vehicle should have been at point

    \begin{align}
      x_o &=x_c + \dfrac{CR-CL}{2}sin\psi \\
      y_o &=y_c + \dfrac{LC-CR}{2}cos\psi
    \end{align}

    With regard to rotation, given the pose of the vehicle at time $t$ as
    $(x_c, y_c, v_c, \psi_v)$, three range scans at $-90^\circ$, $0^\circ$ and
    $+90^\circ$ with respect to the longitudinal axis of the vehicle which are
    denoted as $CL$, $CF$ and $CR$ respectively, the heading angle error is

    \begin{align}
      \phi = \dfrac{\pi}{2} - tan^{-1}\dfrac{CF}{CR}
    \end{align}

    In other words, at time $t$ the vehicle should have a heading angle of

    \begin{align}
      \psi_0 = \psi + \dfrac{\pi}{2} - tan^{-1}\dfrac{CF}{CR}
    \end{align}

\end{itemize}



\section{Issues}

\begin{itemize}
  \item The ethernet adapter for the lidar is broken and needs to be replaced.
    This means that packages \texttt{circular\_mpc} and \texttt{centerline\_mpc}
    cannot be tested until communication with the lidar is fixed.
  \item The SML lab is booked for the week 14/11-18/11 (What about the weekend?),
    hence no MOCAP. This means that packages \texttt{circular\_pid} and
    \texttt{centerline\_pid}(gains need adjusting) cannot be tested until at
    least Saturday 19/11.
  \item Package \texttt{circular\_pid}, which was to be working out-of-the-box,
    does not work. The fault lies somewhere inside ROS: it appears that when
    ROS\_MASTER runs outside Jetson, sometimes communication between Jetson and the
    nodes running outside it is not established. When it is established, no
    messages are getting through to teensy.
\end{itemize}

\end{document}
