\documentclass[oneside,12pt]{article}
\usepackage{fontspec}
\usepackage{lmodern}
%\setmainfont{cmr12}
\defaultfontfeatures{Ligatures=TeX} % To support LaTeX quoting style
\usepackage{amsmath}
\usepackage{lscape}
\usepackage{graphicx}
\usepackage{pgfplots}
\usepackage{subcaption}
\usepackage[margin=1in]{geometry}
\usepackage{float}
\usepackage{url}
\usepackage{pstricks}

\widowpenalty=1000
\clubpenalty=1000


\title{EL2425 - Slip Control - Meeting agenda 2016-11-16}
%\date{}

\begin{document}
\maketitle

\section{Done}

\begin{itemize}
  \item Microcontroller messaging. Teensy was flushed with new firmware that
    allows only messages of type \texttt{slip\_control\_communications/input\_drive}
    to be passed to it. (Previously, a message type of \texttt{f1tenth\_msgs/drive\_values}
    was hardcoded into Arduino's firmware, making communication within package
    \texttt{slip\_control} impossible due to incompatibility.)
  \item Time constant for velocity found, although significant differences
    between time constants have been found for different velocity references.
  \item ROS infrastructure set.
  \item MPC python package found: \texttt{cvxopt}. It seems to be able to
    capture the essence of what our goal is.
    See \url{http://nbviewer.jupyter.org/github/cvxgrp/cvx_short_course/blob/master/intro/control.ipynb}
  \item Theoretical solution involving MPC and traveling on the circumference
    of a circle found.

\end{itemize}



\section{Ongoing}

\begin{itemize}
  \item Theoretical solution involving MPC and traveling in the middle of two
    walls found. The problem can be decomposed into two separate and independent
    components involving a translation and a rotation of the vehicle. Given the
    pose of the vehicle at time $t$ as $(x_c, y_c, v_c, \psi_v)$ and two range
    scans at $-90^\circ$ and $+90^\circ$ with respect to the longitudinal axis of
    the vehicle which are denoted as $CL$ and $CR$ respectively, the error in
    translational terms is

    \begin{align}
      e_x &= \dfrac{CR-CL}{2}sin\psi \\
      e_y &= \dfrac{LC-CR}{2}cos\psi
    \end{align}

    In other words, at time $t$ the vehicle should have been at point

    \begin{align}
      x_o &=x_c + \dfrac{CR-CL}{2}sin\psi \\
      y_o &=y_c + \dfrac{LC-CR}{2}cos\psi
    \end{align}

    With regard to rotation, given the pose of the vehicle at time $t$ as
    $(x_c, y_c, v_c, \psi_v)$ and three range scans at $-90^\circ$, $0^\circ$ and
    $+90^\circ$ with respect to the longitudinal axis of the vehicle which are
    denoted as $CL$, $CF$ and $CR$ respectively, the heading angle error is

    \begin{align}
      \phi = \dfrac{\pi}{2} - tan^{-1}\dfrac{CF}{CR}
    \end{align}

    since $\mu + \phi + \dfrac{\pi}{2} = \pi$ and $tan(\mu) = \dfrac{CF}{CR}$.

    In other words, at time $t$ the vehicle should have a heading angle of

    \begin{align}
      \psi_o = \psi_c + \phi = \psi_c + \dfrac{\pi}{2} - tan^{-1}\dfrac{CF}{CR}
    \end{align}


    \begin{figure}[H]
      \scalebox{1}{% Generated with LaTeXDraw 2.0.8
% Tue Nov 15 16:19:30 CET 2016
% \usepackage[usenames,dvipsnames]{pstricks}
% \usepackage{epsfig}
% \usepackage{pst-grad} % For gradients
% \usepackage{pst-plot} % For axes
\scalebox{1} % Change this value to rescale the drawing.
{
\begin{pspicture}(0,-4.53)(11.419063,4.53)
\psline[linewidth=0.04cm](0.0,-2.95)(4.24,3.49)
\psline[linewidth=0.04cm](5.84,-4.51)(10.08,1.93)
\psdots[dotsize=0.12](5.76,0.59)
\psline[linewidth=0.012cm,linestyle=dashed,dash=0.16cm 0.16cm](3.26,-3.21)(7.5,3.23)
\psline[linewidth=0.04cm,arrowsize=0.05291667cm 2.0,arrowlength=1.4,arrowinset=0.4]{->}(5.76,0.57)(10.62,2.73)
\psline[linewidth=0.012cm,arrowsize=0.05291667cm 2.0,arrowlength=1.4,arrowinset=0.4]{<-}(5.76,2.59)(5.76,-1.41)
\psline[linewidth=0.012cm,arrowsize=0.05291667cm 2.0,arrowlength=1.4,arrowinset=0.4]{->}(3.76,0.59)(7.76,0.59)
\psarc[linewidth=0.02](6.18,0.59){0.26}{0.0}{60.0}
\usefont{T1}{ptm}{m}{n}
\rput(5.4045315,2.32){$y$}
\usefont{T1}{ptm}{m}{n}
\rput(7.5045314,0.26){$x$}
\usefont{T1}{ptm}{m}{n}
\rput(7.1545315,0.82){$\psi_c$}
\psarc[linewidth=0.02](6.11,0.9){0.21}{336.80142}{93.81407}
\usefont{T1}{ptm}{m}{n}
\rput(6.664531,1.3){$\phi$}
\psline[linewidth=0.04cm,arrowsize=0.05291667cm 2.0,arrowlength=1.4,arrowinset=0.4]{->}(5.76,0.59)(4.48,3.89)
\psline[linewidth=0.04cm,arrowsize=0.05291667cm 2.0,arrowlength=1.4,arrowinset=0.4]{->}(5.76,0.59)(7.08,-2.65)
\usefont{T1}{ptm}{m}{n}
\rput(5.324531,0.36){$C$}
\usefont{T1}{ptm}{m}{n}
\rput(3.8545313,3.74){$L$}
\usefont{T1}{ptm}{m}{n}
\rput(7.4445314,-2.88){$R$}
\psline[linewidth=0.04cm](9.8,1.51)(10.84,3.07)
\usefont{T1}{ptm}{m}{n}
\rput(11.054531,2.7){$F$}
\psarc[linewidth=0.02](10.13,2.34){0.21}{154.44003}{306.25385}
\usefont{T1}{ptm}{m}{n}
\rput(9.8645315,2.02){$\phi$}
\psdots[dotsize=0.012](8.08,-0.81)
\psdots[dotsize=0.012](3.46,2.05)
\psline[linewidth=0.04cm](3.78,2.79)(4.9,4.51)
\psarc[linewidth=0.02](7.11,-2.24){0.21}{0.0}{180.0}
\usefont{T1}{ptm}{m}{n}
\rput(7.1745315,-1.72){$\mu$}
\psline[linewidth=0.04cm](6.0,0.67)(6.12,0.43)
\psline[linewidth=0.04cm](6.12,0.45)(5.86,0.31)
\psdots[dotsize=0.034](5.94,0.51)
\end{pspicture} 
}

}
      \caption{}
      \label{}
    \end{figure}

    Hence we can formulate the optimization problem as

    \begin{align}
      min &\sum\limits_{k=0}^{k=N} (X-X_o)^T Q (X-X_o) + U^T R U \\
      \text{subject to } & X[t+1] = A X[t] + B U[t] \\
      & -U_{min} \leq U \leq U_{max}
    \end{align}

    where $X=[x_c, y_c, v_c, \psi_c]^T$ and $X_o = [x_o, y_o, v_o, \psi_o]^T$.
    Positive definite matrices $Q,R$ will have to be adjusted experimentally.

    However, the vehicle's velocity is not measurable since the vehicle does not
    have encoders connected to its wheels and MOCAP or range scans cannot
    provide measurements of velocity. Either a Kalman filter will have to be
    employed in order to estimate the vehicle's velocity, or the ESC feature of
    the vehicle will have to be investigated with regard to its ability to
    ensure that the input velocity is indeed the vehicle's velocity.
\end{itemize}



\section{Issues}

\begin{itemize}
  \item The ethernet adapter for the lidar is broken and needs to be replaced.
    This means that packages \texttt{circular\_mpc} and \texttt{centerline\_mpc}
    cannot be tested until communication with the lidar is fixed.
  \item The SML lab is booked for the week 14/11-18/11 (what about the weekend?),
    hence no MOCAP. This means that packages \texttt{circular\_pid} and
    \texttt{centerline\_pid} (gains need adjusting) cannot be tested until at
    least Saturday 19/11.
  \item Package \texttt{circular\_pid}, which was to be working out-of-the-box,
    does not work. The fault lies somewhere inside ROS: it appears that when
    ROS\_MASTER runs outside Jetson, sometimes communication between Jetson and the
    nodes running outside it is not established. When it is established, no
    messages are getting through to teensy.
\end{itemize}

\end{document}
