\documentclass[oneside,12pt]{article}
\usepackage{fontspec}
\usepackage{lmodern}
%\setmainfont{cmr12}
\defaultfontfeatures{Ligatures=TeX} % To support LaTeX quoting style
\usepackage{amsmath}
\usepackage{lscape}
\usepackage{graphicx}
\usepackage{pgfplots}
\usepackage{subcaption}
\usepackage[margin=1in]{geometry}
\usepackage{float}
\usepackage{url}
\usepackage{pstricks}

\widowpenalty=1000
\clubpenalty=1000


\title{EL2425 - Slip Control \\ Meeting agenda 2016-11-16}
%\date{}

\begin{document}
\maketitle

\section{Done}

\begin{itemize}
  \item \textbf{Microcontroller messaging}. Teensy was flushed with new firmware that
    allows only messages of type \texttt{slip\_control\_communications/input\_drive}
    to be passed to it. (Previously, a message type of \texttt{f1tenth\_msgs/drive\_values}
    was hardcoded into Arduino's firmware, making communication within package
    \texttt{slip\_control} impossible due to incompatibility.)
  \item \textbf{Time constant} pertaining to the velocity response of the vehicle
    found, although significant differences between time constants have been
    found for different velocity references.
  \item \textbf{ROS infrastructure} set.
  \item \textbf{MPC python package} found: \texttt{cvxopt}. It seems to be able to
    capture the essence of what our goal is.
    See \url{http://nbviewer.jupyter.org/github/cvxgrp/cvx_short_course/blob/master/intro/control.ipynb}
  \item Theoretical solution involving PID and traveling on the circumference
    of a circle found, and code for it written. However, the involved gains
    have to be tuned experimentally.
  \item Theoretical solution involving MPC and tracking the centerline of a lane
    circle found, however code for it has not been written yet.

\end{itemize}



\section{Ongoing}

\begin{itemize}
  \item Theoretical solution involving MPC and tracking the centerline of a lane.

    The problem can be decomposed into two separate and independent
    components involving a translation and a rotation of the vehicle.


    \textbf{Translational component} Given the pose of the vehicle at time $t$
    as $(x_c, y_c, v_c, \psi_v)$ and two range scans at $-90^\circ$ and
    $+90^\circ$ with respect to the longitudinal axis of the vehicle which are
    denoted as $CL$ and $CR$ respectively, the error in translational terms is

    \begin{align}
      e_x &= -\dfrac{CL-CR}{2}\text{sin}\psi \\
      e_y &= \dfrac{CL-CR}{2}\text{cos}\psi
    \end{align}

    since $CL + CR = L = OL + OR$, and $CL = OC + OL$, which means that

    \begin{equation}
      OC = OR - CR = \dfrac{L}{2} - CR = \dfrac{CL + CR}{2} - CR = \dfrac{CL-CR}{2}
    \end{equation}
    where $L$ is the width of the lane whose centerline the vehicle is to track.

    Furthermore, in the $COM$ triangle:

    \begin{align}
      O'C &= OC \text{cos}\mu = OC \text{cos}(\dfrac{\pi}{2} - \psi) = OC \text{sin}\psi = \dfrac{CL-CR}{2} \text{sin}\psi\\
      O'O &= OC \text{sin}\mu = OC \text{sin}(\dfrac{\pi}{2} - \psi) = OC \text{cos}\psi = \dfrac{CL-CR}{2} \text{cos}\psi
    \end{align}

    In other words, at time $t$ the vehicle should have been at point $O(x_o, y_o)$:

    \begin{align}
      x_o &=x_c - \dfrac{CL-CR}{2}\text{sin}\psi \\
      y_o &=y_c + \dfrac{CL-CR}{2}\text{cos}\psi
    \end{align}

    \begin{figure}[H]\centering
      \scalebox{1}{% Generated with LaTeXDraw 2.0.8
% Tue Nov 15 16:54:04 CET 2016
% \usepackage[usenames,dvipsnames]{pstricks}
% \usepackage{epsfig}
% \usepackage{pst-grad} % For gradients
% \usepackage{pst-plot} % For axes
\scalebox{1} % Change this value to rescale the drawing.
{
\begin{pspicture}(0,-4.53)(10.1,4.53)
\psline[linewidth=0.04cm](0.0,-2.95)(4.24,3.49)
\psline[linewidth=0.04cm](5.84,-4.51)(10.08,1.93)
\psline[linewidth=0.012cm,linestyle=dashed,dash=0.16cm 0.16cm](3.26,-3.21)(7.5,3.23)
\psline[linewidth=0.04cm,arrowsize=0.05291667cm 2.0,arrowlength=1.4,arrowinset=0.4]{->}(6.94,-0.49)(9.34,3.15)
\psline[linewidth=0.012cm,arrowsize=0.05291667cm 2.0,arrowlength=1.4,arrowinset=0.4]{<-}(6.94,1.51)(6.94,-2.49)
\psline[linewidth=0.012cm,arrowsize=0.05291667cm 2.0,arrowlength=1.4,arrowinset=0.4]{->}(4.94,-0.49)(8.94,-0.49)
\psarc[linewidth=0.02](6.94,-0.49){0.26}{0.0}{60.0}
\usefont{T1}{ptm}{m}{n}
\rput(6.533594,1.3){$y$}
\usefont{T1}{ptm}{m}{n}
\rput(8.813594,-0.84){$x$}
\usefont{T1}{ptm}{m}{n}
\rput(7.743594,-0.2){$\psi_c$}
\usefont{T1}{ptm}{m}{n}
\rput(6.4535937,-0.72){$C$}
\usefont{T1}{ptm}{m}{n}
\rput(2.6435938,2.1){$L$}
\usefont{T1}{ptm}{m}{n}
\rput(8.373594,-1.56){$R$}
\psdots[dotsize=0.012](8.08,-0.81)
\psdots[dotsize=0.012](3.46,2.05)
\psline[linewidth=0.04cm](3.78,2.79)(4.9,4.51)
\psline[linewidth=0.04cm](8.04,-1.17)(3.18,1.87)
\psdots[dotsize=0.12](5.6,0.37)
\usefont{T1}{ptm}{m}{n}
\rput(5.094531,0.28){$O$}
\psline[linewidth=0.04cm](5.78,0.63)(5.5,0.83)
\psline[linewidth=0.04cm](5.5,0.83)(5.3,0.55)
\psdots[dotsize=0.066](5.54,0.61)
\psline[linewidth=0.04cm](8.2,-0.89)(7.92,-0.69)
\psline[linewidth=0.04cm](7.92,-0.69)(7.72,-0.97)
\psdots[dotsize=0.066](7.98,-0.93)
\psline[linewidth=0.04cm](7.12,-0.21)(6.84,-0.01)
\psline[linewidth=0.04cm](6.84,-0.01)(6.64,-0.29)
\psdots[dotsize=0.066](6.88,-0.23)
\psarc[linewidth=0.02](6.56,-0.33){0.24}{135.0}{225.0}
\usefont{T1}{ptm}{m}{n}
\rput(6.114531,-0.3){$\mu$}
\psline[linewidth=0.02cm](5.6,0.33)(5.58,-0.49)
\psdots[dotsize=0.07](5.58,-0.49)
\psdots[dotsize=0.07](5.04,-0.49)
\usefont{T1}{ptm}{m}{n}
\rput(5.6945314,-0.74){$O'$}
\usefont{T1}{ptm}{m}{n}
\rput(4.7345314,-0.76){$M$}
\end{pspicture} 
}

}
      \caption{}
      \label{}
    \end{figure}

    \begin{figure}[H]\centering
      \scalebox{1}{% Generated with LaTeXDraw 2.0.8
% Tue Nov 15 17:04:15 CET 2016
% \usepackage[usenames,dvipsnames]{pstricks}
% \usepackage{epsfig}
% \usepackage{pst-grad} % For gradients
% \usepackage{pst-plot} % For axes
\scalebox{1} % Change this value to rescale the drawing.
{
\begin{pspicture}(0,-3.226)(9.08,3.226)
\psline[linewidth=0.012cm,linestyle=dashed,dash=0.16cm 0.16cm](0.0,-3.22)(4.24,3.22)
\psline[linewidth=0.04cm,arrowsize=0.05291667cm 2.0,arrowlength=1.4,arrowinset=0.4]{->}(6.66,-0.5)(9.06,3.14)
\psline[linewidth=0.012cm,arrowsize=0.05291667cm 2.0,arrowlength=1.4,arrowinset=0.4]{<-}(6.66,1.5)(6.66,-2.5)
\psline[linewidth=0.012cm,arrowsize=0.05291667cm 2.0,arrowlength=1.4,arrowinset=0.4]{->}(0.66,-0.5)(8.66,-0.5)
\psarc[linewidth=0.02](1.82,-0.5){0.26}{0.0}{60.0}
\usefont{T1}{ptm}{m}{n}
\rput(6.253594,1.29){$y$}
\psdots[dotsize=0.012](7.8,-0.82)
\psdots[dotsize=0.012](3.18,2.04)
\psline[linewidth=0.04cm](7.76,-1.18)(2.9,1.86)
\psline[linewidth=0.04cm](6.84,-0.22)(6.56,-0.02)
\psline[linewidth=0.04cm](6.56,-0.02)(6.36,-0.3)
\psdots[dotsize=0.066](6.6,-0.24)
\psarc[linewidth=0.02](6.28,-0.34){0.24}{135.0}{225.0}
\psarc[linewidth=0.02](6.7,-0.5){0.26}{0.0}{60.0}
\usefont{T1}{ptm}{m}{n}
\rput(2.6545312,-0.19){$\psi$}
\usefont{T1}{ptm}{m}{n}
\rput(7.474531,-0.29){$\psi$}
\usefont{T1}{ptm}{m}{n}
\rput(5.594531,-0.21){$\mu$}
\usefont{T1}{ptm}{m}{n}
\rput(8.604531,-0.79){$x$}
\psline[linewidth=0.04cm](3.0197916,1.3791598)(3.3004897,1.1801409)
\psline[linewidth=0.04cm](3.3004897,1.1801409)(3.4995086,1.460839)
\psdots[dotsize=0.07](3.26,1.44)
\usefont{T1}{ptm}{m}{n}
\rput(4.134531,1.93){$O$}
\usefont{T1}{ptm}{m}{n}
\rput(3.5345314,-0.87){$O'$}
\usefont{T1}{ptm}{m}{n}
\rput(1.0545312,-0.81){$M$}
\psline[linewidth=0.02cm](3.2,1.64)(3.22,-0.5)
\usefont{T1}{ptm}{m}{n}
\rput(6.264531,-0.81){$C$}
\end{pspicture} 
}

}
      \caption{}
      \label{}
    \end{figure}

    \textbf{Rotational component} With regard to rotation, given the pose of
    the vehicle at time $t$ as $(x_c, y_c, v_c, \psi_v)$ and three range scans
    at $-90^\circ$, $0^\circ$ and $+90^\circ$ with respect to the longitudinal
    axis of the vehicle which are denoted as $CL$, $CF$ and $CR$ respectively,
    the heading angle error is

    \begin{align}
      \phi = \dfrac{\pi}{2} - tan^{-1}\dfrac{CF}{CR}
    \end{align}

    since $\mu + \phi + \dfrac{\pi}{2} = \pi$ and $tan(\mu) = \dfrac{CF}{CR}$.

    In other words, at time $t$ the vehicle should have a heading angle of

    \begin{align}
      \psi_o = \psi_c + \phi = \psi_c + \dfrac{\pi}{2} - tan^{-1}\dfrac{CF}{CR}
    \end{align}


    \begin{figure}[H]\centering
      \scalebox{1}{% Generated with LaTeXDraw 2.0.8
% Tue Nov 15 16:19:30 CET 2016
% \usepackage[usenames,dvipsnames]{pstricks}
% \usepackage{epsfig}
% \usepackage{pst-grad} % For gradients
% \usepackage{pst-plot} % For axes
\scalebox{1} % Change this value to rescale the drawing.
{
\begin{pspicture}(0,-4.53)(11.419063,4.53)
\psline[linewidth=0.04cm](0.0,-2.95)(4.24,3.49)
\psline[linewidth=0.04cm](5.84,-4.51)(10.08,1.93)
\psdots[dotsize=0.12](5.76,0.59)
\psline[linewidth=0.012cm,linestyle=dashed,dash=0.16cm 0.16cm](3.26,-3.21)(7.5,3.23)
\psline[linewidth=0.04cm,arrowsize=0.05291667cm 2.0,arrowlength=1.4,arrowinset=0.4]{->}(5.76,0.57)(10.62,2.73)
\psline[linewidth=0.012cm,arrowsize=0.05291667cm 2.0,arrowlength=1.4,arrowinset=0.4]{<-}(5.76,2.59)(5.76,-1.41)
\psline[linewidth=0.012cm,arrowsize=0.05291667cm 2.0,arrowlength=1.4,arrowinset=0.4]{->}(3.76,0.59)(7.76,0.59)
\psarc[linewidth=0.02](6.18,0.59){0.26}{0.0}{60.0}
\usefont{T1}{ptm}{m}{n}
\rput(5.4045315,2.32){$y$}
\usefont{T1}{ptm}{m}{n}
\rput(7.5045314,0.26){$x$}
\usefont{T1}{ptm}{m}{n}
\rput(7.1545315,0.82){$\psi_c$}
\psarc[linewidth=0.02](6.11,0.9){0.21}{336.80142}{93.81407}
\usefont{T1}{ptm}{m}{n}
\rput(6.664531,1.3){$\phi$}
\psline[linewidth=0.04cm,arrowsize=0.05291667cm 2.0,arrowlength=1.4,arrowinset=0.4]{->}(5.76,0.59)(4.48,3.89)
\psline[linewidth=0.04cm,arrowsize=0.05291667cm 2.0,arrowlength=1.4,arrowinset=0.4]{->}(5.76,0.59)(7.08,-2.65)
\usefont{T1}{ptm}{m}{n}
\rput(5.324531,0.36){$C$}
\usefont{T1}{ptm}{m}{n}
\rput(3.8545313,3.74){$L$}
\usefont{T1}{ptm}{m}{n}
\rput(7.4445314,-2.88){$R$}
\psline[linewidth=0.04cm](9.8,1.51)(10.84,3.07)
\usefont{T1}{ptm}{m}{n}
\rput(11.054531,2.7){$F$}
\psarc[linewidth=0.02](10.13,2.34){0.21}{154.44003}{306.25385}
\usefont{T1}{ptm}{m}{n}
\rput(9.8645315,2.02){$\phi$}
\psdots[dotsize=0.012](8.08,-0.81)
\psdots[dotsize=0.012](3.46,2.05)
\psline[linewidth=0.04cm](3.78,2.79)(4.9,4.51)
\psarc[linewidth=0.02](7.11,-2.24){0.21}{0.0}{180.0}
\usefont{T1}{ptm}{m}{n}
\rput(7.1745315,-1.72){$\mu$}
\psline[linewidth=0.04cm](6.0,0.67)(6.12,0.43)
\psline[linewidth=0.04cm](6.12,0.45)(5.86,0.31)
\psdots[dotsize=0.034](5.94,0.51)
\end{pspicture} 
}

}
      \caption{}
      \label{}
    \end{figure}

    Hence we can formulate the optimization problem as

    \begin{align}
      min &\sum\limits_{k=0}^N (X-X_o)^T Q (X-X_o) + U^T R U \\
      \text{subject to } & X[t+1] = A X[t] + B U[t] \\
      & -U_{min} \leq U \leq U_{max}
    \end{align}

    where $X=[x_c, y_c, v_c, \psi_c]^T$, $X_o = [x_o, y_o, v_o, \psi_o]^T$,
    $N$ is the horizon and $U=[v, \delta]^T$ is the input vector.
    Positive definite matrices $Q,R$ will have to be adjusted experimentally.

    However, the vehicle's velocity is not measurable since the vehicle does not
    have encoders connected to its wheels and MOCAP or range scans cannot
    provide measurements of velocity. Either a Kalman filter will have to be
    employed in order to estimate the vehicle's velocity, or the ESC feature of
    the vehicle will have to be investigated with regard to its ability to
    ensure that the input velocity is indeed the vehicle's velocity.
\end{itemize}



\section{Issues}

\begin{itemize}
  \item The ethernet adapter for the lidar is broken and needs to be replaced.
    This means that packages \texttt{circular\_mpc} and \texttt{centerline\_mpc}
    cannot be tested until communication with the lidar is fixed.
  \item The SML lab is booked for the week 14/11-18/11 (what about the weekend?),
    hence there is no access to MOCAP. This means that packages
    \texttt{circular\_pid} and \texttt{centerline\_pid} (gains need adjusting)
    cannot be tested until at least Saturday 19/11.
  \item Package \texttt{circular\_pid}, which was to be working out-of-the-box,
    does not work. The fault lies somewhere within ROS: it appears that when
    ROS\_MASTER runs outside Jetson, sometimes communication between Jetson and the
    nodes running outside it is not established. When it is established, no
    messages are getting through to teensy.
\end{itemize}


\section{To do}

\begin{itemize}
  \item Tune gains of the PID concerning package \texttt{centerline\_pid}.
  \item Implement \texttt{centerline\_mpc}.
  \item Implement \texttt{circular\_pid}.
  \item A node that handles the linearization of the kinematic model of the
    vehicle has to be written in ROS.
\end{itemize}

\section{Misc.}

The progress of the project can be observed in \texttt{trello} and \texttt{github}:

\begin{itemize}
  \item \url{https://trello.com/b/uEP0jl0B/slip-control}
  \item \url{https://gits-15.sys.kth.se/alefil/HT16_P2_EL2425}
  \item \url{https://gits-15.sys.kth.se/alefil/HT16_P2_EL2425_resources}
\end{itemize}


\end{document}
