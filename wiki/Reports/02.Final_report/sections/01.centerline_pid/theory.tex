The problem can be decomposed into two separate and independent
components involving a translation and a rotation of the vehicle.

\subsubsection{Translational component}

We assume that the pose of the vehicle at time $t$ is
$(x_c, y_c, v_c, \psi_v)$ and that two range scans at $+90^\circ$ and
$-90^\circ$ with respect to the longitudinal axis of the
vehicle are available, which are denoted as $CL$ and $CR$ respectively. Also,
in figure \ref{fig:centerline_pid_translation}, point $O'$ is a future
reference and the angle $\lambda$ is the angle that the vehicle should
make in order to reach $O'$ in a future moment.

Then, since $CL + CR = L = OL + OR$ and $CL = OC + OL$, this means that

\begin{equation}
  OC = OR - CR = \dfrac{L}{2} - CR = \dfrac{CL + CR}{2} - CR = \dfrac{CL-CR}{2}
\end{equation}
where $L$ is the width of the lane whose centerline the vehicle is to track.

Furthermore, in the $CC'O'$ triangle

\begin{align}
  \text{tan}\lambda &= \dfrac{O'C'}{CC'} = \dfrac{OC}{CC'} = \dfrac{CL-CR}{2CC'} \\
  \lambda &= \text{tan}^{-1}\dfrac{CL-CR}{2CC'}
\end{align}

The length $CC'$ is unknown and can be set beforehand. Its magnitude
will determine the vehicle's rate of convergence to the centerline of the
lane.

\begin{figure}[H]\centering
  \scalebox{1}{% Generated with LaTeXDraw 2.0.8
% Thu Nov 17 16:22:33 CET 2016
% \usepackage[usenames,dvipsnames]{pstricks}
% \usepackage{epsfig}
% \usepackage{pst-grad} % For gradients
% \usepackage{pst-plot} % For axes
\scalebox{1} % Change this value to rescale the drawing.
{
\begin{pspicture}(0,-4.77)(10.711719,4.77)
\psline[linewidth=0.04cm](0.0,-3.19)(4.24,3.25)
\psline[linewidth=0.04cm](5.84,-4.75)(10.08,1.69)
\psline[linewidth=0.012cm,linestyle=dashed,dash=0.16cm 0.16cm](3.26,-3.45)(7.5,2.99)
\psline[linewidth=0.04cm,arrowsize=0.05291667cm 2.0,arrowlength=1.4,arrowinset=0.4]{->}(6.94,-0.73)(9.34,2.91)
\usefont{T1}{ptm}{m}{n}
\rput(2.5671873,1.86){$L$}
\usefont{T1}{ptm}{m}{n}
\rput(8.297188,-1.8){$R$}
\psdots[dotsize=0.012](8.08,-1.05)
\psdots[dotsize=0.012](3.46,1.81)
\psline[linewidth=0.04cm](8.04,-1.41)(3.18,1.63)
\psdots[dotsize=0.12](5.6,0.13)
\usefont{T1}{ptm}{m}{n}
\rput(5.0181246,0.04){$O$}
\psline[linewidth=0.04cm](5.78,0.39)(5.5,0.59)
\psline[linewidth=0.04cm](5.5,0.59)(5.3,0.31)
\psdots[dotsize=0.066](5.54,0.37)
\psline[linewidth=0.04cm](8.2,-1.13)(7.92,-0.93)
\psline[linewidth=0.04cm](7.92,-0.93)(7.72,-1.21)
\psdots[dotsize=0.066](7.98,-1.17)
\psline[linewidth=0.04cm](7.12,-0.45)(6.84,-0.25)
\psline[linewidth=0.04cm](6.84,-0.25)(6.64,-0.53)
\psdots[dotsize=0.066](6.88,-0.47)
\psline[linewidth=0.04cm](9.88,1.41)(5.02,4.45)
\psline[linewidth=0.04cm](3.98,2.85)(5.2,4.75)
\usefont{T1}{ptm}{m}{n}
\rput(4.2671876,4.48){$L'$}
\usefont{T1}{ptm}{m}{n}
\rput(10.277187,1.04){$R'$}
\usefont{T1}{ptm}{m}{n}
\rput(7.3071876,3.32){$O'$}
\usefont{T1}{ptm}{m}{n}
\rput(9.317187,2.18){$C'$}
\psline[linewidth=0.01cm](6.96,-0.75)(7.48,2.95)
\psline[linewidth=0.04cm](10.06,1.69)(9.78,1.89)
\psline[linewidth=0.04cm](9.78,1.89)(9.58,1.61)
\psdots[dotsize=0.066](9.82,1.65)
\psarc[linewidth=0.02](7.32,0.27){0.22}{336.80142}{153.43495}
\usefont{T1}{ptm}{m}{n}
\rput(7.3835936,0.72){$\lambda$}
\usefont{T1}{ptm}{m}{n}
\rput(6.4571877,-1.04){$C$}
\end{pspicture} 
}

}
  \caption{The vehicle's heading angle is equal to that of the lane's.
    However, its position is off track.}
  \label{fig:centerline_pid_translation}
\end{figure}

The angular error of the translational component is then

\begin{align}
  e_t &= -\text{tan}^{-1}\dfrac{CL-CR}{2CC'}
\end{align}

where the minus sign is introduced due to the
convention that a left turn is assigned a negative value.

\subsubsection{Rotational component}

The rotational component problem can be solved in two ways, of which the
second is more robust than the first.

In the first solution, we assume that the the pose of the vehicle at time $t$ as
$(x_c, y_c, v_c, \psi_v)$ and that three range scans
at $+90^\circ$, $0^\circ$ and $-90^\circ$ with respect to the
longitudinal axis of the vehicle are available, which are denoted as
$CL$, $CF$ and $CR$ respectively.

Here we can distinguish two cases: one where the vehicle is facing the right
lane boundary and one where it is facing the left one. It is not obvious
in this configuration where the vehicle is heading: the only available
information so far is only the three range scans.

In the first case, the heading angle error is

\begin{align}
  \phi = \text{tan}^{-1}\dfrac{CR}{CF}
\end{align}

\begin{figure}[H]\centering
  \scalebox{1}{% Generated with LaTeXDraw 2.0.8
% Thu Nov 17 16:07:45 CET 2016
% \usepackage[usenames,dvipsnames]{pstricks}
% \usepackage{epsfig}
% \usepackage{pst-grad} % For gradients
% \usepackage{pst-plot} % For axes
\scalebox{1} % Change this value to rescale the drawing.
{
\begin{pspicture}(0,-3.530157)(11.125559,3.5298147)
\psline[linewidth=0.04cm](0.0,2.850961)(7.7104354,2.8706734)
\psline[linewidth=0.04cm](1.9231296,-2.879726)(9.633565,-2.8600135)
\psdots[dotsize=0.12,dotangle=-56.493206](6.1314507,0.002362453)
\psline[linewidth=0.012cm,linestyle=dashed,dash=0.16cm 0.16cm](1.5828435,-0.010822799)(9.293279,0.0088896165)
\psline[linewidth=0.04cm,arrowsize=0.05291667cm 2.0,arrowlength=1.4,arrowinset=0.4]{->}(6.114774,-0.008678264)(10.598721,-2.8686478)
\rput{-56.493206}(3.0476913,5.436149){\psarc[linewidth=0.02](6.5831475,-0.11834347){0.21}{336.80142}{93.81407}}
\usefont{T1}{ptm}{m}{n}
\rput{-1.4418001}(0.010640653,0.1878805){\rput(7.4411206,-0.3181057){$\phi$}}
\psline[linewidth=0.04cm,arrowsize=0.05291667cm 2.0,arrowlength=1.4,arrowinset=0.4]{->}(6.1314507,0.002362453)(8.176452,2.8913708)
\psline[linewidth=0.04cm,arrowsize=0.05291667cm 2.0,arrowlength=1.4,arrowinset=0.4]{->}(6.1314507,0.002362453)(4.15856,-2.8868766)
\usefont{T1}{ptm}{m}{n}
\rput{1.3045566}(0.0068035475,-0.12917721){\rput(5.6465855,0.23389597){$C$}}
\usefont{T1}{ptm}{m}{n}
\rput{0.040597137}(0.002358909,-0.0054432596){\rput(7.6533914,3.3261545){$L$}}
\usefont{T1}{ptm}{m}{n}
\rput{-1.0306425}(0.060403477,0.07402496){\rput(4.1153083,-3.321177){$R$}}
\psline[linewidth=0.04cm](9.12879,-2.858399)(11.003667,-2.864396)
\usefont{T1}{ptm}{m}{n}
\rput{-0.047520056}(0.0027001218,0.008948633){\rput(10.760876,-3.2514231){$F$}}
\rput{-56.493206}(6.711778,7.142263){\psarc[linewidth=0.02](10.003033,-2.67537){0.21}{154.44003}{306.25385}}
\usefont{T1}{ptm}{m}{n}
\rput{-1.0458043}(0.047882155,0.16672444){\rput(9.127907,-2.529071){$\phi$}}
\psdots[dotsize=0.012,dotangle=-56.493206](6.2448254,-2.7049513)
\psdots[dotsize=0.012,dotangle=-56.493206](6.079146,2.7261217)
\psline[linewidth=0.04cm](6.8728247,2.8678057)(8.925276,2.8834286)
\psline[linewidth=0.04cm](6.3306446,-0.15359159)(6.196772,-0.38613862)
\psline[linewidth=0.04cm](6.2134485,-0.37509787)(5.9531846,-0.23558962)
\psdots[dotsize=0.034,dotangle=-56.493206](6.1641116,-0.19188808)
\end{pspicture} 
}

}
  \caption{The vehicle's position and its reference are equal. However,
    the vehicle's heading is off track. Its heading is towards the right
    lane boundary.}
  \label{}
\end{figure}

The rotational error of the vehicle in this case is

\begin{align}
  e_r &= -\text{tan}^{-1}\dfrac{CR}{CF}
\end{align}

therefore the overall angular error of the vehicle in this case is

\begin{align}
  e &= -\text{tan}^{-1}\dfrac{CL-CR}{2CC'} - \text{tan}^{-1}\dfrac{CR}{CF}
\end{align}

where the minus signs are introduced due to the
convention that a left turn is assigned a negative value.


In the second case, the heading angle error is

\begin{align}
  \phi = \text{tan}^{-1}\dfrac{CL}{CF}
\end{align}

\begin{figure}[H]\centering
  \scalebox{1}{% Generated with LaTeXDraw 2.0.8
% Thu Nov 17 16:07:15 CET 2016
% \usepackage[usenames,dvipsnames]{pstricks}
% \usepackage{epsfig}
% \usepackage{pst-grad} % For gradients
% \usepackage{pst-plot} % For axes
\scalebox{1} % Change this value to rescale the drawing.
{
\begin{pspicture}(0,-3.411386)(11.176769,3.4110463)
\psline[linewidth=0.04cm](0.0,2.869732)(7.7104354,2.889444)
\psline[linewidth=0.04cm](1.9231296,-2.860955)(9.633565,-2.8412428)
\psdots[dotsize=0.12,dotangle=-56.493206](6.1314507,0.02113334)
\psline[linewidth=0.012cm,linestyle=dashed,dash=0.16cm 0.16cm](1.5828435,0.007948088)(9.293279,0.027660504)
\psline[linewidth=0.04cm,arrowsize=0.05291667cm 2.0,arrowlength=1.4,arrowinset=0.4]{->}(6.1360335,0.0029427796)(9.736769,2.8748085)
\rput{-12.778679}(0.12707359,1.4556472){\psarc[linewidth=0.02](6.5631475,0.16042735){0.21}{336.80142}{93.81407}}
\usefont{T1}{ptm}{m}{n}
\rput{-1.4418001}(-0.0059856703,0.18406323){\rput(7.2811203,0.3406652){$\phi$}}
\psline[linewidth=0.04cm,arrowsize=0.05291667cm 2.0,arrowlength=1.4,arrowinset=0.4]{->}(6.1310472,0.022311239)(4.0687327,2.898986)
\psline[linewidth=0.04cm,arrowsize=0.05291667cm 2.0,arrowlength=1.4,arrowinset=0.4]{->}(6.1310472,0.022311239)(8.217139,-2.7862859)
\usefont{T1}{ptm}{m}{n}
\rput{1.3045566}(0.007153142,-0.122342296){\rput(5.3465853,0.2526669){$C$}}
\usefont{T1}{ptm}{m}{n}
\rput{0.040597137}(0.002272063,-0.0027649575){\rput(3.8733912,3.2049253){$L$}}
\usefont{T1}{ptm}{m}{n}
\rput{-1.0306425}(0.058953077,0.1503096){\rput(8.355308,-3.2024062){$R$}}
\psline[linewidth=0.04cm](9.12879,-2.839628)(11.003667,-2.8456254)
\usefont{T1}{ptm}{m}{n}
\rput{-0.047520056}(-0.0026569718,0.008204411){\rput(9.860876,3.2073479){$F$}}
\rput{-56.493206}(1.9299511,8.9591255){\psarc[linewidth=0.02](9.303033,2.6834006){0.21}{154.44003}{306.25385}}
\usefont{T1}{ptm}{m}{n}
\rput{-1.0458043}(-0.044197325,0.1551526){\rput(8.447907,2.5096996){$\phi$}}
\psdots[dotsize=0.012,dotangle=-56.493206](6.2448254,-2.68618)
\psdots[dotsize=0.012,dotangle=-56.493206](6.079146,2.7448926)
\psline[linewidth=0.04cm](6.8728247,2.8865767)(11.156769,2.8948085)
\psline[linewidth=0.04cm](6.3435235,0.15962021)(6.5195694,-0.042883728)
\psline[linewidth=0.04cm](6.514583,-0.023515271)(6.297697,-0.22391587)
\psdots[dotsize=0.034,dotangle=14.436867](6.3253083,-0.0102862455)
\end{pspicture} 
}

}
  \caption{The vehicle's position and its reference are equal. However,
    the vehicle's heading is off track. Its heading is towards the left
    lane boundary.}
  \label{}
\end{figure}

The rotational error of the vehicle in this case is

\begin{align}
  e_r &= \text{tan}^{-1}\dfrac{CL}{CF}
\end{align}

therefore the overall angular error of the vehicle in this case is

\begin{align}
  e &= -\text{tan}^{-1}\dfrac{CL-CR}{2CC'} + \text{tan}^{-1}\dfrac{CL}{CF}
\end{align}

where the minus sign of the translational error is introduced due to the
convention that a left turn is assigned a negative value.


In order to deduce the correct value of $\phi$ ($-\tan^{-1}\dfrac{CR}{CF}$ or
$\tan^{-1}\dfrac{CL}{CF}$) further ranges scans are needed. To this end,
a difference between ranges around point $F$ is taken: starting at the
right of $F$ and moving anti-clockwise, we calculate the difference between
two range scans for a given angle between them. If its sign
is negative then the vehicle is facing the right lane boundary; if not,
it is facing the left lane boundary. This concept is depicted in figures
\ref{fig:range_diff_positive} and  \ref{fig:range_diff_negative}.

\begin{figure}[H]\centering
  \scalebox{1}{% Generated with LaTeXDraw 2.0.8
% Thu Nov 17 16:56:03 CET 2016
% \usepackage[usenames,dvipsnames]{pstricks}
% \usepackage{epsfig}
% \usepackage{pst-grad} % For gradients
% \usepackage{pst-plot} % For axes
\scalebox{1} % Change this value to rescale the drawing.
{
\begin{pspicture}(0,-3.4309542)(12.034648,3.4306142)
\psline[linewidth=0.04cm](0.0,2.8490846)(7.7104354,2.8687966)
\psline[linewidth=0.04cm](1.9231296,-2.8816025)(9.633565,-2.8618903)
\psdots[dotsize=0.12,dotangle=-56.493206](6.1314507,0.0)
\psline[linewidth=0.012cm,linestyle=dashed,dash=0.16cm 0.16cm](1.5828435,-0.012699386)(9.293279,0.0070130304)
\psline[linewidth=0.04cm,arrowsize=0.05291667cm 2.0,arrowlength=1.4,arrowinset=0.4]{->}(6.1360335,-0.017704694)(9.736769,2.854161)
\rput{-12.778679}(0.13164052,1.4551358){\psarc[linewidth=0.02](6.5631475,0.13977982){0.21}{336.80142}{93.81407}}
\usefont{T1}{ptm}{m}{n}
\rput{-1.4418001}(-0.0055199126,0.18278961){\rput(7.2307444,0.3215206){$\phi$}}
\psline[linewidth=0.04cm,arrowsize=0.05291667cm 2.0,arrowlength=1.4,arrowinset=0.4]{->}(6.1310472,0.0016637653)(4.0687327,2.8783386)
\psline[linewidth=0.04cm,arrowsize=0.05291667cm 2.0,arrowlength=1.4,arrowinset=0.4]{->}(6.1310472,0.0016637653)(8.217139,-2.8069334)
\usefont{T1}{ptm}{m}{n}
\rput{1.3045566}(0.006638773,-0.12118897){\rput(5.295677,0.23065355){$C$}}
\usefont{T1}{ptm}{m}{n}
\rput{0.040597137}(0.00225739,-0.002728871){\rput(3.822454,3.1842353){$L$}}
\usefont{T1}{ptm}{m}{n}
\rput{-1.0306425}(0.059296813,0.14939022){\rput(8.30437,-3.2219744){$R$}}
\psline[linewidth=0.04cm](9.12879,-2.8602755)(11.003667,-2.866273)
\usefont{T1}{ptm}{m}{n}
\rput{-0.047520056}(-0.0026730814,0.008162171){\rput(9.809938,3.2267501){$F$}}
\psdots[dotsize=0.012,dotangle=-56.493206](6.2448254,-2.7068274)
\psdots[dotsize=0.012,dotangle=-56.493206](6.079146,2.724245)
\psline[linewidth=0.04cm](6.8728247,2.8659294)(11.156769,2.874161)
\psline[linewidth=0.04cm](6.3435235,0.13897273)(6.5195694,-0.063531205)
\psline[linewidth=0.04cm](6.514583,-0.044162747)(6.297697,-0.24456334)
\psdots[dotsize=0.034,dotangle=14.436867](6.3253083,-0.03093372)
\psline[linewidth=0.04cm,linecolor=red](6.12,0.01042361)(11.14,2.8704236)
\psline[linewidth=0.04cm,linecolor=red](6.08,-0.009576389)(8.5,2.8904235)
\psarc[linewidth=0.04,linecolor=red,arrowsize=0.05291667cm 2.0,arrowlength=1.4,arrowinset=0.4]{->}(7.68,1.3504237){0.4}{318.36646}{114.77514}
\usefont{T1}{ptm}{m}{n}
\rput{-0.047520056}(-0.0026591625,0.0095389495){\rput(11.469952,3.2267501){$F_0$}}
\usefont{T1}{ptm}{m}{n}
\rput{-0.047520056}(-0.0026602356,0.00695128){\rput(8.349952,3.2267501){$F_1$}}
\rput{-186.56827}(18.900784,4.434998){\psarc[linewidth=0.02](9.323148,2.75978){0.21}{336.80142}{93.81407}}
\usefont{T1}{ptm}{m}{n}
\rput{-1.4418001}(-0.062413234,0.22125372){\rput(8.730744,2.6015205){$\phi$}}
\end{pspicture} 
}

}
  \caption{$CF_0 > CF > CF_1$, hence $CF_0 - CF_1 > 0$ and $\phi = tan^{-1} \dfrac{CL}{CF}$}
  \label{fig:range_diff_positive}
\end{figure}

\begin{figure}[H]\centering
  \scalebox{1}{% Generated with LaTeXDraw 2.0.8
% Thu Nov 17 16:53:41 CET 2016
% \usepackage[usenames,dvipsnames]{pstricks}
% \usepackage{epsfig}
% \usepackage{pst-grad} % For gradients
% \usepackage{pst-plot} % For axes
\scalebox{1} % Change this value to rescale the drawing.
{
\begin{pspicture}(0,-3.5298767)(14.32,3.5295343)
\psline[linewidth=0.04cm](0.0,2.8507018)(7.7104354,2.870414)
\psline[linewidth=0.04cm](1.9231296,-2.8799853)(9.633565,-2.860273)
\psdots[dotsize=0.12,dotangle=-56.493206](6.1314507,0.0021031746)
\psline[linewidth=0.012cm,linestyle=dashed,dash=0.16cm 0.16cm](1.5828435,-0.011082077)(9.293279,0.008630338)
\psline[linewidth=0.04cm,arrowsize=0.05291667cm 2.0,arrowlength=1.4,arrowinset=0.4]{->}(6.114774,-0.008937542)(10.598721,-2.868907)
\rput{-56.493206}(3.0479074,5.436033){\psarc[linewidth=0.02](6.5831475,-0.118602596){0.21}{336.80142}{93.81407}}
\usefont{T1}{ptm}{m}{n}
\rput{-1.4418001}(0.010620294,0.18724689){\rput(7.4159327,-0.31761354){$\phi$}}
\psline[linewidth=0.04cm,arrowsize=0.05291667cm 2.0,arrowlength=1.4,arrowinset=0.4]{->}(6.1314507,0.0021031746)(8.176452,2.8911116)
\psline[linewidth=0.04cm,arrowsize=0.05291667cm 2.0,arrowlength=1.4,arrowinset=0.4]{->}(6.1314507,0.0021031746)(4.15856,-2.887136)
\usefont{T1}{ptm}{m}{n}
\rput{1.3045566}(0.0067754984,-0.12859794){\rput(5.6211314,0.23295377){$C$}}
\usefont{T1}{ptm}{m}{n}
\rput{0.040597137}(0.0023587039,-0.0054252143){\rput(7.627923,3.3258739){$L$}}
\usefont{T1}{ptm}{m}{n}
\rput{-1.0306425}(0.06039431,0.073566884){\rput(4.089839,-3.3208966){$R$}}
\psline[linewidth=0.04cm](9.12879,-2.8586583)(14.3,-2.8504658)
\usefont{T1}{ptm}{m}{n}
\rput{-0.047520056}(0.0027003074,0.008927509){\rput(10.735407,-3.2516575){$F$}}
\psdots[dotsize=0.012,dotangle=-56.493206](6.2448254,-2.7052107)
\psdots[dotsize=0.012,dotangle=-56.493206](6.079146,2.7258625)
\psline[linewidth=0.04cm](6.8728247,2.8675463)(8.925276,2.8831694)
\psline[linewidth=0.04cm](6.3306446,-0.15385087)(6.196772,-0.3863979)
\psline[linewidth=0.04cm](6.2134485,-0.37535715)(5.9531846,-0.2358489)
\psdots[dotsize=0.034,dotangle=-56.493206](6.1641116,-0.19214736)
\psline[linewidth=0.04cm,linecolor=red](6.14,0.009534282)(12.7,-2.8504658)
\psline[linewidth=0.04cm,linecolor=red](6.12,-0.010465718)(8.96,-2.8704658)
\usefont{T1}{ptm}{m}{n}
\rput{-0.047520056}(0.0026965104,0.007583913){\rput(9.115407,-3.2316575){$F_0$}}
\usefont{T1}{ptm}{m}{n}
\rput{-0.047520056}(0.0026811955,0.010652631){\rput(12.815407,-3.2116575){$F_1$}}
\psarc[linewidth=0.04,linecolor=red,arrowsize=0.05291667cm 2.0,arrowlength=1.4,arrowinset=0.4]{->}(7.6,-0.9904657){0.34}{225.0}{43.66778}
\usefont{T1}{ptm}{m}{n}
\rput{-1.4418001}(0.06707404,0.23384766){\rput(9.295933,-2.5376136){$\phi$}}
\rput{-224.84753}(15.240408,-11.646244){\psarc[linewidth=0.02](10.023148,-2.6786027){0.21}{336.80142}{93.81407}}
\end{pspicture} 
}

}
  \caption{$CF_0 < CF < CF_1$, hence $CF_0 - CF_1 < 0$ and $\phi = -tan^{-1} \dfrac{CR}{CF}$}
  \label{fig:range_diff_negative}
\end{figure}












In the second solution we assume that the the pose of the vehicle at time $t$ as
$(x_c, y_c, v_c, \psi_v)$ and that \textit{two} range scans at $+90^\circ$
and $-90^\circ$ with respect to the longitudinal axis of the vehicle are
available, which are denoted as $CL$ and $CR$ respectively. Here, again we
distinguish two cases, one where the vehicle is facing the right
lane boundary and one where it is facing the left one. It is not obvious
in this configuration where the vehicle is heading: the only available
information so far is only the two ranges $CR$ and $CL$.

We retrieve the \text{first} minimum distance from the range scan available at
time $t$, and denote the point which corresponds to this distance with $M_0$. The
angle between point $M_0$ and the longitudinal axis of the vehicle is denoted with
$\alpha$. In order to find this angle, we can exploit the fact that each ray
of the scan is separated from the next by $res$ degrees, while all of them
are stored in an array sequentially, in an anti-clockwise manner. In the case
of the \texttt{HOKUYO-UST-10LX-LASER}, the angular resolution is
$res=0.25^{\circ}$, and the starting angle is $-135^{\circ}$ with respect to the
longitudinal axis of the vehicle. This means that we can retrieve the angle
$\alpha$ by first calculating the number of indices between the one that
corresponds to the ray with the minimum range and the one that corresponds
to $+135^{\circ}$ (which in our case is $135 / 0.25 = 540$) with regard to the
laser's system of reference, and then by multiplying this number $\Delta i$ by
the angular resolution $res$. Hence, $\alpha = 0.25 \times \Delta i$.


\begin{figure}[H]\centering
  \scalebox{1}{% Generated with LaTeXDraw 2.0.8
% Mon Nov 28 04:09:02 CET 2016
% \usepackage[usenames,dvipsnames]{pstricks}
% \usepackage{epsfig}
% \usepackage{pst-grad} % For gradients
% \usepackage{pst-plot} % For axes
\scalebox{1} % Change this value to rescale the drawing.
{
\begin{pspicture}(0,-3.1192188)(16.715,3.1192188)
\definecolor{color8}{rgb}{0.00392156862745098,0.00392156862745098,0.00392156862745098}
\psline[linewidth=0.04cm](0.18,2.645781)(16.56,2.645781)
\psline[linewidth=0.04cm](0.0,-2.5342188)(15.86,-2.5342188)
\psline[linewidth=0.03cm,linestyle=dashed,dash=0.16cm 0.16cm](0.3,0.06578115)(16.7,0.0657812)
\psline[linewidth=0.04cm,linecolor=color8,arrowsize=0.05291667cm 2.0,arrowlength=1.4,arrowinset=0.4]{->}(2.192057,0.06578115)(6.66,-1.4742188)
\psline[linewidth=0.04cm,linecolor=color8](1.26,-2.5142188)(3.14,2.645781)
\usefont{T1}{ptm}{m}{n}
\rput(2.0017185,2.855781){$L^{-}$}
\usefont{T1}{ptm}{m}{n}
\rput(0.7917187,-2.864219){$R^{-}$}
\usefont{T1}{ptm}{m}{n}
\rput(1.6917188,0.19578125){$C^{-}$}
\usefont{T1}{ptm}{m}{n}
\rput(1.9617187,-2.884219){$M_0$}
\psline[linewidth=0.04cm,linecolor=color8,arrowsize=0.05291667cm 2.0,arrowlength=1.4,arrowinset=0.4]{->}(10.93792,0.05824818)(13.34,2.2457812)
\psline[linewidth=0.04cm,linecolor=color8](12.92,-2.5142188)(9.02,2.645781)
\usefont{T1}{ptm}{m}{n}
\rput(10.821719,-2.864219){$M_0$}
\usefont{T1}{ptm}{m}{n}
\rput(9.381719,2.9157813){$L^{+}$}
\usefont{T1}{ptm}{m}{n}
\rput(13.091719,-2.8442187){$R^{+}$}
\usefont{T1}{ptm}{m}{n}
\rput(10.371718,-0.24421875){$C^{+}$}
\psarc[linewidth=0.02](11.41,-0.06421885){0.95}{240.70863}{67.036224}
\usefont{T1}{ptm}{m}{n}
\rput(11.691717,0.35578135){$\phi^{+}$}
\psarc[linewidth=0.02](11.65,0.31578115){0.29}{300.96375}{117.474434}
\psline[linewidth=0.04cm](2.54,-0.03421885)(2.4,-0.39421886)
\psline[linewidth=0.04cm](2.4,-0.39421886)(2.06,-0.25421885)
\psdots[dotsize=0.06](2.3,-0.13421886)
\psline[linewidth=0.04cm](11.42,0.06578115)(11.14,-0.17421885)
\psline[linewidth=0.04cm](11.42,0.04578115)(11.22,0.30578116)
\psdots[dotsize=0.06](11.18,0.08578115)
\usefont{T1}{ptm}{m}{n}
\rput(4.5517178,-0.30421886){$\phi^{-}$}
\usefont{T1}{ptm}{m}{n}
\rput(3.1135936,-1.0042188){$\alpha$}
\usefont{T1}{ptm}{m}{n}
\rput(12.573593,-0.82421887){$\alpha$}
\psarc[linewidth=0.02](3.77,-0.20421885){0.29}{273.8141}{62.81889}
\psline[linewidth=0.04cm,linestyle=dashed,dash=0.16cm 0.16cm](2.22,0.06578115)(2.22,-2.5342188)
\psline[linewidth=0.04cm,linestyle=dashed,dash=0.16cm 0.16cm](10.94,0.08578115)(10.94,-2.5142188)
\rput{-146.35826}(4.9684525,0.6336013){\psarc[linewidth=0.02](2.58,-0.43421885){0.42}{0.0}{180.0}}
\end{pspicture} 
}

}
  \caption{}
  \label{}
\end{figure}

At this point we do not know whether the vehicle is pointing to the
left or to the right lane boundary yet. However, we can determine the sign and the
magnitude of the orientation of the vehicle with respect to the orientation of
the lane by examining the sign of the difference $\alpha - 90^{\circ}$:
if it is negative, the vehicle is pointing towards the right boundary lane,
if it is positive, towards the left. Furthermore, we can now
calculate the magnitude of the orientation of the vehicle as the difference
$|\alpha - 90^{\circ}|$, as shown in figure \ref{fig:circular_mpc_rotation_car_at_right_side}.
In conclusion, when the car is located at the centerline, its orientation with
respect to the orientation of the lane is

$$\phi = \alpha - 90^{\circ}$$


therefore the overall angular error of the vehicle in this case is

\begin{align}
  e &= -\text{tan}^{-1}\dfrac{CL-CR}{2CC'} + \alpha - 90
\end{align}



Since the input to the plant is in terms of angular displacement, this
is in fact the error that the PID controller can include and utilize in
order to determine the plant's input.

The angular input to the vehicle will then be

\begin{align}
  \delta = K_p \cdot e + K_d \cdot \dfrac{de}{dt} + K_i \cdot \int e dt
\end{align}
