%We are not able to adequately report on the performance of this functionality;
%we had successfully configured this module, but an undetermined failure
%seems to have occurred in the time between this and when we returned to
%record and log it. The vehicle might have suffered some mechanical
%mis-calibration due to crashes it has had in the meantime. However,
%this does not explain why it is able to follow the centerline with a
%PID controller, since they only difference between these two functionalities
%lies in code.

%As evidence of the successful completion of this task, we can exhibit a
%video which illustrates the performance of our approach prior to this failure.

%\begin{center}
  %\url{https://www.youtube.com/watch?v=w3Wnw5SLmss}
%\end{center}

We tested the performance of this functionality in the corridor that
connects the buildings addressed as Osquldas vag 10 and 6.

The velocity of the vehicle was again kept at its minimum. The vehicle was
positioned approximately one meter away from the centerline at the side
of the negative $y$ axis. Its initial orientation with respect to the
orientation of the centerline was $-65^{\circ}$, meaning that it begun
facing the right-hand side wall.

With penalty matrices

\begin{equation}
  Q =
  \begin{bmatrix}
    1000 & 0    \\
      0  & 1200 \\
  \end{bmatrix}, R = 10000
\end{equation}

and a horizon length of $N=10$, we plot the two main states of significance in
figures \ref{fig:centerline_mpc_y} and \ref{fig:centerline_mpc_psi}:
the displacement of the vehicle from the centerline and its orientation with
respect to the orientation of the corridor itself.
Figure \ref{fig:centerline_mpc_input} shows the steering angle that was given
as input to the front wheels.


\noindent\makebox[\linewidth][c]{%
\begin{minipage}{\linewidth}
  \begin{minipage}{0.45\linewidth}
    \begin{figure}[H]
      \scalebox{0.6}{% This file was created by matlab2tikz.
%
%The latest updates can be retrieved from
%  http://www.mathworks.com/matlabcentral/fileexchange/22022-matlab2tikz-matlab2tikz
%where you can also make suggestions and rate matlab2tikz.
%
\definecolor{mycolor1}{rgb}{0.00000,0.44700,0.74100}%
%
\begin{tikzpicture}

\begin{axis}[%
width=4.133in,
height=3.26in,
at={(0.693in,0.44in)},
scale only axis,
xmin=0,
xmax=500,
xmajorgrids,
ymin=-1,
ymax=1,
ymajorgrids,
axis background/.style={fill=white}
]
\addplot [color=mycolor1,solid,forget plot]
  table[row sep=crcr]{%
1	-0.862078011035919\\
2	-0.860797464847565\\
3	-0.861142933368683\\
4	-0.859686732292175\\
5	-0.861081063747406\\
6	-0.863374948501587\\
7	-0.863067924976349\\
8	-0.858868420124054\\
9	-0.861062467098236\\
10	-0.859835684299469\\
11	-0.86039924621582\\
12	-0.862242043018341\\
13	-0.862404465675354\\
14	-0.860375583171844\\
15	-0.860456109046936\\
16	-0.860243678092957\\
17	-0.862569868564606\\
18	-0.863066136837006\\
19	-0.860088467597961\\
20	-0.86155366897583\\
21	-0.861345052719116\\
22	-0.860773861408234\\
23	-0.862078011035919\\
24	-0.863260209560394\\
25	-0.860044658184052\\
26	-0.860951840877533\\
27	-0.859623372554779\\
28	-0.859057128429413\\
29	-0.861367344856262\\
30	-0.861983418464661\\
31	-0.861249268054962\\
32	-0.861865222454071\\
33	-0.860588371753693\\
34	-0.860986530780792\\
35	-0.861132204532623\\
36	-0.862829804420471\\
37	-0.8614382147789\\
38	-0.862881898880005\\
39	-0.861719131469727\\
40	-0.861364603042603\\
41	-0.86230993270874\\
42	-0.861329555511475\\
43	-0.861957848072052\\
44	-0.8598592877388\\
45	-0.862617135047913\\
46	-0.860044658184052\\
47	-0.86111706495285\\
48	-0.860047996044159\\
49	-0.862900733947754\\
50	-0.860537528991699\\
51	-0.859788477420807\\
52	-0.860802948474884\\
53	-0.861510515213013\\
54	-0.859694123268127\\
55	-0.862574517726898\\
56	-0.860919415950775\\
57	-0.860895752906799\\
58	-0.862172603607178\\
59	-0.860614240169525\\
60	-0.861627161502838\\
61	-0.858892500400543\\
62	-0.861719131469727\\
63	-0.860821068286896\\
64	-0.861923575401306\\
65	-0.861752152442932\\
66	-0.860731542110443\\
67	-0.861015260219574\\
68	-0.859387457370758\\
69	-0.858732879161835\\
70	-0.862595558166504\\
71	-0.861624598503113\\
72	-0.858632504940033\\
73	-0.860449075698853\\
74	-0.863160669803619\\
75	-0.860378265380859\\
76	-0.86158150434494\\
77	-0.857831478118896\\
78	-0.861955404281616\\
79	-0.861886978149414\\
80	-0.861199200153351\\
81	-0.86076295375824\\
82	-0.863351345062256\\
83	-0.861603975296021\\
84	-0.861062467098236\\
85	-0.86169970035553\\
86	-0.862097203731537\\
87	-0.862002730369568\\
88	-0.86224353313446\\
89	-0.862959206104279\\
90	-0.861860930919647\\
91	-0.861676037311554\\
92	-0.86328387260437\\
93	-0.862007081508636\\
94	-0.859929919242859\\
95	-0.862191736698151\\
96	-0.861888825893402\\
97	-0.861497759819031\\
98	-0.860220074653625\\
99	-0.859010398387909\\
100	-0.860635161399841\\
101	-0.862101674079895\\
102	-0.8592888712883\\
103	-0.861154794692993\\
104	-0.862226545810699\\
105	-0.861246466636658\\
106	-0.862026333808899\\
107	-0.862196207046509\\
108	-0.860826551914215\\
109	-0.861461818218231\\
110	-0.860526919364929\\
111	-0.86041933298111\\
112	-0.861837267875671\\
113	-0.858805358409882\\
114	-0.861368656158447\\
115	-0.860425472259521\\
116	-0.86280620098114\\
117	-0.860590636730194\\
118	-0.862136483192444\\
119	-0.862834632396698\\
120	-0.861994564533234\\
121	-0.864064157009125\\
122	-0.86210161447525\\
123	-0.860354721546173\\
124	-0.85974794626236\\
125	-0.861187875270844\\
126	-0.862633287906647\\
127	-0.858844816684723\\
128	-0.862810969352722\\
129	-0.861151933670044\\
130	-0.858915567398071\\
131	-0.859198689460754\\
132	-0.862041831016541\\
133	-0.862653017044067\\
134	-0.865849733352661\\
135	-0.870072841644287\\
136	-0.875799775123596\\
137	-0.882054805755615\\
138	-0.885055661201477\\
139	-0.883396506309509\\
140	-0.871016621589661\\
141	-0.883034586906433\\
142	-0.858506560325623\\
143	-0.861996710300446\\
144	-0.874751448631287\\
145	-0.878074288368225\\
146	-0.853605926036835\\
147	-0.870456457138062\\
148	-0.883222460746765\\
149	-0.889798283576965\\
150	-0.870624423027039\\
151	-0.899546444416046\\
152	-0.905179023742676\\
153	-0.906292319297791\\
154	-0.913517653942108\\
155	-0.90942108631134\\
156	-0.931705415248871\\
157	-0.930896520614624\\
158	-0.933786332607269\\
159	-0.933316171169281\\
160	-0.93765777349472\\
161	-0.931691110134125\\
162	-0.924617409706116\\
163	-0.936087012290955\\
164	-0.932690382003784\\
165	-0.924729824066162\\
166	-0.929259896278381\\
167	-0.928460240364075\\
168	-0.928236186504364\\
169	-0.915175676345825\\
170	-0.888369083404541\\
171	-0.912677824497223\\
172	-0.899824202060699\\
173	-0.894942939281464\\
174	-0.893094778060913\\
175	-0.870949983596802\\
176	-0.85504412651062\\
177	-0.847206771373749\\
178	-0.855317950248718\\
179	-0.815333724021912\\
180	-0.817369520664215\\
181	-0.812255024909973\\
182	-0.75690895318985\\
183	-0.754757046699524\\
184	-0.727650821208954\\
185	-0.761677026748657\\
186	-0.717706501483917\\
187	-0.700630366802216\\
188	-0.659670829772949\\
189	-0.649322390556335\\
190	-0.649994254112244\\
191	-0.59495884180069\\
192	-0.596188247203827\\
193	-0.566737592220306\\
194	-0.574458360671997\\
195	-0.504159271717072\\
196	-0.480830550193787\\
197	-0.468517959117889\\
198	-0.446124404668808\\
199	-0.406902611255646\\
200	-0.401513636112213\\
201	-0.374929040670395\\
202	-0.356805205345154\\
203	-0.323664844036102\\
204	-0.303404748439789\\
205	-0.290667027235031\\
206	-0.272063374519348\\
207	-0.242067262530327\\
208	-0.224982082843781\\
209	-0.209829688072205\\
210	-0.192677736282349\\
211	-0.172327131032944\\
212	-0.157971858978271\\
213	-0.144060134887695\\
214	-0.132943108677864\\
215	-0.107119865715504\\
216	-0.0976131930947304\\
217	-0.0860113948583603\\
218	-0.0779053419828415\\
219	-0.0634319335222244\\
220	-0.0535665862262249\\
221	-0.0478917472064495\\
222	-0.0385315231978893\\
223	-0.025116603821516\\
224	-0.0197704546153545\\
225	-0.0131843192502856\\
226	-0.00691169686615467\\
227	0.00304749212227762\\
228	0.00493159564211965\\
229	0.00949917268007994\\
230	0.0132845863699913\\
231	0.0138274738565087\\
232	0.0168808083981276\\
233	0.0195905771106482\\
234	0.0193569604307413\\
235	0.0204759985208511\\
236	0.0199424084275961\\
237	0.018379382789135\\
238	0.017774935811758\\
239	0.0139966700226068\\
240	0.0111672403290868\\
241	0.00897225085645914\\
242	0.00637790048494935\\
243	-0.00044901022920385\\
244	-0.0028982653748244\\
245	-0.00927046220749617\\
246	-0.01327293086797\\
247	-0.0190499052405357\\
248	-0.0245789010077715\\
249	-0.0288896411657333\\
250	-0.0328729972243309\\
251	-0.0418401770293713\\
252	-0.0476556904613972\\
253	-0.0523671433329582\\
254	-0.0574651621282101\\
255	-0.066034123301506\\
256	-0.0680229812860489\\
257	-0.0758848041296005\\
258	-0.080058291554451\\
259	-0.0876206234097481\\
260	-0.0933685600757599\\
261	-0.0976997539401054\\
262	-0.100522391498089\\
263	-0.107038594782352\\
264	-0.112223982810974\\
265	-0.118649028241634\\
266	-0.122220434248447\\
267	-0.131977885961533\\
268	-0.134871765971184\\
269	-0.139047369360924\\
270	-0.145768493413925\\
271	-0.154432415962219\\
272	-0.156876936554909\\
273	-0.16106840968132\\
274	-0.167684003710747\\
275	-0.175761297345161\\
276	-0.178975090384483\\
277	-0.182063043117523\\
278	-0.186035439372063\\
279	-0.194683119654655\\
280	-0.198087707161903\\
281	-0.202547013759613\\
282	-0.205070227384567\\
283	-0.207118734717369\\
284	-0.211801677942276\\
285	-0.213820099830627\\
286	-0.214396253228188\\
287	-0.22287081182003\\
288	-0.220557928085327\\
289	-0.222315117716789\\
290	-0.223732128739357\\
291	-0.226956740021706\\
292	-0.223399072885513\\
293	-0.225837782025337\\
294	-0.228255093097687\\
295	-0.226694867014885\\
296	-0.226850762963295\\
297	-0.225507169961929\\
298	-0.22591795027256\\
299	-0.223537370562553\\
300	-0.224429041147232\\
301	-0.22244755923748\\
302	-0.222372487187386\\
303	-0.219580441713333\\
304	-0.220698803663254\\
305	-0.217990607023239\\
306	-0.213513195514679\\
307	-0.213187336921692\\
308	-0.207880958914757\\
309	-0.206473618745804\\
310	-0.203220248222351\\
311	-0.199807614088058\\
312	-0.196048572659492\\
313	-0.193784534931183\\
314	-0.195113122463226\\
315	-0.156602174043655\\
316	-0.155849054455757\\
317	-0.154262989759445\\
318	-0.149831160902977\\
319	-0.141150116920471\\
320	-0.145987495779991\\
321	-0.143817096948624\\
322	-0.140101298689842\\
323	-0.136927261948586\\
324	-0.133893251419067\\
325	-0.132359385490417\\
326	-0.132479816675186\\
327	-0.126208052039146\\
328	-0.122833341360092\\
329	-0.119642838835716\\
330	-0.119879812002182\\
331	-0.114444918930531\\
332	-0.112666673958302\\
333	-0.110309518873692\\
334	-0.106566421687603\\
335	-0.10415080934763\\
336	-0.09898491948843\\
337	-0.101032227277756\\
338	-0.0967106148600578\\
339	-0.0948426872491837\\
340	-0.0951347053050995\\
341	-0.093622975051403\\
342	-0.092057392001152\\
343	-0.0906052440404892\\
344	-0.0885229632258415\\
345	-0.0896632522344589\\
346	-0.0884277299046516\\
347	-0.0847469940781593\\
348	-0.0858458280563354\\
349	-0.0835126042366028\\
350	-0.0842643529176712\\
351	-0.102452382445335\\
352	-0.100734256207943\\
353	-0.0906811654567719\\
354	-0.0857225731015205\\
355	-0.0800026133656502\\
356	-0.0820000022649765\\
357	-0.082975409924984\\
358	-0.0816405192017555\\
359	-0.0780446380376816\\
360	-0.0789999961853027\\
361	-0.0793503373861313\\
362	-0.0770464986562729\\
363	-0.0757849887013435\\
364	-0.0732983574271202\\
365	-0.074088878929615\\
366	-0.0707298293709755\\
367	-0.0674421042203903\\
368	-0.0655790790915489\\
369	-0.0643179416656494\\
370	-0.0642319321632385\\
371	-0.0610383152961731\\
372	-0.059027373790741\\
373	-0.0511684231460094\\
374	-0.0469258092343807\\
375	-0.044733889400959\\
376	-0.0431565158069134\\
377	-0.0423317216336727\\
378	-0.0412853211164474\\
379	-0.0354469791054726\\
380	-0.0334979668259621\\
381	-0.0305919703096151\\
382	-0.0318422988057137\\
383	-0.0247528683394194\\
384	-0.021734207868576\\
385	-0.0206782314926386\\
386	-0.0209531001746655\\
387	-0.0250561460852623\\
388	-0.0388091579079628\\
389	-0.0416888669133186\\
390	-0.0295248366892338\\
391	-0.0197907593101263\\
392	-0.0196407306939363\\
393	-0.0177530292421579\\
394	-0.0192355066537857\\
395	-0.0216089561581612\\
396	-0.0199430715292692\\
397	-0.0204908028244972\\
398	-0.0199931506067514\\
399	-0.0184936616569757\\
400	-0.0179515462368727\\
401	-0.017356475815177\\
402	-0.0113333323970437\\
403	-0.00687848636880517\\
404	-0.00335662951692939\\
405	-0.00518449721857905\\
406	-0.000927078770473599\\
407	-0.00130950729362667\\
408	0.00209396588616073\\
409	-0.00209474936127663\\
410	0.00380589719861746\\
411	0.00330335786566138\\
412	0.00183317961636931\\
413	-0.00183123105671257\\
414	-0.00175814086105675\\
415	-0.00195118633564562\\
416	-0.00152054720092565\\
417	-0.00275790691375732\\
418	-0.00468964735046029\\
419	-0.00388512434437871\\
420	-0.00390705512836576\\
421	-0.0122838467359543\\
422	-0.0190459880977869\\
423	-0.0184841435402632\\
424	-0.018348578363657\\
425	-0.0196070913225412\\
426	-0.0199512653052807\\
427	-0.0239467341452837\\
428	-0.0253987107425928\\
429	-0.0275214575231075\\
430	-0.0300931818783283\\
431	-0.0349810719490051\\
432	-0.036287397146225\\
433	-0.0395949073135853\\
434	-0.0401294454932213\\
435	-0.0419031605124474\\
436	-0.0425710119307041\\
437	-0.0423805452883244\\
438	-0.0426081903278828\\
439	-0.0388423129916191\\
440	-0.039986290037632\\
441	-0.0379530526697636\\
442	-0.0362615585327148\\
443	-0.0347499921917915\\
444	-0.032094020396471\\
445	-0.0296088326722383\\
446	-0.0289693009108305\\
447	-0.0233166553080082\\
448	-0.0221420042216778\\
449	-0.0183802433311939\\
450	-0.0188079066574574\\
451	-0.00780884502455592\\
452	-0.00402285857126117\\
453	-0.000831041019409895\\
454	0.000474883418064564\\
455	0.00828445050865412\\
456	0.0132131669670343\\
457	0.0152375120669603\\
458	0.0206808298826218\\
459	0.0267888493835926\\
460	0.0301118846982718\\
461	0.0317616127431393\\
462	0.0329971760511398\\
463	0.0378686748445034\\
464	0.0391573421657085\\
465	0.0415520332753658\\
466	0.0433158688247204\\
467	0.0466357432305813\\
468	0.0485761985182762\\
469	0.0487838611006737\\
470	0.047574307769537\\
471	0.0506192371249199\\
472	0.0488000698387623\\
473	0.0498334802687168\\
474	0.0502255745232105\\
475	0.0471233129501343\\
476	0.0468707680702209\\
477	0.0457330010831356\\
478	0.0453221760690212\\
479	0.0400528460741043\\
480	0.0388315916061401\\
481	0.0311536770313978\\
482	0.0335978381335735\\
483	0.0283027980476618\\
484	0.0253806989639997\\
485	0.0233551356941462\\
486	0.0234665051102638\\
487	0.0202623475342989\\
488	0.021880941465497\\
489	0.0190189797431231\\
490	0.0213882550597191\\
491	0.0214155148714781\\
492	0.0137378228828311\\
493	0.0150595949962735\\
494	0.0165836196392775\\
495	0.0208340287208557\\
496	0.0231604296714067\\
497	0.0266055017709732\\
498	0.0104469517245889\\
499	0.0142057575285435\\
500	0.017076825723052\\
501	0.0185363404452801\\
502	0.01999781280756\\
503	0.0244346000254154\\
504	0.028719836845994\\
505	0.0321998223662376\\
506	0.0357954017817974\\
507	0.045319814234972\\
508	0.0511339269578457\\
509	0.0579211190342903\\
510	0.0659339800477028\\
511	0.0937526896595955\\
512	0.112374112010002\\
513	0.109303675591946\\
514	0.128212988376617\\
515	0.159077987074852\\
516	6.72835159301758\\
517	7.55607604980469\\
518	0.175895988941193\\
519	0.215987890958786\\
520	0.188034370541573\\
521	0.152021437883377\\
522	0.158512204885483\\
523	0.124102115631104\\
524	0.138314142823219\\
525	0.153900638222694\\
526	0.177191987633705\\
527	0.285698384046555\\
528	0.348431885242462\\
529	0.384509235620499\\
530	0.420571357011795\\
531	0.497148483991623\\
532	0.47269207239151\\
533	0.441500961780548\\
534	0.404626458883286\\
535	0.337435394525528\\
536	0.32334840297699\\
537	0.354487866163254\\
538	0.358406811952591\\
539	0.372172594070435\\
540	0.382442712783813\\
541	0.403259366750717\\
542	0.79350346326828\\
543	0.451451569795609\\
544	0.399523884057999\\
545	0.379601299762726\\
546	0.367797613143921\\
547	0.33272185921669\\
548	0.459157794713974\\
549	0.46239098906517\\
550	0.463839411735535\\
551	0.468568474054337\\
552	0.47069463133812\\
553	0.470968514680862\\
554	0.471686780452728\\
555	0.474527448415756\\
556	0.412097305059433\\
557	0.473237991333008\\
558	0.473504453897476\\
559	-0.128023937344551\\
560	-0.172533541917801\\
561	-0.242663383483887\\
562	-0.168670624494553\\
563	-0.156932562589645\\
564	-0.229706302285194\\
565	-0.223714917898178\\
};
\end{axis}
\end{tikzpicture}%}
      \caption{The displacement of the vehicle from the centerline in meters
        through time.}
      \label{fig:centerline_mpc_y}
    \end{figure}
  \end{minipage}
  \hfill
  \begin{minipage}{0.45\linewidth}
    \begin{figure}[H]
      \scalebox{0.6}{% This file was created by matlab2tikz.
%
%The latest updates can be retrieved from
%  http://www.mathworks.com/matlabcentral/fileexchange/22022-matlab2tikz-matlab2tikz
%where you can also make suggestions and rate matlab2tikz.
%
\definecolor{mycolor1}{rgb}{0.00000,0.44700,0.74100}%
%
\begin{tikzpicture}

\begin{axis}[%
width=4.133in,
height=3.26in,
at={(0.693in,0.44in)},
scale only axis,
xmin=200,
xmax=505,
xmajorgrids,
ymin=-180,
ymax=50,
ymajorgrids,
axis background/.style={fill=white}
]
\addplot [color=mycolor1,solid,forget plot]
  table[row sep=crcr]{%
1	-65.0000006704249\\
2	-65.0000006704249\\
3	-65.4999978384265\\
4	-64.4999966722341\\
5	-65.2499992544257\\
6	-64.7500020864241\\
7	-64.4999966722341\\
8	-65.0000006704249\\
9	-65.0000006704249\\
10	-64.4999966722341\\
11	-64.4999966722341\\
12	-64.7500020864241\\
13	-65.2499992544257\\
14	-64.4999966722341\\
15	-65.2499992544257\\
16	-65.0000006704249\\
17	-65.2499992544257\\
18	-64.7500020864241\\
19	-64.7500020864241\\
20	-64.7500020864241\\
21	-65.0000006704249\\
22	-65.2499992544257\\
23	-65.0000006704249\\
24	-72.0000020035825\\
25	-64.4999966722341\\
26	-65.0000006704249\\
27	-76.4999970061646\\
28	-65.0000006704249\\
29	-65.4999978384265\\
30	-65.0000006704249\\
31	-65.0000006704249\\
32	-64.7500020864241\\
33	-65.0000006704249\\
34	-163.500002842255\\
35	-65.0000006704249\\
36	-64.7500020864241\\
37	-64.7500020864241\\
38	-64.4999966722341\\
39	-64.7500020864241\\
40	-65.0000006704249\\
41	-65.2499992544257\\
42	-65.0000006704249\\
43	-64.7500020864241\\
44	-65.0000006704249\\
45	-65.0000006704249\\
46	-64.7500020864241\\
47	-65.0000006704249\\
48	-65.0000006704249\\
49	-64.7500020864241\\
50	-65.2499992544257\\
51	-65.2499992544257\\
52	-65.2499992544257\\
53	-65.0000006704249\\
54	33.2499995023095\\
55	-65.0000006704249\\
56	-65.0000006704249\\
57	-65.0000006704249\\
58	-65.2499992544257\\
59	-64.4999966722341\\
60	-61.9999971718477\\
61	-65.0000006704249\\
62	-65.4999978384265\\
63	-82.0000000051282\\
64	-65.0000006704249\\
65	-64.4999966722341\\
66	-64.4999966722341\\
67	-65.4999978384265\\
68	-65.2499992544257\\
69	-64.7500020864241\\
70	-65.0000006704249\\
71	-72.4999991715842\\
72	-64.7500020864241\\
73	-64.4999966722341\\
74	-65.0000006704249\\
75	-65.2499992544257\\
76	-64.7500020864241\\
77	-66.4999990046189\\
78	-64.7500020864241\\
79	-65.4999978384265\\
80	-65.2499992544257\\
81	-64.4999966722341\\
82	-65.0000006704249\\
83	-64.7500020864241\\
84	-65.0000006704249\\
85	-64.7500020864241\\
86	-65.0000006704249\\
87	-75.9999998381629\\
88	-64.4999966722341\\
89	-64.4999966722341\\
90	-65.0000006704249\\
91	-65.2499992544257\\
92	-65.0000006704249\\
93	25.5000007096021\\
94	-64.4999966722341\\
95	-64.7500020864241\\
96	-65.0000006704249\\
97	-64.7500020864241\\
98	-65.0000006704249\\
99	-63.7500009202317\\
100	-65.0000006704249\\
101	-64.7500020864241\\
102	-65.0000006704249\\
103	-64.7500020864241\\
104	-65.0000006704249\\
105	-65.0000006704249\\
106	-65.0000006704249\\
107	-64.7500020864241\\
108	-64.7500020864241\\
109	-64.4999966722341\\
110	-64.7500020864241\\
111	-64.7500020864241\\
112	-65.0000006704249\\
113	-64.4999966722341\\
114	-65.0000006704249\\
115	-64.7500020864241\\
116	-64.7500020864241\\
117	-64.7500020864241\\
118	-65.2499992544257\\
119	-65.0000006704249\\
120	-64.7500020864241\\
121	-64.7500020864241\\
122	-65.2499992544257\\
123	-64.7500020864241\\
124	-65.0000006704249\\
125	-65.0000006704249\\
126	27.7499999184405\\
127	-64.7500020864241\\
128	-64.4999966722341\\
129	30.2499994188269\\
130	-65.4999978384265\\
131	-64.7500020864241\\
132	-65.0000006704249\\
133	-65.2499992544257\\
134	-64.7500020864241\\
135	-65.0000006704249\\
136	-65.0000006704249\\
137	-64.4999966722341\\
138	-65.0000006704249\\
139	-65.0000006704249\\
140	-65.0000006704249\\
141	-64.7500020864241\\
142	-65.0000006704249\\
143	-65.4999978384265\\
144	-65.2499992544257\\
145	-64.7500020864241\\
146	-65.2499992544257\\
147	-65.2499992544257\\
148	-167.750006091025\\
149	-65.0000006704249\\
150	-65.0000006704249\\
151	-64.7500020864241\\
152	-65.0000006704249\\
153	-65.0000006704249\\
154	-65.2499992544257\\
155	-65.2499992544257\\
156	-65.2499992544257\\
157	-64.4999966722341\\
158	-65.0000006704249\\
159	-64.4999966722341\\
160	-65.2499992544257\\
161	-64.7500020864241\\
162	-65.2499992544257\\
163	-65.0000006704249\\
164	-64.7500020864241\\
165	18.0000005008956\\
166	-64.4999966722341\\
167	-65.0000006704249\\
168	-65.0000006704249\\
169	-65.2499992544257\\
170	-65.4999978384265\\
171	-64.7500020864241\\
172	-65.2499992544257\\
173	-64.7500020864241\\
174	-65.0000006704249\\
175	-65.2499992544257\\
176	-64.7500020864241\\
177	-65.2499992544257\\
178	-65.2499992544257\\
179	-64.7500020864241\\
180	-64.7500020864241\\
181	-65.2499992544257\\
182	-69.9999996711977\\
183	-65.0000006704249\\
184	-65.2499992544257\\
185	-65.0000006704249\\
186	-65.0000006704249\\
187	-65.0000006704249\\
188	-65.0000006704249\\
189	-65.0000006704249\\
190	-65.0000006704249\\
191	-65.2499992544257\\
192	-64.7500020864241\\
193	-65.0000006704249\\
194	-64.7500020864241\\
195	-64.7500020864241\\
196	-64.4999966722341\\
197	-64.7500020864241\\
198	-64.4999966722341\\
199	-65.0000006704249\\
200	-64.7500020864241\\
201	-64.7500020864241\\
202	-64.7500020864241\\
203	-65.2499992544257\\
204	-65.2499992544257\\
205	-65.0000006704249\\
206	-64.4999966722341\\
207	-64.2499980882333\\
208	-64.4999966722341\\
209	-65.0000006704249\\
210	-65.0000006704249\\
211	-65.0000006704249\\
212	-64.7500020864241\\
213	-64.4999966722341\\
214	-63.5000023362309\\
215	-61.9999971718477\\
216	-61.0000028358445\\
217	-59.7500030856513\\
218	-58.7500019194589\\
219	-56.7499995870741\\
220	-161.50000050987\\
221	-54.4999986706885\\
222	-53.2499989204953\\
223	-50.5000008361081\\
224	-48.7500005028187\\
225	-47.0000001695293\\
226	-60.7499974216545\\
227	-59.4999976714613\\
228	-64.2499980882333\\
229	-57.9999993372673\\
230	-53.7499995035915\\
231	-56.7499995870741\\
232	-43.7500015020459\\
233	-51.7500005863013\\
234	-48.7500005028187\\
235	-53.5000009195907\\
236	-47.7499993366263\\
237	-47.0000001695293\\
238	-48.2499999197225\\
239	-39.7500002523709\\
240	-42.9999989198543\\
241	-36.0000010017913\\
242	-39.7500002523709\\
243	-26.2499998766992\\
244	-34.2500006685019\\
245	-35.7499990026959\\
246	-26.5000001682473\\
247	-26.7500004597954\\
248	-28.7500010846329\\
249	-25.5000007096021\\
250	-19.750000834185\\
251	-24.4999995434097\\
252	-20.500000001282\\
253	-14.2500003965424\\
254	-14.7500001258649\\
255	-13.2500000841236\\
256	-10.250000000641\\
257	-0.249999998063408\\
258	-3.50000002624857\\
259	-5.50000001075584\\
260	-0.499999996126817\\
261	1.75000001312428\\
262	3.00000008348261\\
263	10.5000002921891\\
264	13.2500000841236\\
265	2.25000006261195\\
266	9.75000027131847\\
267	7.24999991715841\\
268	12.7500003548011\\
269	9.49999997977037\\
270	6.75000018783586\\
271	14.2500003965424\\
272	6.25000003162649\\
273	12.7500003548011\\
274	11.7500000423823\\
275	12.2499997717049\\
276	14.4999998343168\\
277	16.5000004591543\\
278	11.4999997508342\\
279	19.2500002510888\\
280	23.7500003763127\\
281	16.9999993347032\\
282	15.000000417413\\
283	14.7500001258649\\
284	17.2499996262513\\
285	17.4999999177994\\
286	22.5000006261195\\
287	18.4999993764445\\
288	19.750000834185\\
289	20.500000001282\\
290	18.4999993764445\\
291	18.9999999595407\\
292	24.2499992518617\\
293	22.5000006261195\\
294	21.4999994599271\\
295	20.500000001282\\
296	21.0000005843782\\
297	18.2500007924437\\
298	13.2500000841236\\
299	21.4999994599271\\
300	17.7500002093475\\
301	21.249999168379\\
302	22.0000000430233\\
303	19.5000005426369\\
304	22.2500003345714\\
305	18.2500007924437\\
306	24.7499998349579\\
307	16.9999993347032\\
308	16.9999993347032\\
309	13.2500000841236\\
310	23.5000000847646\\
311	19.5000005426369\\
312	13.7499998134462\\
313	19.5000005426369\\
314	19.9999994181858\\
315	18.4999993764445\\
316	15.9999998760581\\
317	19.750000834185\\
318	16.7500007507024\\
319	18.4999993764445\\
320	17.7500002093475\\
321	15.000000417413\\
322	19.5000005426369\\
323	12.9999997925755\\
324	14.7500001258649\\
325	11.0000000215117\\
326	14.2500003965424\\
327	14.0000001049943\\
328	9.99999970909292\\
329	11.4999997508342\\
330	15.2500007089611\\
331	15.2500007089611\\
332	11.4999997508342\\
333	10.250000000641\\
334	13.7499998134462\\
335	-168.999999011029\\
336	8.74999995889972\\
337	7.99999993802907\\
338	9.24999968822227\\
339	10.250000000641\\
340	2.49999992727323\\
341	4.99999985454646\\
342	7.00000005249714\\
343	4.74999998988518\\
344	1.99999998450727\\
345	0.999999992253633\\
346	-3.24999994814388\\
347	0.249999998063408\\
348	0.249999998063408\\
349	2.75000000537792\\
350	-0.249999998063408\\
351	-4.50000012522391\\
352	-4.74999998988518\\
353	-3.00000008348261\\
354	-3.00000008348261\\
355	-4.99999985454646\\
356	-4.99999985454646\\
357	-8.25000022957717\\
358	-1.5000000417413\\
359	-4.50000012522391\\
360	-6.00000016696521\\
361	-4.50000012522391\\
362	-7.24999991715841\\
363	-4.74999998988518\\
364	-7.74999964648097\\
365	-3.75000010435326\\
366	-5.25000014609456\\
367	-7.74999964648097\\
368	-4.74999998988518\\
369	-4.74999998988518\\
370	-3.50000002624857\\
371	-3.99999996901453\\
372	-4.99999985454646\\
373	-3.75000010435326\\
374	-5.74999987541711\\
375	-5.74999987541711\\
376	-8.74999995889972\\
377	-2.25000006261195\\
378	-5.50000001075584\\
379	-7.50000020870651\\
380	-8.74999995889972\\
381	-3.00000008348261\\
382	-9.75000027131847\\
383	-7.00000005249714\\
384	-1.5000000417413\\
385	-9.75000027131847\\
386	-9.49999997977037\\
387	-7.99999993802907\\
388	-7.50000020870651\\
389	-4.74999998988518\\
390	-4.74999998988518\\
391	-1.99999998450727\\
392	-6.25000003162649\\
393	-2.75000000537792\\
394	-7.00000005249714\\
395	-4.99999985454646\\
396	-1.24999996363662\\
397	-3.99999996901453\\
398	-1.99999998450727\\
399	-2.25000006261195\\
400	0.249999998063408\\
401	-0.499999996126817\\
402	-0.999999992253633\\
403	1.99999998450727\\
404	6.00000016696521\\
405	11.0000000215117\\
406	12.7500003548011\\
407	18.7499996679926\\
408	0.499999996126817\\
409	1.75000001312428\\
410	2.25000006261195\\
411	2.49999992727323\\
412	2.75000000537792\\
413	4.50000012522391\\
414	0.750000020870651\\
415	3.75000010435326\\
416	-0.249999998063408\\
417	3.00000008348261\\
418	3.99999996901453\\
419	4.74999998988518\\
420	12.2499997717049\\
421	3.50000002624857\\
422	3.75000010435326\\
423	3.24999994814388\\
424	1.75000001312428\\
425	3.24999994814388\\
426	3.24999994814388\\
427	0.499999996126817\\
428	3.75000010435326\\
429	3.24999994814388\\
430	3.99999996901453\\
431	6.00000016696521\\
432	3.50000002624857\\
433	1.99999998450727\\
434	1.5000000417413\\
435	3.75000010435326\\
436	-9.00000025044782\\
437	-6.00000016696521\\
438	-3.99999996901453\\
439	2.25000006261195\\
440	5.50000001075584\\
441	3.99999996901453\\
442	7.00000005249714\\
443	13.5000003756717\\
444	15.74999958451\\
445	4.50000012522391\\
446	1.99999998450727\\
447	3.24999994814388\\
448	1.75000001312428\\
449	2.75000000537792\\
450	2.75000000537792\\
451	4.50000012522391\\
452	3.99999996901453\\
453	4.74999998988518\\
454	4.24999983367581\\
455	6.49999989628776\\
456	18.4999993764445\\
457	3.24999994814388\\
458	4.99999985454646\\
459	21.0000005843782\\
460	4.74999998988518\\
461	3.50000002624857\\
462	4.74999998988518\\
463	3.99999996901453\\
464	9.75000027131847\\
465	5.25000014609456\\
466	2.25000006261195\\
467	4.24999983367581\\
468	4.99999985454646\\
469	7.00000005249714\\
470	1.75000001312428\\
471	5.25000014609456\\
472	2.49999992727323\\
473	-10.7499997299636\\
474	-7.24999991715841\\
475	3.99999996901453\\
476	-0.249999998063408\\
477	0.499999996126817\\
478	1.75000001312428\\
479	-173.5000008438\\
480	-171.249999927415\\
481	-168.750000427028\\
482	2.25000006261195\\
483	2.25000006261195\\
484	2.75000000537792\\
485	1.99999998450727\\
486	1.75000001312428\\
487	-0.999999992253633\\
488	1.75000001312428\\
489	-0.499999996126817\\
490	2.49999992727323\\
491	-0.499999996126817\\
492	-135.499997509624\\
493	-1.75000001312428\\
494	0.750000020870651\\
495	-172.750005091798\\
496	1.5000000417413\\
497	0.750000020870651\\
498	-2.49999992727323\\
499	-0.750000020870651\\
500	-3.50000002624857\\
501	-1.75000001312428\\
502	1.75000001312428\\
503	-1.24999996363662\\
504	1.5000000417413\\
505	-0.750000020870651\\
506	3.50000002624857\\
507	171.999995679417\\
508	173.2500022598\\
509	177.249993264191\\
510	-1.99999998450727\\
511	1.5000000417413\\
512	3.99999996901453\\
513	6.49999989628776\\
514	1.75000001312428\\
515	-0.750000020870651\\
516	3.50000002624857\\
517	-0.999999992253633\\
518	0\\
519	1.5000000417413\\
520	9.00000025044782\\
521	1.24999996363662\\
522	2.75000000537792\\
523	-3.99999996901453\\
524	-0.499999996126817\\
525	-1.24999996363662\\
526	-3.24999994814388\\
527	-0.249999998063408\\
528	32.2499983361171\\
529	-2.25000006261195\\
530	0.499999996126817\\
531	-0.750000020870651\\
532	0.249999998063408\\
533	-3.99999996901453\\
534	-0.999999992253633\\
535	2.25000006261195\\
536	2.75000000537792\\
537	-0.750000020870651\\
538	-2.49999992727323\\
539	3.00000008348261\\
540	3.00000008348261\\
541	1.99999998450727\\
542	1.5000000417413\\
543	2.25000006261195\\
544	2.49999992727323\\
545	-0.750000020870651\\
546	-1.5000000417413\\
547	-0.999999992253633\\
548	1.99999998450727\\
549	5.25000014609456\\
550	6.49999989628776\\
551	6.49999989628776\\
552	5.74999987541711\\
553	6.00000016696521\\
554	3.99999996901453\\
555	8.74999995889972\\
556	7.74999964648097\\
557	4.99999985454646\\
558	8.25000022957717\\
559	12.2499997717049\\
560	9.00000025044782\\
561	5.25000014609456\\
562	12.500000063253\\
563	8.74999995889972\\
564	9.00000025044782\\
565	9.99999970909292\\
566	11.0000000215117\\
567	12.9999997925755\\
568	12.9999997925755\\
569	11.2500003130598\\
570	9.49999997977037\\
571	11.0000000215117\\
572	11.2500003130598\\
573	11.0000000215117\\
574	13.5000003756717\\
575	16.9999993347032\\
576	12.9999997925755\\
577	15.4999992929619\\
578	12.7500003548011\\
579	13.2500000841236\\
580	15.2500007089611\\
581	11.7500000423823\\
582	13.7499998134462\\
583	12.9999997925755\\
584	16.2500001676062\\
585	14.7500001258649\\
586	17.2499996262513\\
587	19.750000834185\\
588	16.2500001676062\\
589	18.0000005008956\\
590	19.2500002510888\\
591	16.2500001676062\\
592	18.2500007924437\\
593	18.0000005008956\\
594	18.4999993764445\\
595	18.4999993764445\\
596	23.2499997932165\\
597	19.9999994181858\\
598	23.2499997932165\\
599	21.7499997514752\\
600	177.50000550857\\
601	18.0000005008956\\
602	18.4999993764445\\
603	22.7499992101203\\
604	18.4999993764445\\
605	21.0000005843782\\
606	23.7500003763127\\
607	24.0000006678608\\
608	23.2499997932165\\
609	18.0000005008956\\
610	25.0000001265059\\
611	25.9999995851511\\
612	21.0000005843782\\
613	22.9999995016684\\
614	21.4999994599271\\
615	19.9999994181858\\
616	26.7500004597954\\
617	26.2499998766992\\
618	26.5000001682473\\
619	12.0000003339304\\
620	25.0000001265059\\
621	23.5000000847646\\
622	26.5000001682473\\
623	24.7499998349579\\
624	21.4999994599271\\
625	25.5000007096021\\
626	25.5000007096021\\
627	27.0000007513434\\
628	20.7500002928301\\
629	22.7499992101203\\
630	25.749999293603\\
631	26.5000001682473\\
632	25.9999995851511\\
633	18.9999999595407\\
634	23.2499997932165\\
635	17.2499996262513\\
636	25.5000007096021\\
637	31.4999991690201\\
638	24.4999995434097\\
639	23.5000000847646\\
640	25.5000007096021\\
641	24.7499998349579\\
642	25.5000007096021\\
643	26.7500004597954\\
644	22.0000000430233\\
645	27.4999996268924\\
646	25.9999995851511\\
647	19.750000834185\\
};
\end{axis}
\end{tikzpicture}%}
      \caption{The orientation of the vehicle with respect to the orientation
        of the centerline through time.}
      \label{fig:centerline_mpc_psi}
    \end{figure}
  \end{minipage}
\end{minipage}
}\\


\begin{figure}[H]\centering
  \scalebox{0.6}{% This file was created by matlab2tikz.
%
%The latest updates can be retrieved from
%  http://www.mathworks.com/matlabcentral/fileexchange/22022-matlab2tikz-matlab2tikz
%where you can also make suggestions and rate matlab2tikz.
%
\definecolor{mycolor1}{rgb}{0.00000,0.44700,0.74100}%
%
\begin{tikzpicture}

\begin{axis}[%
width=4.133in,
height=3.26in,
at={(0.693in,0.44in)},
scale only axis,
xmin=34,
xmax=80,
xmajorgrids,
ymin=-20,
ymax=50,
ymajorgrids,
axis background/.style={fill=white}
]
\addplot [color=mycolor1,solid,forget plot]
  table[row sep=crcr]{%
1	31.6634421807639\\
2	43.0494563196344\\
3	53.1579826119969\\
4	27.7315686529652\\
5	55.8029767831789\\
6	41.4638074132484\\
7	44.1156488490733\\
8	52.4931490736608\\
9	47.2103870714386\\
10	51.2507240032571\\
11	46.970039544735\\
12	46.8152947788992\\
13	51.092130425824\\
14	48.5739592071508\\
15	29.2636859517312\\
16	59.9629137424589\\
17	39.4352104939033\\
18	27.9437758002883\\
19	59.7708078418632\\
20	39.4708845719383\\
21	23.1832107884176\\
22	60.0000016696521\\
23	39.7987268219097\\
24	54.9314412299907\\
25	6.66295711082687\\
26	60.0000016696521\\
27	44.0041767467241\\
28	47.6053734960459\\
29	31.9469155218869\\
30	58.769488449161\\
31	39.8516846936394\\
32	43.8644720574407\\
33	22.4738905204699\\
34	43.9004603241775\\
35	32.9414183857975\\
36	24.6693506687856\\
37	11.5133356951887\\
38	3.77687519118996\\
39	6.35407333155726\\
40	2.96654752447518\\
41	-0.875923202678005\\
42	-1.40451997356005\\
43	-2.79513923149865\\
44	-0.0885438376563347\\
45	-10.4587348505448\\
46	-3.95084267067113\\
47	-9.06240085870505\\
48	-6.95504801624835\\
49	-8.18328891818\\
50	-1.28951463569755\\
51	-0.476360818131108\\
52	0.655020050602689\\
53	1.11899117238366\\
54	2.96074549221866\\
55	4.91597614803743\\
56	6.36352076383986\\
57	3.68779885224242\\
58	4.42344096170406\\
59	-2.13289348108432\\
60	-0.166389651401299\\
61	0.196987484820354\\
62	2.07344436832218\\
63	1.00288339930044\\
64	-0.488843682530394\\
65	0.437612781443656\\
66	-3.27202730821717\\
67	-2.08510158024469\\
68	-1.7677673628679\\
69	-2.49728535396497\\
70	-1.45116845804394\\
71	-1.79393082937045\\
72	-0.228021970098615\\
73	-1.10169040993772\\
74	-0.994094760107628\\
75	0.24385002636562\\
76	0.457579292017454\\
77	0.791476237883848\\
78	-1.58187031767107\\
79	1.47616780417984\\
80	-1.66246153395705\\
81	-8.68727747797784\\
82	-9.40037313298927\\
83	-3.51405036570152\\
84	-4.29052761488973\\
85	-16.9963759193485\\
86	-15.5249226532433\\
87	-7.68941928363754\\
88	-22.6544943825034\\
89	-10.132472081819\\
90	-23.2647615395602\\
91	-22.4991417298314\\
92	-21.8665620523034\\
};
\end{axis}
\end{tikzpicture}%}
  \caption{The steering angle given to the front wheels.}
  \label{fig:centerline_mpc_input}
\end{figure}

The vehicle succeeded in positioning itself in the centerline of the corridor.
The displacement error is eliminated entirely by the end of the corridor,
and the same applies for the vehicle's orientation.
