This report illustrated the theoretical approaches, theoretical and
practical solutions, and results to three control problems. These problems,
although seemingly simple, required tailored solutions depending on the
nature of their respective available resources (e.g. for localisation),
and solution regimes (e.g. predictive control). Practical experience showed us
that the apparent simplicity of problems does not go hand-in-hand with the
difficulty of their solutions: the real world is not sterile like a simulation
environment. In light of the discrepancy between the performance of purely
theoretically sound solutions and their expected performance, we manipulated
minor or major parts of these solutions and tweaked others, so as to gain an
understanding of the hidden dynamics and implications present in the real world,
and exploit it in order to reduce the aforementioned discrepancies to a level
that we deemed to be tolerable.

In the problem of approaching and maintaining the centerline of a virtual
lane, our solutions made it possible for the vehicle to execute this task
with success. The predictive control approach featured a slower settling time
than the PID one, however the former proved to be more robust than the latter.
In the more difficult problem of approaching and maintaining the trajectory
of a circumference of a circle with a radius of $1.5$ m, the vehicle's
displacement error does not exceed $3.5$ cm.

Lastly, the length of this report does not meet the 10-page requirement.To
have been so, that would mean that its quality would have been worse. The
report is written in a way so as to be read fluently, comprehensively and
concisely: the length of the report is proportional to the joint difficulty of
the problems tackled and the representation of their solution. The report uses
signposts to divide and link all the separate sub-problems and the approach
taken to handle them in order to order meaning and increase comprehension.
Furthermore, in the spirit of, and interest in providing a solid documentation
for future potential successors of our work, we deemed it would be best to
combine all documentation and methodical approaches in one document. This we
think will augment clarity and set a foundation for understanding the handling
of dynamics of a F1/10 (or other similar) vehicle.
