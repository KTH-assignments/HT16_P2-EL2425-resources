The reference circle tested was kept constant, with a radius of $r=1.5$ meters,
centered around point $(x_{O'}, y_{O'}) \equiv (0.95, -0.1)$.
In practice, equation \ref{eq:circular_mpc:theta_k} was found not to be
precise enough for the vehicle to follow the references that the angles
$\theta_k$ dictated: the vehicle could travel in a circular fashion,
but the radius of its trajectory was larger than that of the reference
trajectory, as can be seen in figure \ref{fig:circular_mpc_without_H}.
The trajectory of the vehicle is marked with blue, while the reference
trajectory is marked with red.

\begin{figure}[H]\centering
  \scalebox{0.8}{% This file was created by matlab2tikz.
%
%The latest updates can be retrieved from
%  http://www.mathworks.com/matlabcentral/fileexchange/22022-matlab2tikz-matlab2tikz
%where you can also make suggestions and rate matlab2tikz.
%
\definecolor{mycolor1}{rgb}{0.00000,0.44700,0.74100}%
\definecolor{mycolor2}{rgb}{0.85000,0.32500,0.09800}%
%
\begin{tikzpicture}

\begin{axis}[%
width=3.26in,
height=3.26in,
at={(1.13in,0.44in)},
scale only axis,
xmin=-1.5,
xmax=3.5,
xlabel={meters},
ymin=-2.5,
ymax=2.5,
ylabel={meters},
axis background/.style={fill=white}
]
\addplot [color=mycolor1,solid,forget plot]
  table[row sep=crcr]{%
0.755446791648865	-1.92765522003174\\
0.997251868247986	-1.94691145420074\\
1.23979032039642	-1.93458354473114\\
1.23979032039642	-1.93458354473114\\
1.23979032039642	-1.93458354473114\\
1.23979032039642	-1.93458354473114\\
2.13184332847595	-1.56651592254639\\
2.3202109336853	-1.42087268829346\\
2.4790894985199	-1.2372442483902\\
2.60945677757263	-1.0340850353241\\
2.71316146850586	-0.81447845697403\\
2.78548264503479	-0.583753883838654\\
2.82592368125916	-0.34506967663765\\
2.83632040023804	-0.103753045201302\\
2.81562495231628	0.137739881873131\\
2.76676869392395	0.375363230705261\\
2.68737244606018	0.604087769985199\\
2.58298325538635	0.8232541680336\\
2.45067119598389	1.02536118030548\\
2.29514002799988	1.21165573596954\\
2.11762022972107	1.37604510784149\\
1.92036867141724	1.51652634143829\\
1.70591199398041	1.63349878787994\\
1.47919094562531	1.72203171253204\\
1.24200248718262	1.78184354305267\\
0.998267948627472	1.80944502353668\\
0.752689063549042	1.80571484565735\\
0.508723497390747	1.77060925960541\\
0.272869318723679	1.70391809940338\\
0.048636369407177	1.60482001304626\\
-0.159742817282677	1.47862660884857\\
-0.348202556371689	1.32503628730774\\
-0.513682782649994	1.14739012718201\\
-0.653517782688141	0.949056923389435\\
-0.763985633850098	0.73638129234314\\
-0.845144331455231	0.510270833969116\\
-0.894858717918396	0.27400067448616\\
-0.914202809333801	0.0345293767750263\\
-0.902915894985199	-0.204258129000664\\
-0.860493421554565	-0.442157417535782\\
-0.789954900741577	-0.673126518726349\\
-0.690703630447388	-0.893589854240417\\
-0.565318465232849	-1.10047221183777\\
-0.416278451681137	-1.29269981384277\\
-0.243769630789757	-1.4639345407486\\
-0.0534829720854759	-1.61366772651672\\
0.156762599945068	-1.74130213260651\\
0.380153447389603	-1.84031081199646\\
0.613715827465057	-1.90782046318054\\
0.85484367609024	-1.94404196739197\\
1.09792375564575	-1.94766473770142\\
1.34096920490265	-1.92070341110229\\
1.34096920490265	-1.92070341110229\\
1.34096920490265	-1.92070341110229\\
2.01628828048706	-1.65326857566833\\
2.21696043014526	-1.51000082492828\\
2.39527344703674	-1.34390985965729\\
2.54714202880859	-1.15403664112091\\
2.67172503471375	-0.944810271263123\\
2.76743006706238	-0.720972001552582\\
2.82914233207703	-0.486809194087982\\
2.85981941223145	-0.245166182518005\\
2.86165332794189	-0.00440277485176921\\
2.830402135849	0.23641063272953\\
2.76739406585693	0.470549732446671\\
2.68684101104736	0.676186382770538\\
2.56118607521057	0.907911062240601\\
2.418705701828	1.10398173332214\\
2.25316524505615	1.28070545196533\\
2.0663526058197	1.43275964260101\\
1.86079180240631	1.56204700469971\\
1.64034748077393	1.66506922245026\\
1.40907299518585	1.74146509170532\\
1.16898155212402	1.78768444061279\\
0.923721611499786	1.80420696735382\\
0.678164482116699	1.78890001773834\\
0.435515224933624	1.74311292171478\\
0.20393605530262	1.66446828842163\\
-0.0152933187782764	1.55593180656433\\
-0.21561561524868	1.41807079315186\\
-0.39691948890686	1.25635135173798\\
-0.554153501987457	1.07082843780518\\
-0.68410450220108	0.866062760353088\\
-0.784692823886871	0.648669958114624\\
-0.856628239154816	0.418573468923569\\
-0.896226644515991	0.180346593260765\\
-0.90610671043396	-0.0604202300310135\\
-0.885870337486267	-0.300131857395172\\
-0.833723366260529	-0.535382568836212\\
-0.754890024662018	-0.7654909491539\\
-0.645513653755188	-0.980587780475616\\
-0.5124791264534	-1.18310391902924\\
-0.352799892425537	-1.36718332767487\\
-0.173278227448463	-1.52942180633545\\
0.0265247654169798	-1.67104256153107\\
0.242585763335228	-1.78699457645416\\
0.470258563756943	-1.87281060218811\\
0.706939458847046	-1.92751801013947\\
0.949128091335297	-1.95063650608063\\
1.19222915172577	-1.94112133979797\\
1.43096220493317	-1.8984968662262\\
1.43096220493317	-1.8984968662262\\
1.43096220493317	-1.8984968662262\\
2.09517240524292	-1.59943115711212\\
2.28620433807373	-1.44580364227295\\
2.45185971260071	-1.26770615577698\\
2.58846521377563	-1.06723153591156\\
2.67634153366089	-0.849115908145905\\
2.77069354057312	-0.620447933673859\\
2.81804490089417	-0.384013712406158\\
2.83687829971313	-0.141934052109718\\
2.82391309738159	0.100684106349945\\
2.78602862358093	0.338697105646133\\
2.71585059165955	0.571651041507721\\
2.61691927909851	0.792527914047241\\
2.49191927909851	0.999286293983459\\
2.34138822555542	1.18928074836731\\
2.16749095916748	1.35733246803284\\
1.9733362197876	1.50017690658569\\
1.76086807250977	1.62018179893494\\
1.53597009181976	1.71252977848053\\
1.30094122886658	1.77617514133453\\
1.05803871154785	1.80855536460876\\
0.812735617160797	1.81052148342133\\
0.568201422691345	1.78101134300232\\
0.330842941999435	1.71982669830322\\
0.103293336927891	1.62699604034424\\
-0.107074968516827	1.50488519668579\\
-0.298530966043472	1.35629916191101\\
-0.467982441186905	1.18165755271912\\
-0.614237427711487	0.987277150154114\\
-0.728822708129883	0.776105046272278\\
-0.81565260887146	0.552050948143005\\
-0.871373534202576	0.316713124513626\\
-0.897094011306763	0.0771795213222504\\
-0.891961753368378	-0.163093835115433\\
-0.856146693229675	-0.40153107047081\\
-0.791018068790436	-0.634422719478607\\
-0.698815524578094	-0.859058976173401\\
-0.578017234802246	-1.06840395927429\\
-0.433380514383316	-1.26251518726349\\
-0.264321714639664	-1.43887686729431\\
-0.0771458968520164	-1.59487247467041\\
0.130186781287193	-1.72576940059662\\
0.351509422063828	-1.83100390434265\\
0.583444833755493	-1.90372085571289\\
0.823310136795044	-1.94588649272919\\
1.06662285327911	-1.95423233509064\\
1.30805611610413	-1.93093848228455\\
1.55616796016693	-1.90763986110687\\
1.55616796016693	-1.90763986110687\\
1.9873514175415	-1.67462050914764\\
2.17667055130005	-1.51622354984283\\
2.35065150260925	-1.35702419281006\\
2.49202752113342	-1.16181361675262\\
2.60740971565247	-0.950334429740906\\
2.69782876968384	-0.725549280643463\\
2.76052284240723	-0.491860508918762\\
2.79468369483948	-0.252350836992264\\
2.79777598381042	-0.0103976260870695\\
2.77348279953003	0.231382593512535\\
2.71946263313293	0.467768251895905\\
2.63789200782776	0.698130667209625\\
2.52791523933411	0.914671778678894\\
2.3917407989502	1.11522316932678\\
2.23006510734558	1.29410111904144\\
2.04717707633972	1.45134627819061\\
1.8432924747467	1.58462011814117\\
1.62450790405273	1.69220638275146\\
1.39416611194611	1.77037155628204\\
1.15528964996338	1.81762635707855\\
0.910539865493774	1.83338952064514\\
0.665875196456909	1.81854474544525\\
0.425524741411209	1.76999247074127\\
0.195037171244621	1.68938195705414\\
-0.0217952467501163	1.57706713676453\\
-0.219426393508911	1.4386225938797\\
-0.396863549947739	1.27453303337097\\
-0.551758348941803	1.0877902507782\\
-0.678338527679443	0.881888508796692\\
-0.775316298007965	0.663808226585388\\
-0.842897832393646	0.432714998722076\\
-0.87923139333725	0.194945141673088\\
-0.887253820896149	-0.0450282581150532\\
-0.864304482936859	-0.284469127655029\\
-0.813374400138855	-0.520621955394745\\
-0.734459340572357	-0.750091791152954\\
-0.627753913402557	-0.967470705509186\\
-0.49539902806282	-1.17021822929382\\
-0.338378846645355	-1.35714769363403\\
-0.16076622903347	-1.52256536483765\\
0.0366044268012047	-1.66773450374603\\
0.251460552215576	-1.78742980957031\\
0.478289484977722	-1.87643814086914\\
0.714580655097961	-1.93487989902496\\
0.956739068031311	-1.96029770374298\\
1.19953954219818	-1.95218503475189\\
1.42141032218933	-1.91864788532257\\
1.42141032218933	-1.91864788532257\\
1.42141032218933	-1.91864788532257\\
1.42141032218933	-1.91864788532257\\
2.2983570098877	-1.46411228179932\\
2.46882748603821	-1.28985917568207\\
2.6172468662262	-1.09472417831421\\
2.73579168319702	-0.881989061832428\\
2.82424259185791	-0.655336618423462\\
2.87943553924561	-0.420368283987045\\
2.90334486961365	-0.180075421929359\\
2.89142966270447	0.0614482499659061\\
2.8512237071991	0.299313545227051\\
2.78057456016541	0.530555844306946\\
2.68072748184204	0.750638365745544\\
2.55483365058899	0.955904722213745\\
2.40444827079773	1.14487063884735\\
2.23006319999695	1.31035435199738\\
2.03786540031433	1.45430433750153\\
1.82662284374237	1.57458662986755\\
1.6025755405426	1.67062270641327\\
1.36891329288483	1.73928332328796\\
1.12783575057983	1.77835810184479\\
0.882141292095184	1.78786206245422\\
0.636087954044342	1.76942086219788\\
0.394751518964767	1.71835625171661\\
0.163449645042419	1.63542222976685\\
-0.0526979453861713	1.5219212770462\\
-0.251424759626389	1.38057172298431\\
-0.428036153316498	1.21346855163574\\
-0.583316087722778	1.0246467590332\\
-0.707644939422607	0.818110466003418\\
-0.804391860961914	0.597996890544891\\
-0.871244370937347	0.365672081708908\\
-0.906679630279541	0.126236170530319\\
-0.909547686576843	-0.114196628332138\\
-0.884248733520508	-0.353711038827896\\
-0.827111303806305	-0.588330447673798\\
-0.740082323551178	-0.814280927181244\\
-0.626386821269989	-1.02716088294983\\
-0.485846489667892	-1.2244588136673\\
-0.321138083934784	-1.40244030952454\\
-0.1362124979496	-1.56037724018097\\
0.0671139881014824	-1.6947021484375\\
0.286682963371277	-1.80482935905457\\
0.516567707061768	-1.88227927684784\\
0.755149662494659	-1.92923200130463\\
0.99797111749649	-1.94294309616089\\
1.24090814590454	-1.92686033248901\\
1.49190843105316	-1.91623556613922\\
1.49190843105316	-1.91623556613922\\
1.49190843105316	-1.91623556613922\\
2.12979388237	-1.55354726314545\\
2.30559015274048	-1.38812816143036\\
};
\addplot [color=mycolor2,solid,forget plot]
  table[row sep=crcr]{%
2.44977154273459	-0.0738213903440747\\
2.44908624052864	-0.0476507549462486\\
2.44794430213186	-0.0214960656355843\\
2.44634607538974	0.00463471061618795\\
2.44429204713762	0.0307336141214873\\
2.44178284305241	0.0567926949014802\\
2.43881922746198	0.0828040151077212\\
2.43540210311236	0.108759651440098\\
2.43153251089271	0.134651697560346\\
2.42721162951831	0.160472266500396\\
2.4224407751715	0.186213493064817\\
2.41722140110071	0.211867536226639\\
2.41155509717785	0.237426581515798\\
2.40544358941399	0.262882843399502\\
2.3988887394336	0.288228567653781\\
2.39189254390748	0.313456033725499\\
2.38445713394455	0.338557557084105\\
2.37658477444273	0.363525491562421\\
2.36827786339898	0.388352231685735\\
2.35953893117886	0.413030214988503\\
2.3503706397458	0.43755192431795\\
2.34077578185018	0.461909890123868\\
2.33075728017866	0.486096692733911\\
2.3203181864639	0.5101049646137\\
2.30946168055497	0.533927392611049\\
2.29819106944875	0.557556720183616\\
2.28650978628255	0.58098574960932\\
2.27442138928839	0.604207344178836\\
2.26192956070909	0.627214430369506\\
2.24903810567666	0.65\\
2.23575095105317	0.672557112365081\\
2.22207214423464	0.694878896349807\\
2.20800585191814	0.716958552522541\\
2.19355635883256	0.73878935520612\\
2.17872806643349	0.760364654526569\\
2.16352549156242	0.78167787843871\\
2.14795326507094	0.802722534728072\\
2.13201613041008	0.823492212988488\\
2.11571894218546	0.843980586574756\\
2.09906666467847	0.864181414529809\\
2.08206437033416	0.884088543485761\\
2.06471723821609	0.903695909538287\\
2.04703055242876	0.922997540093748\\
2.02900970050798	0.941987555688496\\
2.01066017177982	0.960660171779821\\
1.9919875556885	0.979009700507977\\
1.97299754009375	0.997030552428756\\
1.95369590953829	1.01471723821609\\
1.93408854348576	1.03206437033416\\
1.91418141452981	1.04906666467847\\
1.89398058657476	1.06571894218546\\
1.87349221298849	1.08201613041008\\
1.85272253472807	1.09795326507094\\
1.83167787843871	1.11352549156242\\
1.81036465452657	1.12872806643349\\
1.78878935520612	1.14355635883256\\
1.76695855252254	1.15800585191814\\
1.74487889634981	1.17207214423464\\
1.72255711236508	1.18575095105317\\
1.7	1.19903810567666\\
1.67721443036951	1.21192956070909\\
1.65420734417884	1.22442138928839\\
1.63098574960932	1.23650978628255\\
1.60755672018362	1.24819106944875\\
1.58392739261105	1.25946168055497\\
1.5601049646137	1.2703181864639\\
1.53609669273391	1.28075728017866\\
1.51190989012387	1.29077578185018\\
1.48755192431795	1.3003706397458\\
1.4630302149885	1.30953893117886\\
1.43835223168574	1.31827786339898\\
1.41352549156242	1.32658477444273\\
1.38855755708411	1.33445713394455\\
1.3634560337255	1.34189254390748\\
1.33822856765378	1.3488887394336\\
1.3128828433995	1.35544358941399\\
1.2874265815158	1.36155509717785\\
1.26186753622664	1.36722140110071\\
1.23621349306482	1.3724407751715\\
1.2104722665004	1.37721162951831\\
1.18465169756035	1.38153251089271\\
1.1587596514401	1.38540210311236\\
1.13280401510772	1.38881922746198\\
1.10679269490148	1.39178284305241\\
1.08073361412149	1.39429204713762\\
1.05463471061619	1.39634607538974\\
1.02850393436442	1.39794430213186\\
1.00234924505375	1.39908624052864\\
0.976178609655925	1.39977154273459\\
0.95	1.4\\
0.923821390344075	1.39977154273459\\
0.897650754946248	1.39908624052864\\
0.871496065635584	1.39794430213186\\
0.845365289383812	1.39634607538974\\
0.819266385878513	1.39429204713762\\
0.79320730509852	1.39178284305241\\
0.767195984892279	1.38881922746198\\
0.741240348559902	1.38540210311236\\
0.715348302439654	1.38153251089271\\
0.689527733499605	1.37721162951831\\
0.663786506935183	1.3724407751715\\
0.638132463773361	1.36722140110071\\
0.612573418484202	1.36155509717785\\
0.587117156600498	1.35544358941399\\
0.561771432346219	1.3488887394336\\
0.536543966274501	1.34189254390748\\
0.511442442915895	1.33445713394455\\
0.486474508437579	1.32658477444273\\
0.461647768314265	1.31827786339898\\
0.436969785011497	1.30953893117886\\
0.41244807568205	1.3003706397458\\
0.388090109876132	1.29077578185018\\
0.363903307266089	1.28075728017866\\
0.3398950353863	1.2703181864639\\
0.316072607388951	1.25946168055497\\
0.292443279816384	1.24819106944875\\
0.26901425039068	1.23650978628255\\
0.245792655821164	1.22442138928839\\
0.222785569630494	1.21192956070909\\
0.2	1.19903810567666\\
0.177442887634919	1.18575095105317\\
0.155121103650193	1.17207214423464\\
0.133041447477459	1.15800585191814\\
0.11121064479388	1.14355635883256\\
0.0896353454734307	1.12872806643349\\
0.0683221215612904	1.11352549156242\\
0.0472774652719274	1.09795326507094\\
0.0265077870115125	1.08201613041008\\
0.00601941342524404	1.06571894218546\\
-0.0141814145298091	1.04906666467847\\
-0.0340885434857607	1.03206437033416\\
-0.0536959095382874	1.01471723821609\\
-0.0729975400937477	0.997030552428756\\
-0.0919875556884961	0.979009700507977\\
-0.110660171779821	0.960660171779821\\
-0.129009700507977	0.941987555688496\\
-0.147030552428756	0.922997540093748\\
-0.164717238216091	0.903695909538288\\
-0.182064370334158	0.884088543485761\\
-0.199066664678467	0.864181414529809\\
-0.215718942185456	0.843980586574756\\
-0.232016130410083	0.823492212988488\\
-0.24795326507094	0.802722534728072\\
-0.263525491562421	0.78167787843871\\
-0.278728066433488	0.760364654526569\\
-0.293556358832562	0.73878935520612\\
-0.308005851918136	0.716958552522541\\
-0.322072144234639	0.694878896349807\\
-0.335750951053168	0.672557112365082\\
-0.349038105676658	0.65\\
-0.361929560709094	0.627214430369506\\
-0.374421389288391	0.604207344178836\\
-0.386509786282552	0.58098574960932\\
-0.398191069448751	0.557556720183616\\
-0.409461680554975	0.533927392611049\\
-0.420318186463901	0.510104964613701\\
-0.430757280178661	0.486096692733911\\
-0.440775781850181	0.461909890123868\\
-0.450370639745803	0.43755192431795\\
-0.459538931178862	0.413030214988503\\
-0.468277863398975	0.388352231685735\\
-0.47658477444273	0.363525491562421\\
-0.484457133944553	0.338557557084105\\
-0.491892543907478	0.313456033725499\\
-0.498888739433602	0.288228567653782\\
-0.505443589413995	0.262882843399502\\
-0.511555097177853	0.237426581515798\\
-0.517221401100709	0.211867536226639\\
-0.522440775171496	0.186213493064817\\
-0.527211629518312	0.160472266500395\\
-0.531532510892706	0.134651697560346\\
-0.535402103112356	0.108759651440098\\
-0.538819227461983	0.0828040151077213\\
-0.54178284305241	0.0567926949014806\\
-0.544292047137618	0.0307336141214873\\
-0.546346075389736	0.00463471061618828\\
-0.547944302131861	-0.0214960656355843\\
-0.549086240528644	-0.0476507549462483\\
-0.549771542734587	-0.0738213903440748\\
-0.55	-0.0999999999999998\\
-0.549771542734587	-0.126178609655925\\
-0.549086240528644	-0.152349245053751\\
-0.547944302131861	-0.178503934364415\\
-0.546346075389736	-0.204634710616188\\
-0.544292047137618	-0.230733614121487\\
-0.54178284305241	-0.25679269490148\\
-0.538819227461983	-0.282804015107721\\
-0.535402103112355	-0.308759651440098\\
-0.531532510892707	-0.334651697560346\\
-0.527211629518312	-0.360472266500396\\
-0.522440775171496	-0.386213493064817\\
-0.517221401100708	-0.411867536226639\\
-0.511555097177853	-0.437426581515797\\
-0.505443589413995	-0.462882843399501\\
-0.498888739433603	-0.488228567653781\\
-0.491892543907478	-0.513456033725499\\
-0.484457133944553	-0.538557557084105\\
-0.476584774442731	-0.563525491562421\\
-0.468277863398975	-0.588352231685735\\
-0.459538931178863	-0.613030214988503\\
-0.450370639745803	-0.637551924317951\\
-0.440775781850181	-0.661909890123868\\
-0.430757280178661	-0.68609669273391\\
-0.420318186463901	-0.7101049646137\\
-0.409461680554975	-0.733927392611049\\
-0.39819106944875	-0.757556720183616\\
-0.386509786282552	-0.78098574960932\\
-0.37442138928839	-0.804207344178836\\
-0.361929560709094	-0.827214430369505\\
-0.349038105676658	-0.85\\
-0.335750951053168	-0.872557112365081\\
-0.322072144234639	-0.894878896349807\\
-0.308005851918136	-0.916958552522541\\
-0.293556358832563	-0.93878935520612\\
-0.278728066433488	-0.960364654526569\\
-0.263525491562421	-0.98167787843871\\
-0.247953265070939	-1.00272253472807\\
-0.232016130410083	-1.02349221298849\\
-0.215718942185456	-1.04398058657476\\
-0.199066664678467	-1.06418141452981\\
-0.182064370334158	-1.08408854348576\\
-0.164717238216092	-1.10369590953829\\
-0.147030552428756	-1.12299754009375\\
-0.129009700507977	-1.1419875556885\\
-0.110660171779821	-1.16066017177982\\
-0.0919875556884959	-1.17900970050798\\
-0.0729975400937479	-1.19703055242876\\
-0.0536959095382872	-1.21471723821609\\
-0.034088543485761	-1.23206437033416\\
-0.0141814145298093	-1.24906666467847\\
0.00601941342524315	-1.26571894218546\\
0.0265077870115129	-1.28201613041008\\
0.0472774652719276	-1.29795326507094\\
0.0683221215612901	-1.31352549156242\\
0.0896353454734304	-1.32872806643349\\
0.11121064479388	-1.34355635883256\\
0.133041447477459	-1.35800585191814\\
0.155121103650192	-1.37207214423464\\
0.177442887634918	-1.38575095105317\\
0.199999999999999	-1.39903810567666\\
0.222785569630495	-1.41192956070909\\
0.245792655821164	-1.42442138928839\\
0.26901425039068	-1.43650978628255\\
0.292443279816383	-1.44819106944875\\
0.316072607388951	-1.45946168055498\\
0.3398950353863	-1.4703181864639\\
0.363903307266089	-1.48075728017866\\
0.388090109876132	-1.49077578185018\\
0.412448075682049	-1.5003706397458\\
0.436969785011497	-1.50953893117886\\
0.461647768314265	-1.51827786339898\\
0.486474508437579	-1.52658477444273\\
0.511442442915894	-1.53445713394455\\
0.536543966274502	-1.54189254390748\\
0.561771432346219	-1.5488887394336\\
0.587117156600498	-1.55544358941399\\
0.612573418484202	-1.56155509717785\\
0.63813246377336	-1.56722140110071\\
0.663786506935183	-1.5724407751715\\
0.689527733499604	-1.57721162951831\\
0.715348302439653	-1.58153251089271\\
0.741240348559901	-1.58540210311236\\
0.767195984892279	-1.58881922746198\\
0.79320730509852	-1.59178284305241\\
0.819266385878513	-1.59429204713762\\
0.845365289383812	-1.59634607538974\\
0.871496065635583	-1.59794430213186\\
0.897650754946249	-1.59908624052864\\
0.923821390344075	-1.59977154273459\\
0.95	-1.6\\
0.976178609655925	-1.59977154273459\\
1.00234924505375	-1.59908624052864\\
1.02850393436442	-1.59794430213186\\
1.05463471061619	-1.59634607538974\\
1.08073361412149	-1.59429204713762\\
1.10679269490148	-1.59178284305241\\
1.13280401510772	-1.58881922746198\\
1.1587596514401	-1.58540210311236\\
1.18465169756035	-1.58153251089271\\
1.21047226650039	-1.57721162951831\\
1.23621349306482	-1.5724407751715\\
1.26186753622664	-1.56722140110071\\
1.2874265815158	-1.56155509717785\\
1.3128828433995	-1.55544358941399\\
1.33822856765378	-1.5488887394336\\
1.3634560337255	-1.54189254390748\\
1.38855755708411	-1.53445713394455\\
1.41352549156242	-1.52658477444273\\
1.43835223168573	-1.51827786339898\\
1.4630302149885	-1.50953893117886\\
1.48755192431795	-1.5003706397458\\
1.51190989012387	-1.49077578185018\\
1.53609669273391	-1.48075728017866\\
1.5601049646137	-1.4703181864639\\
1.58392739261105	-1.45946168055497\\
1.60755672018362	-1.44819106944875\\
1.63098574960932	-1.43650978628255\\
1.65420734417884	-1.42442138928839\\
1.67721443036951	-1.41192956070909\\
1.7	-1.39903810567666\\
1.72255711236508	-1.38575095105317\\
1.74487889634981	-1.37207214423464\\
1.76695855252254	-1.35800585191814\\
1.78878935520612	-1.34355635883256\\
1.81036465452657	-1.32872806643349\\
1.83167787843871	-1.31352549156242\\
1.85272253472807	-1.29795326507094\\
1.87349221298849	-1.28201613041008\\
1.89398058657476	-1.26571894218546\\
1.91418141452981	-1.24906666467847\\
1.93408854348576	-1.23206437033416\\
1.95369590953829	-1.21471723821609\\
1.97299754009375	-1.19703055242876\\
1.9919875556885	-1.17900970050798\\
2.01066017177982	-1.16066017177982\\
2.02900970050798	-1.1419875556885\\
2.04703055242876	-1.12299754009375\\
2.06471723821609	-1.10369590953829\\
2.08206437033416	-1.08408854348576\\
2.09906666467847	-1.06418141452981\\
2.11571894218546	-1.04398058657476\\
2.13201613041008	-1.02349221298849\\
2.14795326507094	-1.00272253472807\\
2.16352549156242	-0.98167787843871\\
2.17872806643349	-0.96036465452657\\
2.19355635883256	-0.93878935520612\\
2.20800585191814	-0.91695855252254\\
2.22207214423464	-0.894878896349807\\
2.23575095105317	-0.872557112365082\\
2.24903810567666	-0.850000000000001\\
2.26192956070909	-0.827214430369505\\
2.27442138928839	-0.804207344178836\\
2.28650978628255	-0.78098574960932\\
2.29819106944875	-0.757556720183617\\
2.30946168055497	-0.733927392611049\\
2.3203181864639	-0.7101049646137\\
2.33075728017866	-0.686096692733911\\
2.34077578185018	-0.661909890123868\\
2.3503706397458	-0.637551924317951\\
2.35953893117886	-0.613030214988503\\
2.36827786339898	-0.588352231685735\\
2.37658477444273	-0.563525491562421\\
2.38445713394455	-0.538557557084106\\
2.39189254390748	-0.513456033725498\\
2.3988887394336	-0.488228567653781\\
2.40544358941399	-0.462882843399502\\
2.41155509717785	-0.437426581515798\\
2.41722140110071	-0.41186753622664\\
2.4224407751715	-0.386213493064817\\
2.42721162951831	-0.360472266500396\\
2.43153251089271	-0.334651697560347\\
2.43540210311236	-0.308759651440099\\
2.43881922746198	-0.282804015107721\\
2.44178284305241	-0.25679269490148\\
2.44429204713762	-0.230733614121487\\
2.44634607538974	-0.204634710616188\\
2.44794430213186	-0.178503934364417\\
2.44908624052864	-0.152349245053751\\
2.44977154273459	-0.126178609655925\\
2.45	-0.1\\
2.44977154273459	-0.0738213903440754\\
};
\end{axis}
\end{tikzpicture}%
}
  \caption{Trajectory of the vehicle in blue and reference trajectory in red,
    using equation \ref{eq:circular_mpc:theta_k} as a means to set the
    time-varying reference poses.}
  \label{fig:circular_mpc_without_H}
\end{figure}

For this reason, we introduced a corrective factor $H$, which is to
be considered as an additional factor to be tuned. More concretely,
equation \ref{eq:circular_mpc:theta_k} was modified as such:

\begin{align}
  \theta_k &= \theta_T + k T_s \dfrac{v}{R} H \\
\end{align}

By increasing $H$, the references are pushed forward, further along the circle,
so that the discrepancy between the radius of the traveled and reference
trajectories is minimized.

With penalty matrices

\begin{equation}
  Q =
  \begin{bmatrix}
    100 & 0   & 0 \\
      0 & 100 & 0 \\
      0 & 0   & 10000 \\
  \end{bmatrix}, R = 10
\end{equation}

$H=1.3$ and a horizon length equal to $N=2$\footnote{in practice we found that
references dependent on $\theta_1$ and $\theta_2$ give better results than
references dependent on $\theta_0$ and $\theta_1$, with $N=2$, \textit{or}
references dependent on $\theta_{0:2}$ with $N=3$}, we get the results depicted
in figures \ref{fig:circular_mpc_trajectory_full} and
\ref{fig:circular_mpc_error_full}. The vehicle was positioned at $(1.33, -2.52)$.
Its velocity was kept constant, and its value was set to the minimum that
the F1/10 vehicle can do.

These figures illustrate the entire sequence of approaching and maintaining
the reference trajectory. The former shows the reference trajectory in red
and the trajectory of the vehicle in blue. The latter depicts the displacement
error of the vehicle with respect to its trajectory.

\noindent\makebox[\linewidth][c]{%
\begin{minipage}{\linewidth}
  \begin{minipage}{0.45\linewidth}
    \begin{figure}[H]
      \scalebox{0.6}{\input{./figures/results/circular_mpc/trajectory_full.tex}}
      \caption{Reference trajectory (red) and trajectory of the vehicle (blue),
        for the entire sequence of approaching and maintaining the reference
        trajectory.}
      \label{fig:circular_mpc_trajectory_full}
    \end{figure}
  \end{minipage}
  \hfill
  \begin{minipage}{0.45\linewidth}
    \begin{figure}[H]
      \scalebox{0.6}{% This file was created by matlab2tikz.
%
%The latest updates can be retrieved from
%  http://www.mathworks.com/matlabcentral/fileexchange/22022-matlab2tikz-matlab2tikz
%where you can also make suggestions and rate matlab2tikz.
%
\definecolor{mycolor1}{rgb}{0.00000,0.44700,0.74100}%
%
\begin{tikzpicture}

\begin{axis}[%
width=4.133in,
height=3.26in,
at={(0.693in,0.44in)},
scale only axis,
xmin=0,
xmax=400,
xmajorgrids,
ymin=0,
ymax=1,
ymajorgrids,
axis background/.style={fill=white}
]
\addplot [color=mycolor1,solid,forget plot]
  table[row sep=crcr]{%
1	0.957061102626619\\
2	0.957138922542056\\
3	0.957148786174107\\
4	0.957126233828158\\
5	0.957096357742376\\
6	0.957159124678429\\
7	0.957188710403529\\
8	0.957169184953751\\
9	0.957131008203874\\
10	0.957202131042267\\
11	0.957113324893822\\
12	0.957103427182775\\
13	0.957144438826293\\
14	0.957143068099454\\
15	0.957137197771379\\
16	0.957118027221773\\
17	0.957128359818367\\
18	0.957084288285752\\
19	0.957148709137691\\
20	0.957099724292224\\
21	0.957116999857543\\
22	0.95711588620218\\
23	0.957105351049476\\
24	0.957132214766553\\
25	0.957141874925253\\
26	0.957135665359296\\
27	0.957110991278423\\
28	0.957100004494277\\
29	0.957134427956941\\
30	0.95709461681845\\
31	0.957175286554005\\
32	0.957140860925472\\
33	0.956703578436937\\
34	0.956406223449901\\
35	0.954542629740049\\
36	0.946570666701505\\
37	0.935236821984972\\
38	0.920475887709511\\
39	0.895609175809667\\
40	0.866633428201879\\
41	0.827960599257834\\
42	0.778623486021064\\
43	0.717749587890855\\
44	0.643909379208098\\
45	0.557524677489729\\
46	0.460844364808611\\
47	0.356755119157114\\
48	0.246072700926205\\
49	0.13701790465757\\
50	0.0384996619501276\\
51	0.043152926958582\\
52	0.0830857269493104\\
53	0.0921999492936298\\
54	0.0665135406881128\\
55	0.0263656127873072\\
56	0.0190280521959298\\
57	0.0505705183050796\\
58	0.0676505024138403\\
59	0.0658773221630117\\
60	0.0508983556484498\\
61	0.024310708033397\\
62	0.0175609939205781\\
63	0.0671454060591617\\
64	0.116373465700551\\
65	0.153589829300766\\
66	0.174024842693589\\
67	0.177383130219184\\
68	0.16981628924386\\
69	0.159615054028374\\
70	0.149436752402283\\
71	0.14285181054548\\
72	0.141226037815625\\
73	0.141687376762103\\
74	0.141618334677773\\
75	0.143991629445978\\
76	0.147092326809131\\
77	0.152253498938905\\
78	0.159166550791697\\
79	0.163596415881757\\
80	0.168711503624219\\
81	0.169816265491436\\
82	0.170835728297897\\
83	0.171506076809181\\
84	0.172376457807154\\
85	0.17230824241554\\
86	0.171875203300279\\
87	0.170381687301571\\
88	0.17106560397293\\
89	0.169191309446963\\
90	0.165716560440322\\
91	0.162189122445398\\
92	0.156083735193758\\
93	0.147772070410039\\
94	0.140102564810872\\
95	0.130659197036299\\
96	0.118439721222917\\
97	0.105569792007977\\
98	0.0903132147861711\\
99	0.0767623577559251\\
100	0.0622228550110655\\
101	0.0484915190071225\\
102	0.0341030933261152\\
103	0.023503820330661\\
104	0.0180740027414129\\
105	0.00834240868072566\\
106	0.00135684435729767\\
107	0.00485197302464132\\
108	0.0101097094539261\\
109	0.0121652717894393\\
110	0.00573224539958319\\
111	0.00939530781721152\\
112	0.0222269912557379\\
113	0.0283096589561186\\
114	0.0386227099534122\\
115	0.0438243446277975\\
116	0.0502782631758142\\
117	0.0524731706234338\\
118	0.0538596455697209\\
119	0.0537882711177494\\
120	0.053106143340585\\
121	0.0501761765365575\\
122	0.0458615750708306\\
123	0.0417706126584333\\
124	0.0355389825524583\\
125	0.0307192892420773\\
126	0.0241098302183228\\
127	0.0205129362482761\\
128	0.0182624496779699\\
129	0.0166859171737545\\
130	0.0163371823556143\\
131	0.0135558196377312\\
132	0.015434969699541\\
133	0.0147420004809081\\
134	0.0194950551953673\\
135	0.0143976245611289\\
136	0.0178137504175469\\
137	0.0121813528173853\\
138	0.0118505002855049\\
139	0.00784593120622962\\
140	0.0087586334269154\\
141	0.00514984843474172\\
142	0.00931516863003828\\
143	0.00333596669796592\\
144	0.0118127871467371\\
145	0.00317696404243787\\
146	0.0070180298065586\\
147	0.00856370122523116\\
148	0.00778616679507168\\
149	0.0168052353004226\\
150	0.0170464080149598\\
151	0.0244657816826504\\
152	0.0250322716643123\\
153	0.0288084536287228\\
154	0.0301447197435599\\
155	0.0298925479374944\\
156	0.031571213889459\\
157	0.0302179349014429\\
158	0.0276415457574005\\
159	0.0268550769762252\\
160	0.0241628752479281\\
161	0.0248221089922406\\
162	0.0237263801499888\\
163	0.0232435172014448\\
164	0.0228694866495597\\
165	0.0197956611823989\\
166	0.0215828841109874\\
167	0.0154297927378728\\
168	0.0171045896164251\\
169	0.0110739034477453\\
170	0.0165498436531877\\
171	0.0118509152290216\\
172	0.0150158230117284\\
173	0.0126105873445025\\
174	0.0143836832190725\\
175	0.0172369997309291\\
176	0.0161800817372619\\
177	0.0217903737908044\\
178	0.0187028925398924\\
179	0.0193053899435908\\
180	0.0175349794772118\\
181	0.0156535168926773\\
182	0.0173251386156446\\
183	0.014734292014621\\
184	0.0189940372945418\\
185	0.0135742880594327\\
186	0.0150097647832906\\
187	0.0120145418033782\\
188	0.0156597277142317\\
189	0.0100436915726616\\
190	0.0153367950079433\\
191	0.0081847238237242\\
192	0.0131647799228745\\
193	0.00713372030235874\\
194	0.0115402572567243\\
195	0.00538807583850811\\
196	0.0111645352327093\\
197	0.00366349321819937\\
198	0.0130937695653275\\
199	0.00262834093752909\\
200	0.0132943832200553\\
201	0.00279507973238776\\
202	0.0135513197993295\\
203	0.00622623888496137\\
204	0.0121306852887091\\
205	0.00945278270298698\\
206	0.0130397666635178\\
207	0.00868719597129941\\
208	0.0121392457597457\\
209	0.00631838297636033\\
210	0.0103069110233111\\
211	0.00390255597467399\\
212	0.00800188183970851\\
213	0.0063985193699412\\
214	0.00886829132318413\\
215	0.00485320006473724\\
216	0.0096459586417473\\
217	0.0050478127582701\\
218	0.0137134024404391\\
219	0.00774881077815155\\
220	0.014882881475271\\
221	0.0102995991953371\\
222	0.0125205137590303\\
223	0.0130755686922646\\
224	0.0114357864903448\\
225	0.013493126298498\\
226	0.0116821151656874\\
227	0.0157022297014039\\
228	0.00898400122845476\\
229	0.0143668169504703\\
230	0.00642588355020322\\
231	0.0117463539257711\\
232	0.0036083902365571\\
233	0.00665382770958505\\
234	0.00829305505101219\\
235	0.00424462185273305\\
236	0.0119320264095719\\
237	0.00341423070611129\\
238	0.0125246881648729\\
239	0.00346694967372778\\
240	0.0129436715825446\\
241	0.0137113444397113\\
242	0.0228660561147383\\
243	0.0238040678023016\\
244	0.0289633227611837\\
245	0.0309018657207819\\
246	0.0306834891446684\\
247	0.0331658136079921\\
248	0.029916371195686\\
249	0.0274046821322429\\
250	0.0250811210089786\\
251	0.0205270012371196\\
252	0.0195035090983249\\
253	0.0128222445902479\\
254	0.0146586511402593\\
255	0.00906142336096963\\
256	0.0152673256497343\\
257	0.00679839019984821\\
258	0.0130418551663956\\
259	0.00783734105875019\\
260	0.0132229172758171\\
261	0.0131761948498955\\
262	0.0148730029091581\\
263	0.0175534636110662\\
264	0.0156306967107358\\
265	0.020837137163826\\
266	0.015582862087374\\
267	0.0188094347738465\\
268	0.0178710796503145\\
269	0.0182235792040085\\
270	0.0190914284421698\\
271	0.0164485520693112\\
272	0.0179796194540936\\
273	0.0156699337566212\\
274	0.0170872768486505\\
275	0.0136930500756367\\
276	0.0152718247479607\\
277	0.0114095338648581\\
278	0.0127889181060989\\
279	0.007721800675178\\
280	0.0110296918469432\\
281	0.0048662523315244\\
282	0.012285067463391\\
283	0.00347083571268358\\
284	0.0132396357498846\\
285	0.00565974331419233\\
286	0.0142003301797304\\
287	0.00888625424219211\\
288	0.0149885458857704\\
289	0.0126137096112322\\
290	0.0192639147995249\\
291	0.0150221658322417\\
292	0.0188181698603077\\
293	0.0132460505100085\\
294	0.0138549562689327\\
295	0.00958053288072613\\
296	0.0108842263186179\\
297	0.00937208345216\\
298	0.00958961196666105\\
299	0.0112772451002198\\
300	0.00896994756976225\\
301	0.0139627381127892\\
302	0.0128217858909009\\
303	0.0191631504813074\\
304	0.0182569014909901\\
305	0.0205356221733808\\
306	0.0232408803634404\\
307	0.0211553169696535\\
308	0.0226258534231264\\
309	0.0232517811247563\\
310	0.0234015612432189\\
311	0.0266432479450701\\
312	0.023474472788402\\
313	0.0222808539448725\\
314	0.0224593802066967\\
315	0.0218155664192784\\
316	0.0217922496467115\\
317	0.0186063514006855\\
318	0.0190507472112569\\
319	0.0158113496305015\\
320	0.0175326300973357\\
321	0.0136410925704033\\
322	0.0174557371681112\\
323	0.0129170324384068\\
324	0.0161443293927599\\
325	0.0128010695480249\\
326	0.0137609896554246\\
327	0.0149259359845725\\
328	0.0129379228148737\\
329	0.0173526278337397\\
330	0.0162073634691127\\
331	0.0207356235257507\\
332	0.0197666578358753\\
333	0.0222290249782682\\
334	0.0168244441959226\\
335	0.0184939063445173\\
336	0.0114494184950896\\
337	0.0136365656839496\\
338	0.0065619316957567\\
339	0.0139503079764484\\
340	0.00953850628464023\\
341	0.0143224035533677\\
342	0.0143465595001058\\
343	0.0171891344827632\\
344	0.0222964301751143\\
345	0.0225096861368335\\
346	0.0274092693391053\\
347	0.0285630607412954\\
348	0.0273495398097827\\
349	0.0288667701648405\\
350	0.0260277100403766\\
351	0.0244540408562652\\
352	0.0280040339910838\\
353	0.0237052832654751\\
354	0.0224894399101512\\
355	0.0214235628919951\\
356	0.0206677587428542\\
357	0.0214921630069158\\
358	0.0202143923479318\\
359	0.0205391095481278\\
360	0.0197608397932811\\
361	0.0213896763789717\\
362	0.020067189435404\\
363	0.0225287315164271\\
364	0.0191312965308041\\
365	0.020940674798267\\
366	0.0196926262877675\\
367	0.0186774553470405\\
368	0.0218028240037817\\
369	0.0191038690506234\\
370	0.0208172458911241\\
371	0.0201715733052612\\
372	0.0221748402448356\\
373	0.0235828025762012\\
374	0.0251370954718872\\
375	0.0260519292037031\\
376	0.0235777184318858\\
377	0.0249830282531989\\
378	0.0202539368179157\\
379	0.0203108430789914\\
380	0.0181226715209524\\
};
\end{axis}
\end{tikzpicture}%}
      \caption{The discrepancy in distance between the trajectory of the vehicle
        and the reference trajectory for the entire sequence of approaching
        and maintaining the reference trajectory.}
      \label{fig:circular_mpc_error_full}
    \end{figure}
  \end{minipage}
\end{minipage}
}\\

The pairs of figures \ref{fig:circular_mpc_trajectory_transient}, \ref{fig:circular_mpc_error_transient}
and \ref{fig:circular_mpc_trajectory_ss}, \ref{fig:circular_mpc_error_ss}
feature the breaking down of the entire sequence into the transient phase
and the steady state. Notably, in steady state, in this configuration, the
error in displacement never exceeds $3.5$ cm. Further tuning, and re-evaluation
of critical modules that deliver the composite functionality of this ROS
package could improve its performance.


\noindent\makebox[\linewidth][c]{%
\begin{minipage}{\linewidth}
  \begin{minipage}{0.45\linewidth}
    \begin{figure}[H]
      \scalebox{0.6}{\input{./figures/results/circular_mpc/trajectory_transient.tex}}
      \caption{Reference trajectory (red) and trajectory of the vehicle (blue),
        in the transient phase.}
      \label{fig:circular_mpc_trajectory_transient}
    \end{figure}
  \end{minipage}
  \hfill
  \begin{minipage}{0.45\linewidth}
    \begin{figure}[H]
      \scalebox{0.6}{\input{./figures/results/circular_mpc/error_transient.tex}}
      \caption{The discrepancy in distance between the trajectory of the vehicle
        and the reference trajectory in the transient phase.}
      \label{fig:circular_mpc_error_transient}
    \end{figure}
  \end{minipage}
\end{minipage}
}\\

\noindent\makebox[\linewidth][c]{%
\begin{minipage}{\linewidth}
  \begin{minipage}{0.45\linewidth}
    \begin{figure}[H]
      \scalebox{0.6}{\input{./figures/results/circular_mpc/trajectory_ss.tex}}
      \caption{Reference trajectory (red) and trajectory of the vehicle (blue),
        in steady state.}
      \label{fig:circular_mpc_trajectory_ss}
    \end{figure}
  \end{minipage}
  \hfill
  \begin{minipage}{0.45\linewidth}
    \begin{figure}[H]
      \scalebox{0.6}{\input{./figures/results/circular_mpc/error_ss.tex}}
      \caption{The discrepancy in distance between the trajectory of the vehicle
        and the reference trajectory in steady state.}
      \label{fig:circular_mpc_error_ss}
    \end{figure}
  \end{minipage}
\end{minipage}
}\\


Figures \ref{fig:circular_mpc_input_transient} and \ref{fig:circular_mpc_input_ss}
show the result of the optimization process, that is, the steering angle
applied to the front wheels of the vehicle during the transient and steady state
respectively.

\noindent\makebox[\linewidth][c]{%
\begin{minipage}{\linewidth}
  \begin{minipage}{0.45\linewidth}
    \begin{figure}[H]
      \scalebox{0.6}{% This file was created by matlab2tikz.
%
%The latest updates can be retrieved from
%  http://www.mathworks.com/matlabcentral/fileexchange/22022-matlab2tikz-matlab2tikz
%where you can also make suggestions and rate matlab2tikz.
%
\definecolor{mycolor1}{rgb}{0.00000,0.44700,0.74100}%
%
\begin{tikzpicture}

\begin{axis}[%
width=4.133in,
height=3.26in,
at={(0.693in,0.44in)},
scale only axis,
xmin=0,
xmax=150,
xmajorgrids,
ymin=-40,
ymax=80,
ymajorgrids,
axis background/.style={fill=white}
]
\addplot [color=mycolor1,solid,forget plot]
  table[row sep=crcr]{%
1	19.6486527797331\\
2	13.5864193441453\\
3	2.39719554842419\\
4	4.15103722279149\\
5	5.26509998680217\\
6	11.6030459610686\\
7	4.94047689036392\\
8	3.4695057929254\\
9	5.82565403887194\\
10	11.3127928034206\\
11	14.3784139293541\\
12	2.62011563401707\\
13	6.70267679528461\\
14	1.00281616462579\\
15	2.77719419011491\\
16	3.4746171223021\\
17	2.79139244585208\\
18	6.76613650975002\\
19	10.0864664888907\\
20	7.62216838729595\\
21	3.72720007913011\\
22	1.89748759624525\\
23	2.18660181984907\\
24	4.10291170989958\\
25	1.85421237143769\\
26	0.958909040598308\\
27	4.28059865427064\\
28	3.87202484897012\\
29	9.38072524006808\\
30	8.52935411285704\\
31	13.4196602755594\\
32	5.63377908404747\\
33	59.9999401979496\\
34	59.9999948394629\\
35	60.0000016696521\\
36	60.0000016696521\\
37	60.0000016696521\\
38	60.0000016696521\\
39	60.0000016696521\\
40	59.9999948394629\\
41	60.0000016696521\\
42	60.0000016696521\\
43	49.8467922797393\\
44	35.0176558746034\\
45	20.1323616541115\\
46	2.49177424507842\\
47	-15.6955407787102\\
48	-28.7198628749202\\
49	-32.0336384337786\\
50	-31.963710957056\\
51	-32.5297150731966\\
52	-6.16283187677832\\
53	22.6483677028179\\
54	44.6741978137349\\
55	43.7945719014748\\
56	57.7498095079698\\
57	49.8519012612385\\
58	52.4609720524809\\
59	43.6392294940866\\
60	41.4081140507561\\
61	29.2391587424217\\
62	12.2002657492634\\
63	9.39260720790297\\
64	8.46646087720592\\
65	16.1971396260722\\
66	24.3424714754872\\
67	30.8920815970377\\
68	34.0327084453417\\
69	33.1765578932589\\
70	33.3066764120422\\
71	31.0537009482731\\
72	32.771923826449\\
73	31.4436808442187\\
74	30.5261166462138\\
75	30.7069254139224\\
76	29.9504751331543\\
77	28.8816803016415\\
78	29.3433771838723\\
79	28.7497210468769\\
80	29.4820027032669\\
81	29.2972802371639\\
82	29.4143189436866\\
83	30.4425868477591\\
84	30.0116087413205\\
85	31.4603567510773\\
86	30.7024209041648\\
87	30.1204068246096\\
88	31.4262672769298\\
89	30.9631804512029\\
90	31.0178868513601\\
91	30.9203005235936\\
92	32.0600541903936\\
93	31.4516789957368\\
94	31.8466722505332\\
95	31.8718107617735\\
96	32.3295563795694\\
97	33.030989486573\\
98	32.8742024941754\\
99	33.1396099849437\\
100	33.5053392942412\\
101	32.706336934944\\
102	33.7022809687694\\
103	32.2811542440161\\
104	31.3254127036454\\
105	32.1409407056395\\
106	32.0613314357684\\
107	34.9926198162007\\
108	33.7435489717346\\
109	32.2900198295588\\
110	27.5192812509193\\
111	27.6422827125875\\
112	26.6800593226944\\
113	27.7002078393911\\
114	26.9994970248922\\
115	27.985122350429\\
116	28.8934896987165\\
117	28.1831261193731\\
118	29.7229342111445\\
119	29.9053685638755\\
120	29.1905653615716\\
121	30.4004309202018\\
122	30.4878334359159\\
123	29.6691806223765\\
124	30.6207523322589\\
125	30.9471978085451\\
126	30.2356082102456\\
127	30.5736854986884\\
128	29.1566398119641\\
129	30.0317236484262\\
130	28.8144746553032\\
131	29.7579833268704\\
132	28.3440931875428\\
133	29.2432568559237\\
134	29.3068527472857\\
135	29.0508060307748\\
136	31.238119470958\\
137	28.7507933865766\\
138	31.3755326317749\\
139	28.3877739548323\\
140	31.2549114910325\\
141	29.9561783411113\\
142	29.8027893678257\\
143	33.0269869957194\\
144	32.3589500986625\\
145	30.9571801300171\\
146	28.9398632680863\\
147	28.5467431926697\\
148	27.8882958812076\\
149	27.591894699533\\
150	28.1954819315816\\
};
\end{axis}
\end{tikzpicture}%}
      \caption{Input through time during the transient phase.}
      \label{fig:circular_mpc_input_transient}
    \end{figure}
  \end{minipage}
  \hfill
  \begin{minipage}{0.45\linewidth}
    \begin{figure}[H]
      \scalebox{0.6}{% This file was created by matlab2tikz.
%
%The latest updates can be retrieved from
%  http://www.mathworks.com/matlabcentral/fileexchange/22022-matlab2tikz-matlab2tikz
%where you can also make suggestions and rate matlab2tikz.
%
\definecolor{mycolor1}{rgb}{0.00000,0.44700,0.74100}%
%
\begin{tikzpicture}

\begin{axis}[%
width=4.133in,
height=3.26in,
at={(0.693in,0.44in)},
scale only axis,
xmin=0,
xmax=250,
xmajorgrids,
xlabel={time [samples]},
ymin=27,
ymax=34,
ymajorgrids,
ylabel={steering angle [degrees]},
axis background/.style={fill=white}
]
\addplot [color=mycolor1,solid,forget plot]
  table[row sep=crcr]{%
1	28.1954819315816\\
2	27.6698852145708\\
3	28.5558495423807\\
4	30.076994142245\\
5	29.2949955388865\\
6	30.2014572643956\\
7	30.5773054989485\\
8	29.7805127058477\\
9	30.7967833826431\\
10	30.842719819906\\
11	29.2045706644647\\
12	29.1420232071403\\
13	30.1045027291273\\
14	29.3781291863693\\
15	28.9452591175306\\
16	29.4971008364272\\
17	28.729640290717\\
18	31.0522187972232\\
19	29.7954093484275\\
20	29.5932938056031\\
21	29.4865755149162\\
22	29.9443996798876\\
23	30.1979226415001\\
24	29.4804898163657\\
25	28.8653971306602\\
26	29.4316642090839\\
27	29.5589994257804\\
28	29.159908057482\\
29	30.1221963341722\\
30	30.3812619942962\\
31	29.2045262682351\\
32	29.1905687766662\\
33	29.2760281035614\\
34	30.165837827874\\
35	29.3407851270823\\
36	30.3070007625452\\
37	29.4825047221709\\
38	29.6212224491192\\
39	29.8241439542657\\
40	28.9187750590239\\
41	30.0587233862152\\
42	29.1163485260502\\
43	28.822500127578\\
44	29.6997867000473\\
45	28.7103108553659\\
46	30.7655318520957\\
47	29.4276412276627\\
48	30.5229064573039\\
49	29.3573346754412\\
50	30.141150109119\\
51	29.1122777333049\\
52	28.6740391357786\\
53	29.6446295074049\\
54	28.742716687883\\
55	28.1140916898844\\
56	30.4368836398022\\
57	28.1786711284869\\
58	29.4107570000345\\
59	28.5300263046761\\
60	30.9747849426028\\
61	28.4177397022684\\
62	31.2923067767383\\
63	28.0250311457494\\
64	29.6987416811043\\
65	29.3105922758562\\
66	30.329410613212\\
67	29.1429316222999\\
68	29.7915366311681\\
69	28.792471200892\\
70	29.711845399027\\
71	30.0543384047681\\
72	29.4507955689491\\
73	30.0396056867284\\
74	29.2714860277633\\
75	28.7729812560954\\
76	29.5584700861198\\
77	30.062247763827\\
78	29.6660250749799\\
79	30.4912553606901\\
80	28.8237124861556\\
81	29.9508029822345\\
82	28.6163547731432\\
83	29.928686829702\\
84	29.073130504077\\
85	32.3579938721787\\
86	31.8158032105794\\
87	32.8570484740749\\
88	33.811420561517\\
89	29.9902712303534\\
90	27.9120820149922\\
91	28.5287080781663\\
92	27.3934059871578\\
93	27.2837268094658\\
94	28.2410785669333\\
95	30.2193762656831\\
96	28.3314368470107\\
97	29.2995376146846\\
98	28.4116796169273\\
99	29.9082987150295\\
100	29.828733841388\\
101	29.2885922365396\\
102	30.2222620206074\\
103	29.210847608312\\
104	30.3533777470096\\
105	28.8170632969986\\
106	29.6859111707484\\
107	30.9332812981112\\
108	29.8430533329829\\
109	30.6051760858567\\
110	27.9993342665445\\
111	30.3688481254797\\
112	28.790545087546\\
113	30.1818477912885\\
114	29.1950186449104\\
115	30.0309962332796\\
116	30.2410996823383\\
117	28.2509550204732\\
118	29.348967693708\\
119	29.7159640030965\\
120	30.4249478842276\\
121	29.5303741029689\\
122	29.5393148205925\\
123	28.7783771055397\\
124	29.4838263637753\\
125	28.7944826916026\\
126	29.7438755711397\\
127	28.9396071359924\\
128	29.8057536699255\\
129	29.1877342481606\\
130	31.1566523896327\\
131	29.4652209284762\\
132	30.6538070327472\\
133	29.4239153594705\\
134	30.1249147494618\\
135	29.0710097303397\\
136	28.7647167271996\\
137	29.2791358396337\\
138	28.7451721408897\\
139	28.0725760925618\\
140	28.6414932843834\\
141	28.2921137404117\\
142	29.1607959820741\\
143	29.4965168552532\\
144	30.2524514567389\\
145	30.7748516452182\\
146	28.9376332113223\\
147	29.8056921982229\\
148	28.5617832692221\\
149	30.6339960690596\\
150	29.0753400702735\\
151	30.1764246210875\\
152	28.5460396831852\\
153	29.4057777921295\\
154	29.7134675689549\\
155	28.9696736287188\\
156	30.0829159162554\\
157	29.0634521260231\\
158	29.5185203096643\\
159	29.9027594316126\\
160	28.9738468743017\\
161	30.0686818020251\\
162	30.389871447745\\
163	29.9258079049668\\
164	30.8526406696754\\
165	28.7644674252949\\
166	29.8044286132265\\
167	28.8384042230603\\
168	29.9429072835539\\
169	29.1255692814298\\
170	29.5457181229393\\
171	28.8772714145323\\
172	29.7926670274757\\
173	29.4648008718422\\
174	30.2776958359113\\
175	30.5317891183195\\
176	29.4832082316554\\
177	30.3622330872685\\
178	29.4022226786665\\
179	30.1661861675216\\
180	28.2099721779058\\
181	29.1175233185875\\
182	27.8760835029716\\
183	29.6101507124747\\
184	28.5432324754363\\
185	29.5022747047235\\
186	30.1227085983599\\
187	30.6148203129648\\
188	29.8839115245979\\
189	29.0882901089398\\
190	29.4399526436417\\
191	29.5400900470632\\
192	29.8922989968987\\
193	29.0500820307228\\
194	29.4646232869238\\
195	28.4345624581942\\
196	29.2329569306553\\
197	28.7106523648244\\
198	29.6304363743096\\
199	29.9479991895802\\
200	30.2428584560496\\
201	29.4835941373435\\
202	30.1910992825193\\
203	30.7316746043799\\
204	29.9743637197764\\
205	30.9637780927553\\
206	28.8813456223722\\
207	30.0508618384806\\
208	28.5234659079783\\
209	29.5669224452176\\
210	28.5084907182231\\
211	29.8809745432548\\
212	28.312465996591\\
213	29.5766930308253\\
214	28.4149154190466\\
215	30.3510213317459\\
216	30.3668673706204\\
217	29.7651242896477\\
218	30.3138616875665\\
219	29.1758223982481\\
220	30.2603130044736\\
221	30.3630185590231\\
222	28.5905093373239\\
223	29.4244890953608\\
224	28.1869885913487\\
225	29.5344278202413\\
226	28.5509181457999\\
227	29.4472848517157\\
228	29.7796930831473\\
229	30.2951094032002\\
230	29.6247639022039\\
231	28.7958760501932\\
};
\end{axis}
\end{tikzpicture}%
}
      \caption{Input through time during steady state.}
      \label{fig:circular_mpc_input_ss}
    \end{figure}
  \end{minipage}
\end{minipage}
}
